\documentclass[twoside]{article}
\usepackage[paperheight=395pt, paperwidth=240pt, width=166pt, height=308pt, top=42pt, headsep=2pt]{geometry}
\usepackage[noinfo, frame, a6, center, lualatex]{crop}
\usepackage{graphicx}
\usepackage{shandean}
%
%
\title{A Sentimental Journey through France and Italy}
\author{Laurence Sterne}
\date{1768}
\begin{document}
\raggedbottom
\def\vol{I}
%------------------
\pagenumbering{roman}
\pagestyle{empty}
\centerline{A}
\centerline{\tstrut 28pt\large \lsv{24}{SENTIMENTAL\enspace JOURNEY}}
\centerline{\tstrut 24pt\small \lsv{36}{\textsc{through}}}
\centerline{\tstrut 28pt\lsss{\Large FRANCE}\quad\lsss{AND}\quad\lsss{\Large ITALY}.}
\centerline{\tstrut 32pt\small \lsv{36}{\textsc{by}}}
\centerline{\tstrut 22pt\large \lsv{36}{\textsc{Mr}}.\quad \lsv{48}{YORICK}.}

\vskip 36pt
\hrule
\centerline{\tstrut 16pt\large \lsv{44}{VOL}.\quad I.}
\vskip 8pt
\hrule
\vskip 2pt
\hrule
\centerline{\tstrut 54pt\small \lsv{72}{LONDON}:}
\centerline{\tstrut 18pt\smaller Printed for T. \lsv{24}{\scshape Becket}\ and\ P.\@
A.\@ \lsv{24}{\scshape De Hondt},}
\centerline{\tstrut 10pt\smaller in the Strand.\enspace MDCCLXVIII.}
\newpage
\null
\newpage
\null\vskip 1in
\centerline{\lsv{10}{A}}
\centerline{\tstrut 24pt\large\lsv{20}{SENTIMENTAL JOURNEY,}}
\centerline{\tstrut 20pt\large \&c.\enspace \&c .}
\vfill
\indent\s{Vol}. I.
\newpage
\null
\newpage
\pagestyle{folio}
\thispagestyle{empty}

\hrule
\vskip 2pt
\hrule
\vskip 28pt
\stick{\large\lsv{76}{SUBSCRIBERS}.}
\vskip 16pt
\centerline{\fnast\enspace\i{Imperial Paper}.}
\vskip 20pt
\hrule
\bgroup\fontsize{10}{14}\selectfont

\head{36pt}{A}

\dropcap[findent=0pt]{E}{arl} of Abingdon \fnast\\
Lord Ancram \fnast\\
Rev. Mr. Arden \fnast\\
Lieutenant-General Armiger\\
Mr. T. Atkison \fnast\\
Mr. Aubrey \fnast

\head{18pt}{B}
\vskip -\baselineskip
\noindent
Duke of Buccleugh\\
Lord Viscount Bolingbroke \fnast\\
Mr. Beauclerc \fnast

\patchv{a}{Sir}
\noindent
Sir W. Boothby \fnast\\
Mr. Boothby \fnast\\
Lord Belasyse \fnast\\
Lady Belasyse \fnast\\
Mr. Bonfoy\\
Mr. Brand \fnast\\
Mr. Bourk\\
Mr. Hawkins Brown\\
Lord Bathurst \fnast\\
Mr. Buller\\
Lord Burgheartst \fnast\\
Mr. Bayly\\
Doctor Brook\\
Mr. Blakeney\\
Mr. T. Brown\\
Mr. Big\\
Honourable Mr. Belasyse\\
Mr. Barton \fnast\\
Mr. Barker Devonshire\\
Mr. Bromley\\
Mr. Joseph Brereton \\
\catch{Mr.}
\noindent
Mr. Baskerville\\
Earl of Berkeley\\
Earl of Barrymore\\
Mrs. Barry\\
Rev. Mr. Thomas Bushe\\
Mr. Amyas Bushe\\
Rev. Mr. Blake\\
Mr. Jervais Busk\\
Mr. Bonfoy\\
Mr. Barker \fnast\\
Mr. Braithwait\\

\head{18pt}{C}
\vskip-\baselineskip\noindent
Mr. Crew, 20 \i{sets} \fnast\\
Mr. Cadogan \fnast\\
Mr. Crowl \fnast\\
Mr. Crawford \fnast\\
Mr. Chad\\
Mr. Crawley\\
Mr. Cæsar\\
Rev. Mr. Cockayne\\
\catchs{a 2}{Lord}

\noindent
Lord Charlemont\\
Captain Crawford \fnast\\
Mr. Cust\\
Mr. Cane\\
Lord Clanbrazel \fnast\\
Mr. Clavering\\
Lord Frederick Cavendish \fnast\\
Lord Cornwallis \fnast\\
Lord John Cavendish \fnast\\
Rev. Mr. Cleavor\\
Mr. Cross\\
Rev. Mr. Cayly, Residentiary of York\\
The Bishop of Cork \fnast\\
Mr. Colman\\
Mrs. Chaloner

\head{18pt}{D}
\vskip-\baselineskip\noindent
Captain Digby \fnast\\
Mr. Dempster\\
Hon. John Demer \fnast\\
Mr. J. Demer \fnast\\
\catch{Sir}
Sir Charles Davers\\
Mrs. Draper, 3 \i{sets} \fnast\\
Mr. Dunbar \fnast\\
Mr. J. Dillon \fnast\\
Mr. Dillon \fnast\\
Mr. H. Duncombe\\
Mr. Dundas \fnast\\
Mr. Eleazer Davy\\
Mr. Richard Davenport \fnast

\head{12pt}{E}
\vskip-\baselineskip\noindent
Mr. G. Eddison\\
Mr. Eftwick\\
Mr. Earl \fnast\\
Rev. Mr. Egerton

\head{12pt}{F}
\vskip-\baselineskip\noindent
Lord Farnham \fnast\\
Lord Fauconberg \fnast\\
\stick{Mons.\@ O'Flannagan, Col.\@ au service de leurs\hss}
\indent Majesties Imp.\\
\catchs{a 3}{J.}
J. Foley, Esq; \fnast\\
Mr. Furye\\
Mr. Fennick\\
Mr. Falkner\\
Mr. Fitzhue\\
Mr. Fothergill\\
Mr. Flood\\
Mr. William Fowler\\
The Hon. Mr. Fitzmorris \fnast\\
Mr. Frere\\
Mr. J. Fuller\\
Mr. Fonnereau

\head{18pt}{G}
\vskip-\baselineskip\noindent
The Duke of Grafton \fnast\\
The Duchess of Grafton \fnast\\
Lord Galway\\
Miss Godfrey \fnast\\
Mr. Garrick \fnast\\
Sir Sampson Gideon \fnast\\
Mr. P. Gibbes \fnast\\
\catch{Lord}
Lord William Gordon\\
Mrs. Goldsworthy\\
Marquis of Granby\\
Sir Alexander Gilmour \fnast\\
Mr. Griffith\\
J. Garland, Esq; \fnast\\
Mr. Garland \fnast\\
Miss Gore

\head{18pt}{H}
\vskip-\baselineskip\noindent
Mr. Heron \fnast\\
Charles Howard, Esq; \fnast\\
Mr. J. Z. Holwell\\
Mr. G. Hart, Pall-Mall\\
Mr. T. Hunt, 2 \i{sets}\\
Mr. Jacob Houbler\\
Mr. Hesselridge\\
Hon. Captain Hervey\\
Mr. Heber \fnast\\
Mr. Hunter \fnast\\
Mr. Hill \fnast\\
\catchs{a 4}{Mr.}
Mr. C. Hanbury \fnast\\
Mr. O. Hanbury \fnast\\
Hon. N. Herbert \fnast\\
Henry Herbert, Esq;\\
Doctor G. Hay \fnast\\
Mrs. Hoan\\
Captain D. Harvey\\
Governor Hamilton \fnast

\head{18pt}{J}
\vskip-\baselineskip\noindent
Mr. James \fnast\\
Hon. Miss Ingram\\
Mr. Thomas Jones\\
M. Julius\\
The Hon. John St. John

\head{18pt}{K}
\vskip-\baselineskip\noindent
Sir John Kay\\
Doctor Knox\\
Joseph Kuling, Esq;\\
Doctor Kilvington\\
\catch{Colonel}

\head{18pt}{L}
\vskip-\baselineskip\noindent
Colonel Lee\\
Mr. C. Ludwidge\\
Sir Matthew Lamb \fnast\\
Mrs. Lamb\\
Mr. J. Langlois\\
Mr. G. Litchfield\\
Mr. John Lowe\\
Lord G. Lenox \fnast\\
Earl of Lincoln \fnast\\
Mr. Peter Lascelles\\
Rev. Mr. Lascelles\\
Mr. Edwin Lascelles\\
Mr. Edward Lascelles\\
Mr. Daniel Lascelles\\
Mr. Her. Langrisk\\
Mr. Robert Lowther.

\head{18pt}{M}
\vskip-\baselineskip\noindent
Colonel Mackay \fnast\\
Sir George Macartney, 5 \i{sets} \fnast\\
\catch{Mr.}
Mr. Mannering\\
Lord Milton \fnast\\
Mrs. Mountague\\
Mrs. Menyl\\
The Duke of Montague \fnast\\
The Duchess of Montague \fnast\\
The Marquis of Monthidmer \fnast\\
Sir William Musgrave\\
Mr. Murray of Broughton \fnast\\
Lord Mount Stewart \fnast\\
Lady Mount Stewart \fnast\\
Sir Francis Mollineux\\
Mr. Minshin\\
Mrs. Minshin\\
Mr. Sawray Morrit\\
Rev. Mr. Marsden\\
Mr. Thomas Mather\\
Mr. Motteux\\
Mr. Maclean\\
Mr. George Morland\\
\catch{N.}

\head{18pt}{N}
\vskip-\baselineskip\noindent
Lord Newnham\\
Miss Anne Newnham \fnast\\
Miss Honoria Newnham \fnast\\
Mr. T. Newnham \fnast\\
Mr. N. Newnham, 2 \i{sets}\\
Mr. W. Newnham, 2 \i{sets}\\
Mr. Richard Norton\\
Hon. and Rev. Mr. Nevill\\
Mr. J. Norton

\head{18pt}{O}
\vskip-\baselineskip\noindent
Lord Ossory \fnast\\
Hon. Mr. Oglethorpe\\
Mr. Hugh Owen\\
Mr. Ogilby, 5 \i{sets} \fnast

\head{18pt}{P}
\vskip-\baselineskip\noindent
Mr. W. Pocock\\
Mr. Perrey\\
Mr. H. Pelham\\
Mr. Parker\\
\catch{Mr.}
Mr. Phipps\\
Mr. Pratt\\
R. Pigot, junior\\
Miss Purling\\
Monsieur Le Compte de Paar\\
Lord Portmore\\
Lord Pembroke\\
Lord Palmerston \fnast\\
Mr. Panchaude, 20 \i{sets}\\
The Bishop of Peterborough \fnast\\
Mr. Palmer\\
Mr. Porter\\
Mr. W. Price\\
Doctor Petit

\head{18pt}{Q}
\vskip-\baselineskip\noindent
The Duke of Queensberry \fnast

\head{18pt}{R}
\vskip-\baselineskip\noindent
The Duke of Rutland\\
The Duchess of Richmond \fnast\\
\catch{The}
The Duke of Roxburgh \fnast\\
Mr. Ruspini\\
Mr. Edward Rolf\\
Mr. Joseph Russel\\
Hon. F. Robinson\\
Mr. John Ransby, Lincoln's Inn\\
Lady Robinson\\
Marquis of Rockingham \fnast\\
Lady Rockingham \fnast.

\head{18pt}{S}
\vskip-\baselineskip\noindent
Mr. Robert Sparrow\\
Mr. Frank Schutze\\
Doctor Smalbrook \fnast\\
Lord Spencer \fnast\\
Mr. H. Stanley \fnast\\
\stck{Mr. Stanley, Commissioner of the Customs}\rlap{ \fnast}
Lord Charles Spencer\\
Mr. John Sivale\\
Hon. Mr. Shelly\\
Mr. Sikes, Hull\\
\catchs{4}{Lord}
Lord Shelburne \fnast\\
Mr. Stewart\\
Mr. Bladen Swiney\\
Mr. G. Selvin \fnast\\
Mr. Smith \fnast\\
Colonel Scot \fnast\\
Lord Strathmore \fnast\\
Lord Strafford\\
Doctor Swiney\\
Mr. Robert Smith, 2 \i{sets}\\
Mr. Sackville\\
Mr. J. Shaftoe \fnast\\
Sir F. Standish \fnast\\
Lady Standish \fnast\\
Mr. Salvadore\\
Mr. Skrine\\
Hon. Edward Stratford\\
Mr. T. Shadwell

\head{18pt}{T}
\vskip-\baselineskip\noindent
Lord Thanet \fnast\\
\catchs{7}{Mr.}
Mr. Thornhill \fnast\\
Mr. G. Thornhill \fnast\\
Mr. Trent\\
Lord Tyrone\\
Mr. F. Trotman\\
Mr. Joseph Tullie\\
Mr. Tankard\\
Mr. C. Turner\\
Mr. Tickle \fnast\\
Mr. Thorougton\\
Mr. C. Tomlins

\head{18pt}{V}
\vskip-\baselineskip\noindent
Honourable F. Vane\\
Mr. R. Vincent\\
Mr. Vane\\
Mr. Vesey\\
Lord Villars \fnast\\
Reverend Dr. Vane\\
Mr. Vernon \fnast\\
\catch{W.}

\head{18pt}{W}
\vskip-\baselineskip\noindent
Mr. John Wonen \fnast\\
Mr. Woodhouse, 2 \i{sets} \fnast\\
Sir Cecil Wray\\
Sir Rowland Winn\\
Mr. Weddle\\
Hon. Mr. Walsingham\\
Mr. Whitehead\\
Mr. Wanley\\
Mr. Wastall\\
Lord Wandersford\\
Mr. Whitwick\\
Mr. Nathaniel Webb

\head{18pt}{Y}
\vskip-\baselineskip\noindent
The Archbishop of York \fnast\\
Mr. R. Young

\egroup

\patch[48pt]{A}




\newpage
\fontsize{11}{15}\selectfont
\pagenumbering{arabic}
\thispagestyle{empty}
\hrule
\vskip 2pt
\hrule
\noindent
\stick{\tstrut 16pt\hfill{A}\hfill}
\stick{\tstrut 28pt{\large\ls{SENTIMENTAL\hfil JOURNEY,}}}
\stick{\tstrut 24pt\hfill{\&c.\enspace\&c.} \hfill}

\vskip 16pt

\dropcap[ante=\lower5pt\hbox{\tsh}]{T}{hey} order, said I, this\break
matter better in France\tsk

\tsk You have been in France? said 
my gentleman, turning quick upon 
me with the most civil triumph in 
the world.\tsk Strange! quoth I, de-
bating the matter with myself, That 
one and twenty miles sailing, for ’tis 
absolutely no further from Dover to 
\stick{Calais, should give a man these}
\catchv{B}{rights}
\stick{rights\tsk I’ll look into them: so giv-}
ing up the argument\tsk I went straight
to my lodgings, put up half a dozen
shirts and a black pair of silk breeches
\tsk \lqq the coat I have on, said I, look-
ing at the sleeve, will do\rqq\tsk took~a 
place in the Dover stage; and the
packet sailing at nine the next morn-
ing\tsk by three I had got sat down to 
\stick{my dinner upon a fricassee’d chicken}
so incontestably in France, that had 
I died that night of an indigestion, 
the whole world could not have sus-
\stick{pended the effects of the \fnast\ \i{Droits}}
\vfill %vskip 8pt
\bgroup\fontsize{9}{12}\selectfont\noindent
\stick{\enspace\fnast\enspace  All the effects of strangers (Swiss and}
Scotch excepted) dying in France, are seized 
by virtue of this law, tho’ the heir be upon 
the spot\tsh the profit of these contingencies 
being farm’d, there is no redress.\par\egroup
\catch{\i{d’aubaine}}
\i{d’aubaine}\tsk my shirts, and black pair 
of silk breeches\tsk portmanteau and 
all must have gone to the King of 
France\tsk even the little picture which 
I have so long worn, and so often 
have told thee, Eliza, I would carry 
with me into my grave, would have 
been torn from my neck.\tsk Ungene-
rous!\tsk to seize upon the wreck of an 
unwary passenger, whom your sub-
jects had beckon’d to their coast\tsk by 
\stick{heaven! \s{Sire}, it is not well done;} 
and much does it grieve me, ’tis the 
monarch of a people so civilized and 
courteous, and so renown’d for sen-
timent and fine feelings, that I have 
to reason with\tsh

But I have scarce set foot in your\break
dominions\tsh

\catchs{B 2}{}

\head{36pt}{CALAIS}
\vskip 2pt

\dropcap{W}{hen} I had finish’d my din-
ner, and drank the King\break
of France’s health, to satisfy my mind 
that I bore him no spleen, but, on the
contrary, high honour for the huma-
nity of his temper\tsk I rose up an inch
taller for the accommodation.

\tsk No\tsk said I\tsk the Bourbon is by\break
\stick{no means a cruel race: they may be}
\stick{misled like other people; but there is}
\stick{a mildness in their blood.  As I ac-}
\stick{knowledged this, I felt a suffusion of }
\stick{a finer kind upon my cheek\tsk more}
\stick{warm and friendly to man, than}
\catch{what}
\stick{what Burgundy (at least of two livres}
\stick{a bottle, which was such as I had}
been drinking) could have produced.

\tsk Just God! said I, kicking my 
\stick{portmanteau aside, what is there in}
\stick{this world’s goods which should}
\stck{sharpen our spirits, and make so many}
\stick{kind-hearted brethren of us, fall out}
so cruelly as we do by the way?

When man is at peace with man, how much lighter than a feather is the
heaviest of metals in his hand! he pulls out his purse, and holding it
airily and uncompress’d, looks round
him, as if he sought for an object to
share it with\tsk In doing this, I felt 
\catchs{B 3}{every}
\stick{every vessel in my frame dilate\tsk the}
\stick{arteries beat all chearily together,}
\stick{and every power which sustained life,}
\stick{perform’d it with so little friction,}
\stick{that ’twould have confounded the}
\stck{most \i{physical precieuse} in France: with}
\stick{all her materialism, she could scarce}
have called me a machine\tsk 

I’m confident, said I to myself, I\break
should have overset her creed.

The accession of that idea, carried 
\stick{nature, at that time, as high as she}
\stick{could go\tsk I was at peace with the}
\stick{world before, and this finish’d the}
treaty with myself\tsk 

\patch[\baselineskip]{\tsk Now,}

\tsk Now, was I a King of France, 
\stick{cried I\tsk what a moment for an or-}
\stick{phan to have begg’d his father’s}
portmanteau of me!

\patchs[14\baselineskip]{B 4}{}

\head{36pt}{THE\enspace MONK}
\head{16pt}{CALAIS}

\vskip 4pt

\initial{I}{HAD} scarce utter’d the words,\break
when a poor monk of the order\break
\stick{of St.~Francis came into the room to}
\stick{beg something for his convent.  No}
\stick{man cares to have his virtues the}
\stick{sport of contingencies\tsk or one man}
\stick{may be generous, as another man is}
\stick{puissant\tsk \i{sed non, quo ad hanc}\tsk or be}
\stick{it as it may\tsk for there is no regular}
\stick{reasoning upon the ebbs and flows of}
\stick{our humours; they may depend}
\stick{upon the same causes, for ought}
\stick{I know, which influence the tides}
\stick{themselves\tsk ’twould oft be no dis-}
\catch{credit}
\stick{credit to us, to suppose it was so:}
\stick{I’m sure at least for myself, that in}
\stick{many a case I should be more highly}
\stick{satisfied, to have it said by the world,}
\stick{\lqq I had had an affair with the moon,}
\stick{in which there was neither sin nor}
\stick{shame,\rqq than have it pass altogether}
\stick{as my own act and deed, wherein}
there was so much of both.

\tsk But, be this as it may. The\break
\stick{moment I cast my eyes upon him, I}
\stick{was predetermined not to give him}
\stick{a single sous; and accordingly I put}
\stick{my purse into my pocket\tsk button’d}
\stick{it up\tsk set myself a little more upon}
\stick{my centre, and advanced up gravely}
\stick{to him: there was something, I fear,}
\stick{forbidding in my look: I have his}
\catch{figure}
figure this moment before my eyes, and
think there was that in it which
deserved better.

The monk, as I judged from the\break
\stick{break in his tonsure, a few scatter’d}
\stick{white hairs upon his temples, being}
\stick{all that remained of it, might be}
\stick{about seventy\tsk but from his eyes,}
\stick{and that sort of fire which was in}
\stick{them, which seemed more temper’d}
\stick{by courtesy than years, could be no}
\stick{more than sixty\tsk Truth might lie}
\stck{between\tsk He was certainly sixty-five;}
\stick{and the general air of his counte-}
\stick{nance, notwithstanding something}
\stck{seem’d to have been planting wrinkles}
\stick{in it before their time, agreed to the}
account.

\patch{It}

It was one of those heads, which\break
\stick{Guido has often painted\tsk mild, pale}
\stick{\tsk penetrating, free from all com-}
\stick{mon-place ideas of fat contented ig-}
\stck{norance looking downwards upon the}
\stck{earth\tsk it look’d forwards; but look’d,}
\stick{as if it look’d at something beyond}
\stick{this world. How one of his order}
\stick{came by it, heaven above, who let}
\stick{it fall upon a monk’s shoulders, best}
\stick{knows: but it would have suited a}
\stick{Bramin, and had I met it upon the}
\stick{plains of Indostan, I had reverenced}
it.

The rest of his outline may be\break
\stick{given in a few strokes; one might}
\stick{put it into the hands of any one to}
\stick{design, for ’twas neither elegant or}
\catch{otherwise,}
\stick{otherwise, but as character and ex-}
\stick{pression made it so: it was a thin,}
\stick{spare form, something above the}
\stick{common size, if it lost not the dis-}
\stick{tinction by a bend forward in the}
\stick{figure\tskk but it was the attitude of}
\stick{Intreaty; and as it now stands pre-}
\stick{sented to my imagination, it gain’d}
more than it lost by it.

When he had enter’d the room\break
\stick{three paces, he stood still; and lay-}
\stick{ing his left hand upon his breast, (a}
\stick{slender white staff with which he}
\stick{journey’d being in his right)\tsk when}
\stick{I had got close up to him, he intro-}
\stick{duced himself with the little story of}
\stick{the wants of his convent, and the}
\stick{poverty of his order\tsk and did it with}
\catch{so}
\stick{so simple a grace\tsk and such an air}
\stick{of deprecation was there in the whole}
\stick{cast of his look and figure\tsk I was be-}
\stick{witch’d not to have been struck with}
it\tsk

\tsk A better reason was, I had pre-\break
\stick{determined not to give him a single}
sous.

\newpage


\head{36pt}{THE MONK}
\head{16pt}{CALAIS}

\vskip 4pt

\dropcap[ante=\smash{\lower6pt\hbox{\tsk}}]{’T}{is}
very true, said I, replying to a cast upwards with 
\stick{his eyes, with which he had con-}
\stick{cluded his address\tsk ’tis very true\tsk}
\stick{and heaven be their resource who}
\stick{have no other but the charity of the}
\stick{world, the stock of which, I fear, is}
\stick{no way sufficient for the many \i{great}}
\stick{\i{claims} which are hourly made upon}
it.

As I pronounced the words \i{great}
\stick{\i{claims}, he gave a slight glance with}
\stick{his eye downwards upon the sleeve}
\catch{of}
\stick{of his tunick\tsk I felt the full force of}
\stick{the appeal\tsk I acknowledge it, said I}
\stick{\tsk a coarse habit, and that but once}
\stick{in three years, with meagre diet\tsk}
\stick{are no great matters; and the true}
\stick{point of pity is, as they can be earn’d}
\stick{in the world with so little industry,}
\stick{that your order should wish to pro-}
\stick{cure them by pressing upon a fund}
\stick{which is the property of the lame, the}
\stick{blind, the aged and the infirm\tsk the}
\stick{captive who lies down counting over}
\stick{and over again the days of his af-}
\stick{flictions, languishes also for his share}
\stick{of it; and had you been of the \i{order}}
\stick{\i{of mercy}, instead of the order of St.}
\stick{Francis, poor as I am, continued I,}
\stick{pointing at my portmanteau, full}
\stick{chearfully should it have been open’d}
\catch{to}
\stick{to you, for the ransom of the unfor-}
\stick{tunate\tsk The monk made me a bow}
\stick{\tsk but of all others, resumed I, the}
\stick{unfortunate of our own country,}
\stick{surely, have the first rights; and I}
\stick{have left thousands in distress upon}
\stick{our own shore\tsk The monk gave a}
\stick{cordial wave with his head\tsk as much}
\stick{as to say, No doubt, there is misery}
\stick{enough in every corner of the world,}
\stick{as well as within our convent\tsk But}
\stick{we distinguish, said I, laying my}
\stick{hand upon the sleeve of his tunick,}
\stick{in return for his appeal\tsk we distin-}
\stick{guish, my good Father! betwixt}
\stick{those who wish only to eat the bread}
\stick{of their own labour\tsk and those who}
\stick{eat the bread of other people’s, and}
\stick{have no other plan in life, but to get}
\catch{through}
\stick{through it in sloth and ignorance, \i{for}}
\i{the love of God}.

The poor Franciscan made no re-
\stick{ply: a hectic of a moment pass’d}
\stick{across his cheek, but could not tarry}
\stick{\tsk Nature seemed to have had done}
\stick{with her resentments in him; he}
\stick{shewed none\tsk but letting his staff}
\stick{fall within his arm, he press’d both}
\stick{his hands with resignation upon his}
breast, and retired.

\patchv[6\baselineskip]{C}{}


\head{24pt}{THE\enspace MONK}
\head{16pt}{CALAIS}

\vskip 4pt

\dropcap{M}{Y} heart smote me the moment
he shut the door\tsk Psha! said
I, with an air of carelessness, three 
several times\tsk but it would not do: 
every ungracious syllable I had ut-
ter’d crouded back into my imagi-
nation: I reflected, I had no right
over the poor Franciscan, but to de-
ny him; and that the punishment of 
that was enough to the disappointed
without the addition of unkind lan-
guage\tsk I consider’d his grey hairs\tsk
\stick{his courteous figure seem’d to re-}
enter and gently ask me what injury
he had done me?\tsk and why I could
\catch{use}
\stick{use him thus?\tsk I would have given} 
\stick{twenty livres for an advocate.\tsk I} 
\stick{have behaved very ill, said I within} 
\stick{myself; but I have only just set out}
\stick{upon my travels; and shall learn} 
better manners as I get along.

\patchs[12\baselineskip]{C 2}{}

\noindent
\tstrut 24pt \centerline{\lss{THE}\enspace \lss{DESOBLIGEANT}.}
\head{16pt}{CALAIS}

\vskip -.5 \baselineskip

\dropcap{W}{hen} a man is discontented 
with himself, it has one ad-\break
\stick{vantage however, that it puts him}
\stick{into an excellent frame of mind for}
\stick{making a bargain.  Now there being}
\stick{no travelling through France and}
\stick{Italy without a chaise\tsk and nature}
\stick{generally prompting us to the thing}
\stick{we are fittest for, I walk’d out into}
\stick{the coach yard to buy or hire some-}
\stick{thing of that kind to my purpose:} 
\stick{an old \fnast\ Desobligeant in the furthest}
\stick{corner of the court, hit my fancy at} 
\stick{first sight, so I instantly got into it,}

\bgroup\fontsize{9}{12}\selectfont\noindent
\stick{\enspace\fnast\enspace A chaise, so called in France, from its}
holding but one person.\par
\egroup
\catch{and}
\stick{and finding it in tolerable harmony}
\stick{with my feelings, I ordered the}
\stick{waiter to call Monsieur Dessein the}
\stick{master of the hôtel\tsk but Monsieur}
\stick{Dessein being gone to vespers, and} 
\stick{not caring to face the Franciscan}
\stick{whom I saw on the opposite side of}
\stick{the court, in conference with a lady}
\stick{just arrived, at the inn\tsk I drew the}
\stick{taffeta curtain betwixt us, and being}
\stick{determined to write my journey, I}
\stick{took out my pen and ink, and wrote}
the preface to it in the \i{Disobligeant}.


\patchs[5\baselineskip]{C 3}{}

\noindent
\centerline{\tstrut 18pt\lsv{48}{PREFACE}}
\centerline{\tstrut 18pt\lsv{12}{\scshape In the DESOBLIGEANT}.}

\dropcap{I}{T} must have been observed by\break
many a peripatetic philosopher,\break 
\stick{That nature has set up by her}
\stck{own unquestionable authority certain}
\stick{boundaries and fences to circum-}
\stick{scribe the discontent of man: she has}
\stick{effected her purpose in the quietest}
\stick{and easiest manner by laying him}
\stick{under almost insuperable obligations}
\stick{to work out his ease, and to sustain}
\stick{his sufferings at home.  It is there}
\stick{only that she has provided him with}
\stick{the most suitable objects to partake}
\stick{of his happiness, and bear a part of}
\stick{that burden which in all countries}
\stick{and ages, has ever been too heavy}
\catch{for} 
\stick{for one pair of shoulders.  ’Tis true}
\stick{we are endued with an imperfect}
\stick{power of spreading our happiness}
\stick{sometimes beyond \i{her} limits, but}
\stick{’tis so ordered, that from the want}
\stick{of languages, connections, and de-}
\stick{pendencies, and from the difference}
\stick{in education, customs and habits,}
\stick{we lie under so many impediments}
\stick{in communicating our sensations out}
\stick{of our own sphere, as often amount}
to a total impossibility.

It will always follow from hence,
\stick{that the balance of sentimental com-}
\stick{merce is always against the expa-}
\stick{triated adventurer: he must buy}
\stick{what he has little occasion for at}
\stick{their own price\tsk his conversation}
\catchs{C 4}{will}
\stick{will seldom be taken in exchange}
\stick{for theirs without a large discount\tsk}
\stick{and this, by the by, eternally driv-}
\stick{ing him into the hands of more} 
\stick{equitable brokers for such conver-}
\stick{sation as he can find, it requires no}
\stick{great spirit of divination to guess at}
his party\tsk

This brings me to my point; and
\stick{naturally leads me (if the see-saw of}
\stick{this \i{Desobligeant} will but let me get}
\stick{on) into the efficient as well as the} 
final causes of travelling\tsk

Your idle people that leave their
\stick{native country and go abroad for}
\stick{some reason or reasons which may}
\stick{be derived from one of these general}
causes\tsk 

\patch{Infirmity}

\noindent
\stick{\kern 32pt Infirmity of body,\hfill}
\stick{\kern 32pt Imbecility of mind, or\hfill}
\stick{\kern 32pt Inevitable necessity.\hfill}
\stick{\tstrut 16pt The first two include all those who}
\stick{travel by land or by water, la-}
\stick{bouring with pride, curiosity, vanity}
\stick{or spleen, subdivided and combined} 
\stick{\i{in infinitum}.\hfill}
\stick{\tstrut 16pt\indent  The third class includes the whole}
\stick{army of peregrine martyrs; more}
\stick{especially those travellers who set}
\stick{out upon their travels with the be-}
\stick{nefit of the clergy, either as delin-}
\stick{quents travelling under the direction}
\stick{of governors recommended by the}
\stick{magistrate \tsk\ or young gentlemen}
\stick{transported by the cruelty of pa-}
\stick{rents and guardians, and travelling}
\stick{under the direction of governors re-}
\catch{commended}
\stick{commended by Oxford, Aberdeen}
and Glasgow.

There is a fourth class, but their
\stick{number is so small that they would}
\stick{not deserve a distinction, were it not}
\stick{necessary in a work of this nature to}
\stick{observe the greatest precision and}
\stick{nicety, to avoid a confusion of cha-}
\stick{racter.  And these men I speak of,}
\stick{are such as cross the seas and sojourn}
\stick{in a land of strangers, with a view of}
\stck{saving money for various reasons and}
\stick{upon various pretences: but as they}
\stck{might also save themselves and others}
\stick{a great deal of unnecessary trouble} 
\stick{by saving their money at home\tsk} 
\stick{and as their reasons for travelling}
\stick{are the least complex of any other}
\stick{species of emigrants, I shall di-}
\catch{stinguish}
\stick{stinguish these gentlemen by the}
\stick{name of\hfill}
\stick{\hfill Simple Travellers.\hfill}
\stick{\tstrut 18pt\indent
Thus the whole circle of travellers}
\stick{may be reduced to the following}
\stick{\i{Heads}.\hfill}
\vskip -2pt
\stick{\kern 32pt Idle Travellers,\hfill}
\stick{\kern 32pt Inquisitive Travellers,\hfill}
\stick{\kern 32pt Lying Travellers,\hfill}
\stick{\kern 32pt Proud Travellers,\hfill}
\stick{\kern 32pt Vain Travellers,\hfill}
\stick{\kern 32pt Splenetic Travellers.\hfill}
\stick{\tstrut 18pt\indent
Then follow the Travellers of}
\stick{Necessity.\hfill}
\stick{\indent  The delinquent and felonious} 
\stick{\indent\indent  Traveller,\hfil}
\stick{\indent The unfortunate and innocent} 
\stick{\indent\indent  Traveller,\hfil}
\stick{\indent The simple Traveller,\hfil}

\patch{And}
\noindent
\stick{\indent And last of all (if you please) The}
\stick{\tstrut14pt\indent Sentimental Traveller (meaning}
\stick{thereby myself) who have travell’d,} 
\stick{and of which I am now sitting down} 
\stick{to give an account\tsk as much out of}
\stick{\i{Necessity}, and the \i{besoin de} Voyager,}
as any one in the class.

I am well aware, at the same time,
\stick{as both my travels and observations}
\stick{will be altogether of a different cast}
\stick{from any of my fore-runners; that I}
\stick{might have insisted upon a whole}
\stick{nitch entirely to myself\tskk but I}
\stick{should break in upon the confines of} 
\stick{the \i{Vain} Traveller, in wishing to}
\stick{draw attention towards me, till I}
\stick{have some better grounds for it,}
\stick{than the mere \i{Novelty of my Ve-}}
\i{hicle}.
\patch{It}

It is sufficient for my reader, if\break
\stick{he has been a traveller himself, that}
\stick{with study and reflection hereupon}
\stick{he may be able to determine his} 
\stick{own place and rank in the cata-}
\stick{logue\tsk it will be one step towards}
\stick{knowing himself; as it is great odds}
\stick{but he retains some tincture and}
\stick{resemblance, of what he imbibed or}
carried out, to the present hour.

The man who first transplanted\break
\stick{the grape of Burgundy to the Cape} 
\stick{of Good Hope (observe he was a}
\stick{Dutch man) never dreamt of drink-}
\stick{ing the same wine at the Cape,}
\stick{that the same grape produced upon}
\stick{the French mountains,\tsk he was too}
\stick{phlegmatic for that\tsk but undoubt-}
\catch{edly}
\stick{edly he expected to drink some sort}
\stick{of vinous liquor; but whether good,}
\stick{bad, or indifferent\tsk he knew enough}
\stick{of this world to know, that it did}
\stick{not depend upon his choice, but}
\stick{that what is generally called \i{chance}}
\stick{was to decide his success: however,}
\stick{he hoped for the best; and in these}
\stick{hopes, by an intemperate confidence}
\stick{in the fortitude of his head, and the}
\stick{depth of his discretion, \i{Mynheer}}
\stick{might possibly overset both in his} 
\stick{new vineyard; and by discovering}
\stick{his nakedness, become a laughing-}
stock to his people.

Even so it fares with the poor\break
\stick{Traveller, sailing and posting through}
\stick{the politer kingdoms of the globe}
\catch{in}
\stick{in pursuit of knowledge and im-}
provements.

Knowledge and improvements are
\stick{to be got by sailing and posting}
\stick{for that purpose; but whether use-}
\stick{ful knowledge and real improve-}
\stick{ments, is all a lottery\tsk and even}
\stick{where the adventurer is successful,}
\stick{the acquired stock must be used}
\stick{with caution and sobriety to turn} 
\stick{to any profit\tsk but as the chances}
\stick{run prodigiously the other way both}
\stick{as to the acquisition and application,} 
\stick{I am of opinion, That a man}
\stick{would act as wisely, if he could}
\stick{prevail upon himself, to live con-}
\stick{tented without foreign knowledge}
\stick{or foreign improvements, especially}
\catch{if}
\stick{if he lives in a country that has no}
\stick{absolute want of either\tsk and indeed,} 
\stick{much grief of heart has it oft and}
\stick{many a time cost me, when I have}
\stick{observed how many a foul step the} 
\stick{inquisitive Traveller has measured}
\stick{to see sights and look into disco-}
\stick{veries; all which, as Sancho Pança}
\stick{said to Don Quixote, they might}
\stick{have seen dry-shod at home.  It is}
\stick{an age so full of light, that there}
\stick{is scarce a country or corner of Eu-}
\stick{rope whose beams are not crossed} 
\stick{and interchanged with others\tsk}
\stick{Knowledge in most of its branches,}
\stick{and in most affairs, is like music} 
\stick{in an Italian street, whereof those}
\stick{may partake, who pay nothing\tsk}
\stick{But there is no nation under heaven}
\catch{\tsk and}
\stick{\tsk and God is my record, (before} 
\stick{whose tribunal I must one day come}
\stick{and give an account of this work)\tsk}
\stick{that I do not speak it vauntingly\tsk}
\stick{But there is no nation under heaven}
\stck{abounding with more variety of learn-}
\stck{ing,\tsk where the sciences may be more}
\stick{fitly woo’d, or more surely won than}
\stick{here\tsk where art is encouraged, and}
\stick{will so soon rise high\tsk where Nature} 
\stick{(take her altogether) has so little}
\stck{to answer for\tsk and, to close all, where}
\stick{there is more wit and variety of cha-}
\stick{racter to feed the mind with\tsk Where}
\stick{then, my dear countrymen, are you}
going\tsk 

\tsk We are only looking at this
\stick{chaise, said they\tsk Your most obe-}
\catchv{D}{dient}
\stick{dient servant, said I, skipping out of}
\stick{it, and pulling off my hat\tsk We were}
\stick{wondering, said one of them, who,}
\stick{I found, was an \i{inquisitive traveller}}
\stick{\tsk what could occasion its motion.\tsk} 
\stick{\tsk ’Twas the agitation, said I coolly,}
\stick{of writing a preface\tsk I never heard,}
\stick{said the other, who was a \i{simple tra-}}
\stick{\i{veller}, of a preface wrote in a \i{Deso-}}
\stick{\i{bligeant}.\tsk It would have been better,} 
said I, in a \i{Vis a Vis}.

\tsk \i{As an English man does not travel}
\stick{\i{to see English men}, I retired to my}
room.

\newpage


\head{36pt}{CALAIS}

\vskip 6pt

\initial{I}{Perceived} \tight{that something darken’d} 
the passage more than myself, as 
\stick{I stepp’d along it to my room; it}
\stick{was effectually Mons.\@ Dessein, the}
\stick{master of the hôtel, who had just re-}
\stick{turn’d from vespers, and, with his}
\stick{hat under his arm, was most com-}
\stick{plaisantly following me, to put me in}
\stick{mind of my wants.  I had wrote my-}
\stick{self pretty well out of conceit with}
\stick{the \i{Desobligeant}; and Mons.\@ Dessein}
\stick{speaking of it, with a shrug, as if it}
\stick{would no way suit me, it immediately}
\stick{struck my fancy that it belong’d to}
\stick{some \i{innocent traveller}, who, on his}
\catchs{D 2}{return}
\stick{return home, had left it to Mons.}
\stick{Dessein’s honour to make the most}
\stick{of.  Four months had elapsed since}
\stick{it had finish’d its career of Europe in}
\stick{the corner of Mons.\@ Dessein’s coach-}
\stick{yard; and having sallied out from}
\stick{thence but a vampt-up business at}
\stick{the first, though it had been twice}
\stick{taken to pieces on Mount Sennis, it}
\stick{had not profited much by its adven-}
\stick{tures\tsk but by none so little as the}
\stick{standing so many months unpitied in}
\stick{the corner of Mons.\@ Dessein’s coach-}
\stick{yard.  Much indeed was not to be}
\stick{said for it\tsk but something might\tsk}
\stick{and when a few words will rescue}
\stick{misery out of her distress, I hate the}
man who can be a churl of them.

\patch[\baselineskip]{\tsk Now}

\tsk Now was I the master of this
\stick{hôtel, said I, laying the point of my}
\stck{fore-finger on Mons.\@ Dessein’s breast,}
\stick{I would inevitably make a point of}
\stick{getting rid of this unfortunate \i{Deso}-}
\stk{\i{bligeant}\tsk it stands swinging reproaches}
at you every time you pass by it\tsk

\i{Mon Dieu}! said Mons.\@ Dessein\tsk
\stick{I have no interest\tsk Except the in-}
\stick{terest, said I, which men of a certain}
\stick{turn of mind take, Mons.\@ Dessein, in}
\stick{their own sensations\tsk I’m persuaded,}
\stick{to a man who feels for others as well}
\stick{as for himself, every rainy night, dis-}
\stick{guise it as you will, must cast a damp}
\stick{upon your spirits\tsk You suffer, Mons.}
Dessein, as much as the machine\tsk 

\patchs[\baselineskip]{D 3}{I}

\tight{I have always observed, when there}
\stick{is as much \i{sour} as \i{sweet} in a compli-}
\stick{ment, that an Englishman is eternally}
\stick{at a loss within himself, whether to}
\stick{take it, or let it alone: a Frenchman}
\stick{never is: Mons.\@ Dessein made me a}
bow.

\i{C’est bien vrai}, said he\tsk But in this
\stick{case I should only exchange one dis-}
\stick{quietude for another, and with loss:}
\stick{figure to yourself, my dear Sir, that}
\stick{in giving you a chaise which would}
\stick{fall to pieces before you had got half}
\stick{way to Paris\tsk figure to yourself how}
\stick{much I should suffer, in giving an ill}
\stick{impression of myself to a man of ho-}
\stick{nour, and lying at the mercy, as I}
must do, \i{d’un homme d’esprit}.

\patch{The}

The dose was made up exactly\break
\stick{after my own prescription; so I could}
\stick{not help tasting it\tsk and returning}
\stick{Mons.\@ Dessein his bow, without more}
\stick{casuistry we walk’d together towards}
\stick{his Remise, to take a view of his}
magazine of chaises.

\patchs[10\baselineskip]{D 4}{}


\head{24pt}{IN THE STREET}
\head{16pt}{CALAIS}

\vskip 8pt

\dropcap{I}{t} must needs be a hostile kind\break
of a world, when the buyer (if\break
\stick{it be but of a sorry post-chaise) can-}
\stick{not go forth with the seller there-}
\stick{of into the street to terminate the}
\stick{difference betwixt them, but he in-}
\stick{stantly falls into the same frame of}
\stick{mind and views his conventionist}
\stick{with the same sort of eye, as if he}
\stick{was going along with him to Hyde-}
\stick{park corner to fight a duel.  For my}
\stick{own part, being but a poor sword’s-}
\stick{man, and no way a match for Mon-}
\stick{sieur \i{Dessein}, I felt the rotation of}
\catch{all}
\stick{all the movements within me, to}
\stick{which the situation is incident\tskk I}
\stick{looked at Monsieur Dessein through}
\stick{and through\tskk ey’d him as he walked}
\stick{along in profile\tskk then, \i{en face}\tskk}
\stick{thought he look’d like a Jew\tskk}
\stick{then a Turk\tskk disliked his wig\tskk}
\stick{cursed him by my gods\tskk wished}
him at the devil\tsk 

\tsk And is all this to be lighted 
\stick{up in the heart for a beggarly ac-}
\stick{count of three or four louisd’ors,}
\stick{which is the most I can be over-}
\stick{reach’d in?\tsk Base passion! said I,}
\stick{turning myself about, as a man na-}
\stick{turally does upon a sudden reverse of}
\stick{sentiment\tsk base, ungentle passion!}
\stick{thy hand is against every man, and}
\catch{every}
\stick{every man’s hand against thee\tsk}
\stick{heaven forbid! said she, raising her}
\stick{hand up to her forehead, for I had}
\stick{turned full in front upon the lady}
\stick{whom I had seen in conference with}
\stick{the monk\tsk she had followed us}
\stick{unperceived\tsk Heaven forbid indeed!}
\stick{said I, offering her my own\tsk she}
\stick{had a black pair of silk gloves open}
\stick{only at the thumb and two fore-}
\stick{fingers, so accepted it without re-}
\stick{serve\tsk and I led her up to the door}
of the Remise.

Monsieur Dessein had \i{diabled} the
\stick{key above fifty times before he found}
\stick{out he had come with a wrong one}
\stick{in his hand: we were as impatient as}
\stick{himself to have it open’d; and so}
\catch{attentive}
\stick{attentive to the obstacle that I con-}
\stick{tinued holding her hand almost with-}
\stick{out knowing it; so that Monsieur}
\stick{\i{Dessein} left us together with her hand}
\stick{in mine, and with our faces turned}
\stick{towards the door of the Remise, and}
\stick{said he would be back in five}
minutes.

Now a colloquy of five mi-\break
\stick{nutes, in such a situation, is worth}
\stick{one of as many ages, with your}
\stick{faces turned towards the street:}
\stick{in the latter case, ’tis drawn from}
\stick{the objects and occurrences with-}
\stick{out\tsk when your eyes are fixed upon}
\stick{a dead blank\tsk you draw purely from}
\stick{yourselves.  A silence of a single}
\stick{moment upon Monsieur \i{Dessein}’s}
\catch{leaving}
\stick{leaving us, had been fatal to the}
\stick{situation\tsk she had infallibly turned}
\stick{about\tsk so I begun the conversation}
instantly.\tsk 

\tsk But what were the temptations,
\stick{(as I write not to apologize for the}
\stick{weaknesses of my heart in this tour}
\stick{\tsk but to give an account of them)\tsk}
\stick{shall be described with the same}
simplicity, with which I felt them.

\newpage


\centerline{\tstrut 24pt\lsv{12}{THE REMISE DOOR}.}
\head{24pt}{CALAIS}

\vskip 4pt

\dropcap{W}{hen} I told the reader that I\break
did not care to get out of the\break
\stick{\i{Desobligeant}, because I saw the monk}
\stick{in close conference with a lady just}
\stick{arrived at the inn\tsk I told him the}
\stick{truth; but I did not tell him the}
\stick{whole truth; for I was full as much}
\stick{restrained by the appearance and}
\stick{figure of the lady he was talking to.}
\stick{Suspicion crossed my brain, and said,}
\stick{he was telling her what had passed:}
\stick{something jarred upon it within me\tsk}
I wished him at his convent.

\patch[\baselineskip]{When}

\indent\tstrut 24pt When the heart flies out before\break
\stick{the understanding, it saves the judg-}
\stick{ment a world of pains\tsk I was cer-}
\stick{tain she was of a better order of}
\stick{beings\tsk however, I thought no more}
\stick{of her, but went on and wrote my}
preface.

The impression returned, upon my
\stick{encounter with her in the street; a}
\stick{guarded frankness with which she}
\stick{gave me her hand, shewed, I}
\stick{thought, her good education and her}
\stick{good sense; and as I led her on, I}
\stick{felt a pleasurable ductility about her,}
\stick{which spread a calmness over all my}
spirits\tsk 

\patch[\baselineskip]{\tsk Good}

\tsk Good God! how a man might lead such a creature as this round the world
with him!\tsk 

I had not yet seen her face\tsk ’twas
\stick{not material; for the drawing was}
\stick{instantly set about, and long before}
\stick{we had got to the door of the}
\stick{Remise, \i{Fancy} had finished the whole}
\stick{head, and pleased herself as much}
\stick{with its fitting her goddess, as if}
\stick{she had dived into the \s{Tiber} for it\tsk}
\stick{but thou art a seduced, and a se-}
\stick{ducing slut; and albeit thou cheatest}
\stick{us seven times a day with thy pic-}
\stick{tures and images, yet with so many}
\stick{charms dost thou do it, and thou}
\stick{deckest out thy pictures in the shapes}
\stick{of so many angels of light, ’tis a}
shame to break with thee.

\patch{When}

\indent\tstrut 18pt
When we had got to the door of\break
\stick{the Remise, she withdrew her hand}
\stick{from across her forehead, and let me}
\stick{see the original\tsk it was a face of}
\stick{about six and twenty\tsk of a clear}
\stick{transparent brown, simply set off}
\stick{without rouge or powder\tsk it was not}
\stick{critically handsome, but there was}
\stick{that in it, which in the frame of}
\stick{mind I was in, which attached me}
\stick{much more to it\tsk it was interesting;}
\stick{I fancied it wore the characters of a}
\stick{widow’d look, and in that state of}
\stick{its declension, which had passed the}
\stick{two first paroxysms of sorrow, and}
\stick{was quietly beginning to reconcile}
\stick{itself to its loss\tsk but a thousand}
\stick{other distresses might have traced}
\stick{the same lines; I wish’d to know}
\catch{what}
\stick{what they had been\tsk and was ready}
\stick{to enquire, (had the same \i{bon ton} of}
\stck{conversation permitted, as in the days}
\stick{of Esdras)\tsk \lqq \i{What aileth thee? and}}
\stick{\i{why art thou disquieted? and why is}}
\stick{\i{thy understanding troubled?}\rqq\tsk In a}
\stick{word, I felt benevolence for her;}
\stick{and resolved some way or other to}
\stick{throw in my mite of courtesy\tsk if}
not of service.

Such were my temptations\tsk and
\stick{in this disposition to give way to}
\stick{them, was I left alone with the lady}
\stick{with her hand in mine, and with our}
\stick{faces both turned closer to the door}
\stick{of the Remise than what was ab-}
solutely necessary.

\patchv{E}{\tstrut 16pt \lss{THE}}

\centerline{\tstrut 24pt\lsv{12}{THE REMISE DOOR}.}
\head{24pt}{CALAIS}

\vskip 4pt

\dropcap{T}{his} certainly, fair lady!\break
said I, raising her hand up\break
\stick{a little lightly as I began, must be}
\stick{one of Fortune’s whimsical doings:}
\stick{to take two utter strangers by their}
\stick{hands\tsk of different sexes, and per-}
\stick{haps from different corners of the}
\stick{globe, and in one moment place}
\stick{them together in such a cordial situ-}
\stick{ation, as Friendship herself could}
\stick{scarce have atchieved for them, had}
she projected it for a month\tsk

\patch[\baselineskip]{\tsk And}

\tsk And your reflection upon it,
\stick{shews how much, Monsieur, she has}
embarrassed you by the adventure.\tsk 

When the situation is, what we 
\stick{would wish, nothing is so ill-timed as}
\stick{to hint at the circumstances which}
\stick{make it so: you thank Fortune, con-}
\stick{tinued she\tsk you had reason\tsk the}
\stick{heart knew it, and was satisfied;}
\stick{and who but an English philosopher}
\stick{would have sent notice of it to the}
brain to reverse the judgment?

In saying this, she disengaged her 
\stick{hand with a look which I thought}
\stick{a sufficient commentary upon the}
text.

\patchs[\baselineskip]{E 2}{It}

It is a miserable picture which I
\stick{am going to give of the weakness of}
\stick{my heart, by owning, that it suffered}
\stick{a pain, which worthier occasions}
\stick{could not have inflicted.\tsk I was}
\stick{mortified with the loss of her hand,}
\stick{and the manner in which I had lost}
\stick{it carried neither oil nor wine to the}
\stick{wound: I never felt the pain of a}
\stick{sheepish inferiority so miserably in}
my life.

The triumphs of a true feminine
\stick{heart are short upon these discomfi-}
\stick{tures.  In a very few seconds she}
\stick{laid her hand upon the cuff of my}
\stick{coat, in order to finish her reply; so}
\stick{some way or other, God knows how,}
I regained my situation.

\patch[\baselineskip]{\tsk She}

\tsk\tstrut 18pt She had nothing to add.

I forthwith began to model a dif-
\stick{ferent conversation for the lady,}
\stick{thinking from the spirit as well as}
\stick{moral of this, that I had been mi-}
\stick{staken in her character; but upon}
\stick{turning her face towards me, the}
\stick{spirit which had animated the reply}
\stick{was fled\tsk the muscles relaxed, and}
\stick{I beheld the same unprotected look}
\stick{of distress which first won me to her}
\stick{interest\tsk melancholy! to see such}
\stick{sprightliness the prey of sorrow.\tsk I}
\stick{pitied her from my soul; and though}
\stick{it may seem ridiculous enough to a}
\stick{torpid heart,\tsk I could have taken her}
\stick{into my arms, and cherished her,}
\catchs{E 3}{though} 
\stick{though it was in the open street,}
without blushing.

\tight{The pulsations of the arteries along}
\stick{my fingers pressing across hers, told}
\stick{her what was passing within me: she}
\stick{looked down\tsk a silence of some mo-}
ments followed.

I fear, in this interval, I must\break
\stick{have made some slight efforts to-}
\stick{wards a closer compression of her}
\stick{hand, from a subtle sensation I felt}
\stick{in the palm of my own\tskk not as if}
\stick{she was going to withdraw hers\tskk}
\stick{but, as if she thought about it\tskk}
\stick{and I had infallibly lost it a second}
\stick{time, had not instinct more than rea-}
\stick{son directed me to the last resource}
\catch{in}
\stick{in these dangers\tsk to hold it loosely,}
\stick{and in a manner as if I was every}
\stick{moment going to release it, of my-}
\stick{self; so she let it continue, till Mon-}
\stick{sieur \i{Dessein} returned with the key;}
\stick{and in the mean time I set myself to}
\stick{consider how I should undo the ill}
\stick{impressions which the poor monk’s}
\stick{story, in case he had told it her,}
\stick{must have planted in her breast}
against me.

\patchs[6\baselineskip]{E 4}{}


\centerline{\tstrut 36pt\lsv{24}{THE SNUFF-BOX}.}
\head{24pt}{CALAIS}

\vskip 8pt

\dropcap{T}{HE} good old monk was within\break
six paces of us, as the idea of\break
\stick{him cross’d my mind; and was ad-}
\stick{vancing towards us a little out of the}
\stck{line, as if uncertain whether he should}
\stick{break in upon us or no.\tsk He stopp’d,}
\stick{however, as soon as he came up to}
\stick{us, with a world of frankness; and}
\stick{having a horn snuff-box in his hand,}
\stick{he presented it open to me\tsk You}
\stick{shall taste mine\tsk said I, pulling out}
\stick{my box (which was a small tortoise}
\stick{one) and putting it into his hand\tsk}
\catch{’Tis}
\stick{’Tis most excellent, said the monk;}
\stick{Then do me the favour, I replied, to}
\stick{accept of the box and all, and when}
\stick{you take a pinch out of it, sometimes}
\stick{recollect it was the peace-offering of}
\stick{a man who once used you unkindly,}
but not from his heart.

The poor monk blush’d as red as
\stick{scarlet.  \i{Mon Dieu!} said he, press-}
\stick{ing his hands together\tsk you never}
\stick{used me unkindly.\tsk I should think,}
\stick{said the lady, he is not likely.  I}
\stick{blush’d in my turn; but from what}
\stick{movements, I leave to the few who}
\stick{feel to analyse\tsk Excuse me, Madame,}
\stick{replied I\tsk I treated him most un-}
\stick{kindly; and from no provocations\tsk}
\stick{’Tis impossible, said the lady.\tsk My}
\catch{God!}
God! cried the monk, with a warmth
of asseveration which seemed not to
belong to him\tsk the fault was in me, 
and in the indiscretion of my zeal\tsk 
the lady opposed it, and I joined with
her in maintaining it was impossible,
that a spirit so regulated as his, could
give offence to any.

I knew not that contention could 
\stick{be rendered so sweet and pleasurable}
\stick{a thing to the nerves as I then felt it.}
\stick{\tsk We remained silent, without any}
\stick{sensation of that foolish pain which}
\stick{takes place, when in such a circle you}
\stick{look for ten minutes in one another’s}
\stick{faces without saying a word.  Whilst}
\stick{this lasted, the monk rubb’d his horn}
\stick{box upon the sleeve of his tunick;}
\catch{and}
\stick{and as soon as it had acquired a little}
\stick{air of brightness by the friction\tsk he}
\stick{made a low bow, and said, ’twas too}
\stick{late to say whether it was the weak-}
\stick{ness or goodness of our tempers}
\stick{which had involved us in this contest}
\stick{\tsk but be it as it would\tsk he begg’d}
\stick{we might exchange boxes\tsk In say-}
\stick{ing this, he presented his to me with}
\stick{one hand, as he took mine from me}
\stick{in the other; and having kiss’d it\tsk}
\stick{with a stream of good nature in his}
\stick{eyes he put it into his bosom\tsk and}
took his leave.

I guard this box, as I would the 
\stick{instrumental parts of my religion, to}
\stick{help my mind on to something bet-}
\stick{ter: in truth, I seldom go abroad}
\catch{without}
\stick{without it; and oft and many a time}
\stick{have I called up by it the courteous}
\stick{spirit of its owner to regulate my}
\stick{own, in the justlings of the world;}
\stick{they had found full employment for}
\stck{his, as I learnt from his story, till about}
\stick{the forty-fifth year of his age, when}
\stick{upon some military services ill re-}
\stick{quited, and meeting at the same time}
\stick{with a disappointment in the ten-}
\stick{derest of passions, he abandon’d the}
\stick{sword and the sex together, and took}
\stick{sanctuary, not so much in his con-}
vent as in himself.

I feel a damp upon my spirits, as
\stick{I am going to add, that in my last}
\stick{return through Calais, upon inquir-}
\stick{ing after Father Lorenzo, I heard}
\catch{he}
\stick{he had been dead near three months,}
\stick{and was buried, not in his convent,}
\stick{but, according to his desire, in a little}
\stick{cimetiery belonging to it, about two}
\stick{leagues off: I had a strong desire to}
\stick{see where they had laid him\tsk when,}
\stick{upon pulling out his little horn box,}
\stick{as I sat by his grave, and plucking}
\stick{up a nettle or two at the head of it,}
\stick{which had no business to grow there,}
\stick{they all struck together so forcibly}
\stick{upon my affections, that I burst into}
\stick{a flood of tears\tsk but I am as weak as}
\stick{a woman; and I beg the world not}
to smile, but to pity me.

\newpage


\centerline{\tstrut 24pt\lsv{12}{THE REMISE DOOR}.}
\head{24pt}{CALAIS}

\vskip 4pt

\initial{I}{HAD} never quitted the lady’s\break
hand all this time, and had held\break 
\stick{it so long, that it would have been}
\stick{indecent to have let it go, without}
\stick{first pressing it to my lips: the blood}
\stick{and spirits, which had suffer’d a re-}
\stick{vulsion from her, crouded back to}
her, as I did it.

Now the two travellers who had
\stick{spoke to me in the coach-yard, hap-}
\stick{pening at that crisis to be passing by,}
\stick{and observing our communications,}
\stick{naturally took it into their heads that}
\catch{we}
\stick{we must be \i{man and wife} at least; so}
\stick{stopping as soon as they came up to}
\stick{the door of the Remise, the one of}
\stick{them, who was the inquisitive tra-}
\stick{veller, ask’d us, if we set out for Paris}
\stick{the next morning?\tsk I could only}
\stick{answer for myself, I said; and the}
\stick{lady added, she was for Amiens.\tsk}
\stick{We dined there yesterday, said the}
\stick{simple traveller\tsk You go directly}
\stick{through the town, added the other,}
\stick{in your road to Paris.  I was going}
\stick{to return a thousand thanks for the}
\stick{intelligence, \i{that Amiens was in the}}
\stick{\i{road to Paris\/}; but, upon pulling out}
\stick{my poor monk’s little horn box to}
\stick{take a pinch of snuff\tsk I made them}
\stick{a quiet bow, and wishing them a}
\catch{good}
\stick{good passage to Dover\tsk they left us}
alone\tsk 

\tsk Now where would be the harm,\break
\stick{said I to myself, if I was to beg of}
\stick{this distressed lady to accept of half}
\stick{of my chaise?\tskk and what mighty}
mischief could ensue?

Every dirty passion, and bad pro-\break
\stick{pensity in my nature took the alarm,}
\stick{as I stated the proposition\tsk It will}
\stick{oblige you to have a third horse, said}
\stick{\s{Avarice}, which will put twenty}
\stick{livres out of your pocket.\tsk You know}
\stick{not what she is, said \s{Caution}\tsk or}
\stick{what scrapes the affair may draw you}
into, whisper’d \s{Cowardice}\tsk 

\patch[\baselineskip]{Depend}

Depend upon it, Yorick!\@ said\break
\stick{\s{Discretion}, ’twill be said you went}
\stick{off with a mistress, and came by as-}
signation to Calais for that purpose\tsk 

\tsk You can never after, cried \s{Hy}-\break
\stick{\s{pocrisy} aloud, shew your face in the}
\stick{world\tsk or rise, quoth \s{Meanness}, in}
\stick{the church\tsk or be any thing in it, said}
\s{Pride}, but a lousy prebendary.

\tsk But ’tis a civil thing, said I\tsk 
\stick{and as I generally act from the first}
\stick{impulse, and therefore seldom listen}
\stick{to these cabals, which serve no pur-}
\stick{pose, that I know of, but to encom-}
\stick{pass the heart with adamant\tsk I turn’d}
instantly about to the lady\tsk 

\patchv[\baselineskip]{E}{\tsk But}

\tsk But she had glided off unper-
\stick{ceived, as the cause was pleading,}
\stick{and had made ten or a dozen}
\stick{paces down the street, by the time I}
\stick{had made the determination; so I}
\stick{set off after her with a long stride, to}
\stick{make her the proposal with the best}
\stick{address I was master of; but observ-}
\stick{ing she walk’d with her cheek half}
\stick{resting upon the palm of her hand\tsk}
\stick{with the slow, short-measur’d step of}
\stick{thoughtfulness, and with her eyes, as}
\stick{she went step by step, fix’d upon the}
\stick{ground, it struck me, she was trying}
\stick{the same cause herself.\tsk God help}
\stick{her! said I, she has some mother-in-}
\stick{law, or tartufish aunt, or nonsensical}
\stick{old woman, to consult upon the oc-}
\stick{casion, as well as myself: so not car-}
\catch{ing}
\stick{ing to interrupt the processe, and}
\stick{deeming it more gallant to take her}
\stick{at discretion than by surprize, I faced}
\stick{about, and took a short turn or two}
\stick{before the door of the Remise, whilst}
she walk’d musing on one side.

\patchs[10\baselineskip]{F 2}{}


\centerline{\tstrut 24pt\lsv{24}{IN THE STREET}.}
\head{18pt}{CALAIS}

\vskip 8pt

\dropcap{H}{aving}, on first sight of the\break
lady, settled the affair in my\break
\stick{fancy \lqq that she was of the better}
\stick{\lqq order of beings\rqq\tsk and then laid it}
\stick{down as a second axiom, as indisput-}
\stick{able as the first, That she was a wi-}
\stick{dow, and wore a character of dis-}
\stick{tress\tsk I went no further; I got ground}
\stck{enough for the situation which pleased}
\stick{me\tsk and had she remained close be-}
\stick{side my elbow till midnight, I should}
\stick{have held true to my system, and}
\stick{considered her only under that gene-}
ral idea.

\patch{She}

She had scarce got twenty paces 
\stck{distant from me, ere something within}
\stick{me called out for a more particular}
\stick{inquiry;\tsk it brought on the idea of a}
\stick{further separation\tsk I might possibly}
\stick{never see her more\tsk the heart is for}
\stick{saving what it can; and I wanted the}
\stick{traces thro’ which my wishes might}
\stick{find their way to her, in case I should}
\stick{never rejoin her myself: in a word,}
\stick{I wish’d to know her name\tsk her fa-}
\stick{mily’s\tsk her condition; and as I knew}
\stick{the place to which she was going, I}
\stick{wanted to know from whence she}
\stick{came: but there was no coming at}
\stick{all this intelligence; a hundred little}
\stick{delicacies stood in the way.  I form’d}
\stick{a score different plans\tsk There was}
\catchs{F 3}{no}
\stick{no such thing as a man’s asking her}
directly\tsk the thing was impossible.

A little French \i{debonaire} captain,
\stick{who came dancing down the street,}
\stick{shewed me, it was the easiest thing}
\stick{in the world; for popping in betwixt}
\stck{us, just as the lady was returning back}
\stick{to the door of the Remise, he intro-}
\stick{duced himself to my acquaintance,}
\stick{and before he had well got announ-}
\stick{ced, begg’d I would do him the ho-}
\stick{nour to present him to the lady\tskk I}
\stick{had not been presented myself\tskk so}
\stick{turning about to her, he did it just}
\stick{as well by asking her, if she had come}
\stick{from Paris?  No: she was going}
\stick{that rout, she said.\tsk \i{Vous n’etez pas}}
\stick{\i{de Londre?}\tsk She was not, she replied.}
\catch{\tsk Then}
\stick{\tsk Then Madame must have come}
\stick{thro’ Flanders.\tsk \i{Apparamment vous}}
\stick{\i{etez Flammande?} said the French cap-}
\stick{tain.\tsk The lady answered, she was.\tsk}
\stick{\i{Peutetre de Lisle?} added he\tsk She}
\stick{said, she was not of Lisle.\tsk Nor Ar-}
\stick{ras?\tsk nor Cambray?\tsk nor Ghent?\tsk}
\stick{nor Brussels? She answered, she was}
of Brussels.

He had had the honour, he said,
\stick{to be at the bombardment of it last}
\stick{war\tsk that it was finely situated, \i{pour}}
\stick{\i{cela}\tsk and full of noblesse when the}
\stick{Imperialists were driven out by the}
\stick{French (the lady made a slight curtsy)}
\stick{\tsk so giving her an account of the af-}
\stick{fair, and of the share he had had in}
\catchs{F 4}{it\tsk}
\stick{it\tsk he begg’d the honour to know}
her name\tsk so made his bow.

\tsk \i{Et Madame a son Mari?}\tsk said
\stick{he, looking back when he had made}
\stck{two steps\tsk and without staying for an}
answer\tsk danced down the street.

Had I served seven years appren-
\stick{ticeship to good breeding, I could}
not have done as much.

\newpage

\centerline{\tstrut 24pt\lsv{24}{THE REMISE}.}
\head{24pt}{CALAIS}

\vskip 4pt

\dropcap{A}{s} the little French captain left\break
us, Mons.\@ Dessein came up\break
\stick{with the key of the Remise in his}
\stick{hand, and forthwith let us into his}
magazine of chaises.

The first object which caught my\break
\stick{eye, as Mons.\@ Dessein open’d the}
\stick{door of the Remise, was another old}
\stick{tatter’d \i{Desobligeant:} and notwith-}
\stick{standing it was the exact picture of}
\stick{that which had hit my fancy so much}
\stick{in the coach-yard but an hour before}
\stick{\tsk the very sight of it stirr’d up a}
\catch{disagreeable}
\stk{disagreeable sensation within me now;}
\stick{and I thought ’twas a churlish beast}
\stick{into whose heart the idea could first}
\stick{enter, to construct such a machine;}
\stick{nor had I much more charity for the}
man who could think of using it.

I observed the lady was as little\break
\stick{taken with it as myself: so Mons.}
\stick{Dessein led us on to a couple of}
\stck{chaises which stood abreast, telling us}
\stick{as he recommended them, that they}
\stick{had been purchased by my lord A.}
\stick{and B.\@ to go the \i{grand tour}, but}
\stick{had gone no further than Paris, so}
\stick{were in all respects as good as new\tsk}
\stick{They were too good\tsk so I pass’d on}
\stick{to a third, which stood behind, and}
\stick{forthwith begun to chaffer for the}
\catch{price}
\stick{price\tsk But ’twill scarce hold two, said}
\stick{I, opening the door and getting in\tsk}
\stick{Have the goodness, Madam, said}
\stick{Mons.\@ Dessein, offering his arm, to}
\stick{step in\tsk The lady hesitated half a se-}
\stick{cond, and stepp’d in; and the waiter}
\stick{that moment beckoning to speak to}
\stick{Mons.\@ Dessein, he shut the door of}
the chaise upon us, and left us.

\newpage

\centerline{\tstrut 24pt\lsv{24}{THE REMISE}.}
\head{24pt}{CALAIS}

\vskip 4pt

\dropcap[loversize=-.2,lraise=.33,nindent=0em]{\itshape C}{\itshape ’est} \i{bien comique}, ’tis very\break
droll, said the lady smiling, from\break
\stick{the reflection that this was the se-}
\stick{cond time we had been left to-}
\stick{gether by a parcel of nonsensical}
\stck{contingencies\tsk \i{c’est bien comique}, said}
she\tsk 

\tsk There wants nothing, said I, to\break
\stick{make it so, but the comick use}
\stick{which the gallantry of a French-}
\stick{man would put it to\tsk to make love}
\stick{the first moment, and an offer of his}
person the second.

\patch[\baselineskip]{’Tis}

’Tis their \i{fort:} replied the lady.

It is supposed so at least\tsk and how
\stick{it has come to pass, continued I, I}
\stick{know not; but they have certainly}
\stick{got the credit of understanding more}
\stick{of love, and making it better than}
\stick{any other nation upon earth: but for}
\stick{my own part I think them errant}
\stick{bunglers, and in truth the worst set}
\stick{of marksmen that ever tried Cupid’s}
patience.

\tsk To think of making love by\break
\i{sentiments}!

I should as soon think of making\break
\stick{a genteel suit of cloaths out of rem-}
\stick{nants:\tsk and to do it\tsk pop\tsk at first}
\stick{sight by declaration\tsk is submitting}
\catch{the}
\stick{the offer and themselves with it, to}
\stick{be sifted with all their \i{pours} and}
\i{contres}, by an unheated mind.

The lady attended as if she ex-\break
pected I should go on.

Consider then, madam, continued\break
I, laying my hand upon hers\tsk 

That grave people hate Love for\break
the name’s sake\tsk 

That selfish people hate it for their\break
own\tsk 

Hypocrites for heaven’s\tsk 

And that all of us both old\break
\stick{and young, being ten times worse}
\catch{frighten’d}
\stick{frighten’d than hurt by the very}
\stick{\i{report}\tsk What a want of knowledge}
\stick{in this branch of commerce a man}
\stick{betrays, whoever lets the word}
\stick{come out of his lips, till an hour or}
\stick{two at least after the time, that his}
\stick{silence upon it becomes tormenting.}
\stick{A course of small, quiet attentions,}
\stick{not so pointed as to alarm\tsk nor so}
\stick{vague as to be misunderstood,\tsk with}
\stick{now and then a look of kindness,}
\stick{and little or nothing said upon it\tsk}
\stick{leaves Nature for your mistress, and}
she fashions it to her mind.\tsk 

Then I solemnly declare, said the
\stick{lady, blushing\tsk you have been mak-}
ing love to me all this while.

\newpage

\centerline{\tstrut 24pt\lsv{40}{THE REMISE}.}
\head{24pt}{CALAIS}

\vskip 8pt

\dropcap{M}{onsieur} \i{Dessein} came\break
back to let us out of the\break
\stick{chaise, and acquaint the lady, the}
\stick{Count de L\bzz{ongines} her brother was}
\stick{just arrived at the hotel. Though I}
\stick{had infinite good will for the lady,}
\stick{I cannot say, that I rejoiced in my}
\stick{heart at the event\tsk and could not}
\stick{help telling her so\tsk for it is fatal to}
\stick{a proposal, Madam, said I, that I}
was going to make to you\tsk 

\tsk You need not tell me what the\break
\stick{proposal was, said she, laying her}
\catch{hand}
\stick{hand upon both mine, as she inter-}
\stick{rupted me.\tsk A man, my good Sir,}
\stick{has seldom an offer of kindness to}
\stick{make to a woman, but she has a}
\stick{presentiment of it some moments}
before\tsk 

Nature arms her with it, said I,\break
\stick{for immediate preservation\tsk But I}
\stick{think, said she, looking in my face,}
\stick{I had no evil to apprehend\tsk and to}
\stick{deal frankly with you, had deter-}
\stick{mined to accept it.\tsk If I had\tsk (she}
\stick{stopped a moment)\tsk I believe your}
\stick{good will would have drawn a story}
\stick{from me, which would have made}
\stick{pity the only dangerous thing in the}
journey.


\patchv[\baselineskip]{G}{In}

In saying this, she suffered me to 
\stick{kiss her hand twice, and with a look}
\stick{of sensibility mixed with concern}
\stick{she got out of the chaise\tsk and bid}
adieu.

\newpage
\centerline{\tstrut 24pt\lsv{24}{IN THE STREET}.}
\head{18pt}{CALAIS}

\vskip 8pt

\initial[loversize=-0.07,lraise=0.22,nindent=0pt]{I}{NEVER} finished a twelve-guinea\break
bargain so expeditiously in my\break
\stick{life: my time seemed heavy upon}
\stick{the loss of the lady, and knowing}
\stick{every moment of it would be as}
\stick{two, till I put myself into motion}
\stick{\tsk I ordered post horses directly, and}
walked towards the hotel.

Lord!\@ said I, hearing the town\break
\stick{clock strike four, and recollecting}
\stick{that I had been little more than}
a single hour in Calais\tsk 

\patchs[\baselineskip]{G 2}{\tsk What}

\tsk What a large volume of ad-\break
\stick{ventures may be grasped within this}
\stick{little span of life by him who in-}
\stick{terests his heart in every thing, and}
\stick{who, having eyes to see what time}
\stick{and chance are perpetually holding}
\stick{out to him as he journeyeth on his}
\stick{way, misses nothing he can \i{fairly} lay}
his hands on.\tsk

\tsk If this won’t turn out something\break
\stick{\tsk another will\tsk no matter\tsk ’tis an}
\stick{assay upon human nature\tsk I get my}
\stick{labour for my pains\tsk ’tis enough\tsk}
\stick{the pleasure of the experiment has}
\stick{kept my senses and the best part of}
\stick{my blood awake, and laid the gross}
to sleep.

\patch[\baselineskip]{I}

I pity the man who can travel\break
\stick{from \i{Dan} to \i{Beersheba}, and cry, ’Tis}
\stick{all barren\tsk and so it is; and so is all}
\stick{the world to him who will not culti-}
\stick{vate the fruits it offers.  I declare,}
\stick{said I, clapping my hands chearily}
\stick{together, that were I in a desart,\sic I}
\stick{would find out wherewith in it to call}
\stick{forth my affections\tskk If I could not}
\stick{do better, I would fasten them upon}
\stick{some sweet myrtle, or seek some me-}
\stick{lancholy cypress to connect myself}
\stick{to\tsk I would court their shade, and}
\stick{greet them kindly for their protec-}
\stick{tion\tsk I would cut my name upon}
\stick{them, and swear they were the love-}
\stick{liest trees throughout the desert: if}
\stick{their leaves wither’d, I would teach}
\catchs{G 3}{myself}
\stick{myself  to mourn, and when they re-}
\stick{joiced, I would rejoice along with}
them.

The learned \s{Smelfungus} travel-\break
\stick{led from Boulogne to Paris\tsk from}
\stick{Paris to Rome\tsk and so on\tsk but he}
\stick{set out with the spleen and jaundice,}
\stick{and every object he pass’d by was}
\stick{discoloured or distorted\tsk He wrote}
\stck{an account of them, but ’twas nothing}
\stick{but the account of his miserable feel-}
ings.

I met Smelfungus in the grand\break
\stick{portico of the Pantheon\tsk he was just}
\stick{coming out of it\tsk \i{’Tis nothing but}}
\stick{\i{a huge cockpit}\fnast, said he\tsk I wish you}
\stick{had said nothing worse of the Venus}
\stick{\hfill{\footnotesize\fnast\ Vide S\bzz{mollet}’s Travels.}\hfill}
\catch{of}
\stick{of Medicis, replied I\tsk for in passing}
\stick{through Florence, I had heard he}
\stick{had fallen foul upon the goddess,}
\stick{and used her worse than a common}
\stick{strumpet, without the least provo-}
cation in nature.

I popp’d upon Smelfungus again\break
\stick{at Turin, in his return home; and}
\stick{a sad tale of sorrowful adventures}
\stick{had he to tell, \lqq wherein he spoke of}
\stick{\lqq moving accidents by flood and field,}
\stick{\lqq and of the cannibals which each}
\stick{\lqq other eat: the Anthropophagi\rqq\tsk}
\stck{he had been flea’d alive, and bedevil’d,}
\stick{and used worse than St.\@ Bartholo-}
\stick{mew, at every stage he had come}
at\tsk

\patchs[\baselineskip]{G 4}{\tsk I’ll}

\tsk I’ll tell it, cried Smelfungus, to\break
\stick{the world.  You had better tell it,}
said I, to your physician.

Mundungus, with an immense for-\break
\stick{tune, made the whole tour; going}
\stick{on from Rome to Naples\tsk from}
\stick{Naples to Venice\tsk from Venice to}
\stick{Vienna\tsk to Dresden, to Berlin, with-}
\stk{out one generous connection or pleasur-}
\stick{able anecdote to tell of; but he had}
\stick{travell’d straight on looking neither}
\stick{to his right hand nor his left, lest}
\stick{Love or Pity should seduce him out}
of his road.

Peace be to them!\@ if it is to be\break
\stick{found; but heaven itself, was it pos-}
\stick{sible to get there with such tempers,}
\catch{would}
\stick{would want objects to give it\tsk every}
\stick{gentle spirit would come flying upon}
\stick{the wings of Love to hail their ar-}
\stick{rival\tsk Nothing would the souls of}
\stick{Smelfungus and Mundungus hear of,}
\stick{but fresh anthems of joy, fresh rap-}
\stick{tures of love, and fresh congratula-}
\stick{tions of their common felicity\tskk I}
\stck{heartily pity them: they have brought}
\stick{up no faculties for this work; and}
\stick{was the happiest mansion in heaven}
\stick{to be allotted to Smelfungus and}
\stick{Mundungus, they would be so far}
\stick{from being happy, that the souls of}
\stick{Smelfungus and Mundungus would}
do penance there to all eternity.

\newpage


\head{36pt}{MONTRIUL}

\vskip 8pt

\initial[nindent=0.5pt]{I}{HAD} once lost my portmanteau\break 
from behind my chaise, and twice\break
\stick{got out in the rain, and one of the}
\stick{times up to the knees in dirt, to help}
\stick{the postilion to tie it on, without be-}
\stick{ing able to find out what was want-}
\stick{ing\tsk Nor was it till I got to Mon-}
\stick{triul, upon the landlord’s asking me}
\stick{if I wanted not a servant, that it oc-}
\stick{curred to me, that that was the very}
thing.

A servant!  That I do most sadly,\break
\stick{quoth I\tsk Because, Monsieur, said the}
\stick{landlord, there is a clever young fel-}
\catch{low,}
\stick{low, who would be very proud of the}
\stick{honour to serve an Englishman\tsk But}
\stick{why an English one, more than any}
\stick{other?\tsk They are so generous, said}
\stick{the landlord\tsk I’ll be shot if this is not}
\stick{a livre out of my pocket, quoth I to}
\stick{myself, this very night\tskk But they}
\stick{have wherewithal to be so, Monsieur,}
\stick{added he\tskk Set down one livre more}
\stick{for that, quoth I\tskk It was but last}
\stick{night, said the landlord, \i{qu’un my}}
\stick{\i{Lord Anglois presentoit un ecu a la fille}}
\stick{\i{de chambre\tskk Tant pis, pour Mad}\raise0.6ex\hbox{\scriptsize\itshape lle}}
\i{Janatone}, said I.

Now Janatone being the land-\break
\stick{lord’s daughter, and the landlord}
\stick{supposing I was young in French,}
\stck{took the liberty to inform me, I should}
\catch{not}
\stick{not have said \i{tant pis}\tskk but, \i{tant}}
\stick{\i{mieux}.  \i{Tant mieux}, \i{toujours}, \i{Monsieur},}
\stick{said he, when there is any thing to be}
\stick{got\tsk \i{tant pis}, when there is nothing.}
\stick{It comes to the same thing, said I.}
\i{Pardonnez-moi}, said the landlord.

I cannot take a fitter opportunity\break
\stick{to observe once for all, that \i{tant pis}}
\stick{and \i{tant mieux} being two of the great}
\stck{hinges in French conversation, a stran-}
\stick{ger would do well to set himself right}
\stick{in the use of them, before he gets to}
Paris.

A prompt French Marquis at our\break
\stick{ambassador’s table demanded of Mr.}
\stick{H\bzz{orace}, if he was H\bzz{orace} the poet?}
\catch{No,}
\stick{No, said H\bzz{orace} mildly\tskk \i{Tant pis},}
replied the Marquis.

It is H\bzz{orace} the historian, said an-\break
\stick{other\tsk \i{Tant mieux}, said the Marquis.}
\stick{And Mr.\@ H\bzz{orace}, who is a man of}
\stick{an excellent heart, return’d thanks}
for both.

When the landlord had set me\break
\stick{right in this matter, he called in La}
\stick{Fleur, which was the name of the}
\stick{young man he had spoke of\tsk saying}
\stick{only first, That as for his talents, he}
\stck{would presume to say nothing\tsk Mon-}
\stick{sieur was the best judge what would}
\stick{suit him; but for the fidelity of La}
\stick{Fleur, he would stand responsible in}
all he was worth.

\patch[8pt]{The}

The landlord deliver’d this in a\break
\stick{manner which instantly set my mind}
\stick{to the business I was upon\tsk and La}
\stick{Fleur, who stood waiting without, in}
\stk{that breathless expectation which every}
\stick{son of nature of us have felt in our}
turns, came in.

\newpage


\head{36pt}{MONTRIUL}

\vskip 8pt

\initial[loversize=-0.07,lraise=0.22,nindent=0.3pt]{I}{AM} apt to be taken with all kinds
of people at first sight; but never\break
\stick{more so, than when a poor devil}
\stick{comes to offer his service to so poor a}
\stick{devil as myself; and as I know this}
\stick{weakness, I always suffer my judg-}
\stick{ment to draw back something on that}
\stick{very account\tsk and this more or less,}
\stick{according to the mood I am in, and}
\stick{the case\tsk and I may add the gender}
too, of the person I am to govern.

When La Fleur enter’d the room,\break
\stick{after every discount I could make for}
\stick{my soul, the genuine look and air of}
\catch{the}
\stick{the fellow determined the matter at}
\stick{once in his favour; so I hired him}
\stick{first\tsk and then began to inquire what}
\stick{he could do: But I shall find out his}
\stick{talents, quoth I, as I want them\tsk}
\stick{besides, a Frenchman can do every}
thing.

Now poor La Fleur could do no-\break
\stick{thing in the world but beat a drum,}
\stick{and play a march or two upon the}
\stick{fife. I was determined to make his}
\stick{talents do; and can’t say my weak-}
\stick{ness was ever so insulted by my wis-}
dom as in the attempt.

La Fleur had set out early in life,\break
\stick{as gallantly as most Frenchmen do,}
\stck{with \i{serving} for a few years; at the end}
\catch{of}
\stick{of which, having satisfied the senti-}
\stick{ment, and found moreover, That the}
\stick{honour of beating a drum was likely}
\stick{to be its own reward, as it open’d no}
\stick{further track of glory to him\tsk he}
\stick{retired \i{a ses terres}, and lived \i{comme}}
\stick{\i{il plaisoit a Dieu}\tsk that is to say, upon}
nothing.

\tsk And so, quoth \i{Wisdome}, you\break
\stick{have hired a drummer to attend you}
\stick{in this tour of your’s thro’ France}
\stick{and Italy! Psha! said I, and do}
\stick{not one half of our gentry go with a}
\stick{hum-drum \i{compagnon du voiage} the}
\stick{same round, and have the piper and}
\stick{the devil and all to pay besides?}
\stick{When man can extricate himself with}
\catchv{H}{an}
\stck{an \i{equivoque} in such an unequal match}
\stick{\tsk he is not ill of\sic\tsk But you can do}
\stick{something else, La Fleur? said I\tsk}
\stick{\tsk\i{O qu’oui!} he could make spat-}
\stick{terdashes, and play a little upon the}
\stick{fiddle\tsk Bravo! said Wisdome\tsk Why,}
\stick{I play a bass myself, said I\tsk we shall}
\stick{do very well.\tskk  You can shave, and}
\stick{dress a wig a little, La Fleur?\tsk He}
\stick{had all the dispositions in the world\tsk}
\stick{It is enough for heaven!\@ said I, in-}
\stick{terrupting him\tsk and ought to be}
\stick{enough for me\tsk So, supper coming}
\stick{in, and having a frisky English spa-}
\stick{niel on one side of my chair, and a}
\stick{French valet, with as much hilarity}
\stick{in his countenance as ever nature}
\stick{painted in one, on the other\tsk I was}
\catch{satisfied}
\stick{satisfied to my heart’s content with}
\stick{my empire; and if monarchs knew}
\stick{what they would be at, they might}
be as satisfied as I was.

\patchs[12\baselineskip]{H 2}{}


\head{36pt}{MONTRIUL}

\vskip 8pt

\dropcap{A}{s} La Fleur went the whole tour 
of France and Italy with me,\break
\stick{and will be often upon the stage, I}
\stick{must interest the reader a little fur-}
\stick{ther in his behalf, by saying, that I}
\stick{had never less reason to repent of the}
\stick{impulses which generally do deter-}
\stick{mine me, than in regard to this fel-}
\stick{low\tsk he was a faithful, affectionate,}
\stick{simple soul as ever trudged after the}
\stick{heels of a philosopher; and, notwith-}
\stick{standing his talents of drum-beating}
\stick{and spatterdash-making, which, tho’}
\stick{very good in themselves, happen’d}
\stick{to be of no great service to me, yet}
\catch{was}
\stick{was I hourly recompenced by the}
\stick{festivity of his temper\tsk it supplied}
\stick{all defects\tsk I had a constant resource}
\stick{in his looks in all difficulties and dis-}
\stick{tresses of my own\tsk I was going to}
\stick{have added, of his too; but La Fleur}
\stick{was out of the reach of every thing;}
\stick{for whether ’twas hunger or thirst,}
\stick{or cold or nakedness, or watchings,}
\stick{or whatever stripes of ill luck La}
\stick{Fleur met with in our journeyings,}
\stck{there was no index in his physiognomy}
\stick{to point them out by\tsk he was eter-}
\stick{nally the same; so that if I am a piece}
\stick{of a philosopher, which Satan now}
\stick{and then puts it into my head I am\tsk}
\stick{it always mortifies the pride of the}
\stick{conceit, by reflecting how much I}
\stick{owe to the complexional philosophy}
\catchs{H 3}{of}
\stick{of this poor fellow, for shaming me}
\stick{into one of a better kind.  With all}
\stick{this, La Fleur had a small cast of}
\stick{the coxcomb\tsk but he seemed at first}
\stick{sight to be more a coxcomb of nature}
\stick{than of art; and before I had been}
\stick{three days in Paris with him\tsk he}
seemed to be no coxcomb at all.

\newpage

\head{18pt}{MONTRIUL}

\vskip 0pt

\dropcap{T}{HE} next morning, La Fleur\break
entering upon his employ-\break
\stick{ment, I delivered to him the key of}
\stck{my portmanteau with an inventory of}
\stick{my half a dozen shirts and silk pair}
\stick{of breeches; and bid him fasten all}
\stick{upon the chaise\tsk get the horses put}
\stick{to\tsk and desire the landlord to come}
in with his bill.

\i{C’est un garcon de bonne fortune},\break
\stick{said the landlord, pointing through}
\stick{the window to half a dozen wenches}
\stick{who had got round about La Fleur,}
\stick{and were most kindly taking their}
\stick{leave of him, as the postilion was}
\stick{leading out the horses.  La Fleur}
\stick{kissed all their hands round and}
\catchs{H 4}{round}
\stick{round again, and thrice he wiped}
\stick{his eyes, and thrice he promised he}
\stick{would bring them all pardons from}
Rome.

The young fellow, said the land-\break
\stick{lord, is beloved by all the town, and}
\stick{there is scarce a corner in Montriul}
\stick{where the want of him will not be}
\stick{felt: he has but one misfortune in}
\stick{the world, continued he, \lqq He is}
\stick{\lqq always in love.\rqq\tsk I am heartily}
\stick{glad of it, said I,\tsk ’twill save me the}
\stick{trouble every night of putting my}
\stick{breeches under my head.  In saying}
\stick{this, I was making not so much La}
\stick{Fleur’s eloge, as my own, having}
\stick{been in love with one princess or}
\stick{another almost all my life, and I}
\catch{hope}
\stick{hope I shall go on so, till I die,}
\stick{being firmly persuaded, that if ever}
\stick{I do a mean action, it must be in}
\stick{some interval betwixt one passion}
\stick{and another: whilst this interregnum}
\stick{lasts, I always perceive my heart}
\stick{locked up\tsk I can scarce find in it,}
\stick{to give Misery a sixpence; and there-}
\stick{fore I always get out of it as fast as}
\stick{I can, and the moment I am re-}
\stick{kindled, I am all generosity and}
\stick{good will again; and would do any}
\stick{thing in the world either for, or}
\stick{with any one, if they will but satisfy}
me there is no sin in it.

\tsk But in saying this\tsk surely I am\break
\stick{commanding the passion\tsk not my-}
self.

\patch{A}


\head{28pt}{A\enspace FRAGMENT}

\tsh \lsss{THE} town of Abdera,\break
\stick{notwithstanding Democritus lived}
\stick{there trying all the powers of irony}
\stick{and laughter to reclaim it, was the}
\stick{vilest and most profligate town in all}
\stick{Thrace.  What for poisons, conspi-}
\stck{racies and assassinations\tsk libels, pas-}
\stick{quinades and tumults, there was no}
\stick{going there by day\tsk ’twas worse by}
night.

Now, when things were at the\break
\stick{worst, it came to pass, that the An-}
\stick{dromeda of Euripides being repre-}
\stck{sented at Abdera, the whole orchestra}
\stick{was delighted with it: but of all the}
\catch{passages}
\stick{passages which delighted them, no-}
\stick{thing operated more upon their ima-}
\stick{ginations, than the tender strokes of}
\stick{nature which the poet had wrought}
\stick{up in that pathetic speech of Perseus,}
\stick{\tstrut 18pt\hfill \i{O Cupid}, \i{prince of God\sic and men}, \&c.\hfill}
\stick{\tstrut 18pt Every man almost spoke pure iam-}
\stick{bics the next day, and talk’d of no-}
\stick{thing but Perseus his pathetic ad-}
\stick{dress\tsk \lqq O Cupid! prince of God}
\stick{\lqq and men\rqq\tsk in every street of Ab-}
\stick{dera, in every house\tsk \lqq O Cupid!}
\stick{\lqq Cupid!\rqq\tsk in every mouth, like the}
\stick{natural notes of some sweet melody}
\stick{which drop from it whether it will}
\stick{or no\tsk nothing but \lqq Cupid!  Cupid!}
\stick{\lqq prince of God and men\rqq\tsk The}
\stick{fire caught\tsk and the whole city, like}
\catch{the}
\stick{the heart of one man, open’d itself}
to Love.

No pharmacopolist could sell one\break
\stick{grain of hellebore\tsk not a single ar-}
\stick{mourer had a heart to forge one in-}
\stick{strument of death\tsk Friendship and}
\stick{Virtue met together, and kiss’d each}
\stick{other in the street\tsk the golden age}
\stick{return’d, and hung o’er the town of}
\stick{Abdera\tsk every Abderite took his}
\stick{oaten pipe, and every Abderitish}
\stick{woman left her purple web, and}
\stick{chastly sat her down and listen’d to}
the song\tsk

’Twas only in the power, says the\break
\stick{Fragment, of the God whose empire}
\catch{extendeth}
\stick{extendeth from heaven to earth, and}
\stick{even to the depths of the sea, to have}
done this.

\newpage

\head{36pt}{MONTRIUL}

\vskip 8pt

\dropcap{W}{HEN} all is ready, and every 
article is disputed and paid\break
\stick{for in the inn, unless you are a little}
\stick{sour’d by the adventure, there is al-}
\stick{ways a matter to compound at the}
\stick{door, before you can get into your}
\stick{chaise; and that is with the sons and}
\stick{daughters of poverty, who surround}
\stick{you.  Let no man say, \lqq let them}
\stick{\lqq go to the devil!\rqq\tsk ’tis a cruel jour-}
\stick{ney to send a few miserables, and}
\stick{they have had sufferings enow with-}
\stick{out it: I always think it better to}
\stick{take a few sous out in my hand; and}
\stick{I would counsel every gentle travel-}
\catch{ler}
\stick{ler to do so likewise: he need not be}
\stick{so exact in setting down his motives}
\stick{for giving them\tsk they will be re-}
gister’d elsewhere.

For my own part, there is no man\break
\stick{gives so little as I do; for few that I}
\stick{know have so little to give: but as}
\stick{this was the first publick act of my}
\stick{charity in France, I took the more}
notice of it.

A well-a-way!\@ said I.\quad  I have but\break
\stck{eight sous in the world, shewing them}
\stick{in my hand, and there are eight poor}
men and eight poor women for ’em.

A poor tatter’d soul without a shirt\break
\stick{on instantly withdrew his claim, by}
\catch{retiring}
\stick{retiring two steps out of the circle,}
\stick{and making a disqualifying bow on}
\stick{his part.  Had the whole parterre}
\stick{cried out, \i{Place aux dames}, with one}
\stick{voice, it would not have conveyed}
\stick{the sentiment of a deference for the}
sex with half the effect.

Just heaven!\@ for what wise reasons\break
\stick{hast thou order’d it, that beggary and}
\stick{urbanity, which are at such variance}
\stick{in other countries, should find a way}
to be at unity in this?

\tsk I insisted upon presenting him\break
\stick{with a single sous, merely for his \i{po}-}
\i{litesse}.

\patch{A}

A poor little dwarfish brisk fellow,\break
\stick{who stood over against me in the}
\stick{circle, putting something first under}
\stick{his arm, which had once been a hat,}
\stick{took his snuff-box out of his pocket,}
\stick{and generously offer’d a pinch on}
\stick{both sides of him: it was a gift of}
\stick{consequence, and modestly declined}
\stick{\tsk The poor little fellow press’d it}
\stck{upon them with a nod of welcomeness}
\stick{\tsk \i{Prenez en}\tsk \i{prenez}, said he, look-}
\stick{ing another way; so they each took}
\stick{a pinch\tsk Pity thy box should ever}
\stick{want one!\@ said I to myself; so I put}
\stick{a couple of sous into it\tsk taking a}
\stick{small pinch out of his box, to en-}
\stick{hance their value, as I did it\tsk He}
\stick{felt the weight of the second obliga-}
\stick{tion more than that of the first\tsk ’twas}
\catchv{I}{doing}
\stick{doing him an honour,\tsk the other was}
\stick{only doing him a charity\tsk and he}
\stick{made me a bow down to the ground}
for it.

\tsk Here! said I to an old soldier\break
\stick{with one hand, who had been cam-}
\stick{paign’d and worn out to death in the}
\stick{service\tsk here’s a couple of sous for}
\stick{thee\tsk \i{Vive le Roi!} said the old sol-}
dier.

I had then but three sous left: so\break
\stick{I gave one, simply, \i{pour l’amour de}}
\stick{\i{Dieu}, which was the footing on which}
\stick{it was begg’d\tsk The poor woman had}
\stick{a dislocated hip; so it could not be}
well, upon any other motive.

\patch{\i{Mon}}

\i{Mon cher et tres charitable Mon}-\break
\stk{\i{sieur}\tsk There’s no opposing this, said I.}

\i{My Lord Anglois}\tsk the very sound\break
\stick{was worth the money\tsk so I gave \i{my}}
\stick{\i{last sous for it}.  But in the eagerness}
\stick{of giving, I had overlook’d a \i{pauvre}}
\stick{\i{honteux}, who had no one to ask a sous}
\stick{for him, and who, I believed, would}
\stick{have perish’d, ere he could have ask’d}
\stick{one for himself: he stood by the}
\stick{chaise a little without the circle, and}
\stick{wiped a tear from a face which I}
\stick{thought had seen better days\tsk Good}
\stick{God!\@ said I\tsk and I have not one}
\stick{single sous left to give him\tsk But you}
\stick{have a thousand!\@ cried all the powers}
\stick{of nature, stirring within me\tsk so I}
\stick{gave him\tsk no matter what\tsk I am}
\catchs{I 2}{ashamed}
\stick{ashamed to say \i{how much}, now\tsk and}
\stick{was ashamed to think, how little,}
\stick{then: so if the reader can form any}
\stick{conjecture of my disposition, as these}
\stick{two fixed points are given him, he}
\stick{may judge within a livre or two what}
was the precise sum.

I could afford nothing for the rest,\break
\stick{but, \i{Dieu vous benisse}\tsk \i{Et le bon Dieu}}
\stick{\i{vous benisse encore}\tsk said the old sol-}
\stick{dier, the dwarf, \&c.  The \i{pauvre}}
\stick{\i{honteux} could say nothing\tsk he pull’d}
\stick{out a little handkerchief, and wiped}
\stick{his face as he turned away\tsk and I}
\stick{thought he thank’d me more than}
them all.

\newpage



\head{24pt}{THE BIDET}


\dropcap{H}{AVING} settled all these little matters, I got into my post-chaise with
more ease than ever I got into a post-chaise in my life; and La Fleur
having got one large jack-boot on the far side of a little \i{bidet}, {588}
and another on this (for I count nothing of his legs)\tsk he canter’d away
before me as happy and as perpendicular as a prince.\tsk But what is
happiness! what is grandeur in this painted scene of life!  A dead ass,
before we had got a league, put a sudden stop to La Fleur’s career\tsk his
bidet would not pass by it\tsk a contention arose betwixt them, and the poor
fellow was kick’d out of his jack-boots the very first kick.

La Fleur bore his fall like a French Christian, saying neither more nor
less upon it, than \i{Diable}!  So presently got up, and came to the charge
again astride his bidet, beating him up to it as he would have beat his
drum.

The bidet flew from one side of the road to the other, then back
again\tsk then this way, then that way, and in short, every way but by the
dead ass\tsk La Fleur insisted upon the thing\tsk and the bidet threw him.

What’s the matter, La Fleur, said I, with this bidet of thine?  Monsieur,
said he, \i{c’est un cheval le plus opiniâtre du monde}.\tsk Nay, if he is a
conceited beast, he must go his own way, replied I.  So La Fleur got off
him, and giving him a good sound lash, the bidet took me at my word, and
away he scampered back to Montriul.\tsk \i{Peste}! said La Fleur.

It is not \i{mal-à-propos} to take notice here, that though La Fleur
availed himself but of two different terms of exclamation in this
encounter\tsk namely, \i{Diable}! and \i{Peste}! that there are, nevertheless,
three in the French language: like the positive, comparative, and
superlative, one or the other of which serves for every unexpected throw
of the dice in life.

\i{Le Diable}! which is the first, and positive degree, is generally used
upon ordinary emotions of the mind, where small things only fall out
contrary to your expectations; such as\tsk the throwing once doublets\tsk La
Fleur’s being kick’d off his horse, and so forth.\tsk Cuckoldom, for the same
reason, is always\tsk \i{Le Diable}!

But, in cases where the cast has something provoking in it, as in that of
the bidet’s running away after, and leaving La Fleur aground in
jack-boots\tsk ’tis the second degree.

’Tis then \i{Peste}!

And for the third\tsk 

\tsk But here my heart is wrung with pity and fellow feeling, when I reflect
what miseries must have been their lot, and how bitterly so refined a
people must have smarted, to have forced them upon the use of it.\tsk 

Grant me, O ye powers which touch the tongue with eloquence in
distress!\tsk what ever is my \i{cast}, grant me but decent words to exclaim
in, and I will give my nature way.

\tsk But as these were not to be had in France, I resolved to take every evil
just as it befell me, without any exclamation at all.

La Fleur, who had made no such covenant with himself, followed the bidet
with his eyes till it was got out of sight\tsk and then, you may imagine, if
you please, with what word he closed the whole affair.

As there was no hunting down a frightened horse in jack-boots, there
remained no alternative but taking La Fleur either behind the chaise, or
into it.\tsk 

I preferred the latter, and in half an hour we got to the post-house at
Nampont.




\head{24pt}{NAMPONT}
\head{24pt}{THE DEAD ASS}


\tsk AND this, said he, putting the remains of a crust into his wallet\tsk and
this should have been thy portion, said he, hadst thou been alive to have
shared it with me.\tsk I thought, by the accent, it had been an apostrophe to
his child; but ’twas to his ass, and to the very ass we had seen dead in
the road, which had occasioned La Fleur’s misadventure.  The man seemed
to lament it much; and it instantly brought into my mind Sancho’s
lamentation for his; but he did it with more true touches of nature.

The mourner was sitting upon a stone bench at the door, with the ass’s
pannel and its bridle on one side, which he took up from time to
time\tsk then laid them down,\tsk look’d at them, and shook his head.  He then
took his crust of bread out of his wallet again, as if to eat it; held it
some time in his hand\tsk then laid it upon the bit of his ass’s
bridle\tsk looked wistfully at the little arrangement he had made\tsk and then
gave a sigh.

The simplicity of his grief drew numbers about him, and La Fleur amongst
the rest, whilst the horses were getting ready; as I continued sitting in
the post-chaise, I could see and hear over their heads.

\tsk He said he had come last from Spain, where he had been from the furthest
borders of Franconia; and had got so far on his return home, when his ass
died.  Every one seemed desirous to know what business could have taken
so old and poor a man so far a journey from his own home.

It had pleased heaven, he said, to bless him with three sons, the finest
lads in Germany; but having in one week lost two of the eldest of them by
the small-pox, and the youngest falling ill of the same distemper, he was
afraid of being bereft of them all; and made a vow, if heaven would not
take him from him also, he would go in gratitude to St. Iago in Spain.

When the mourner got thus far on his story, he stopp’d to pay Nature her
tribute\tsk and wept bitterly.

He said, heaven had accepted the conditions; and that he had set out from
his cottage with this poor creature, who had been a patient partner of
his journey\tsk that it had eaten the same bread with him all the way, and
was unto him as a friend.

Every body who stood about, heard the poor fellow with concern.\tsk La Fleur
offered him money.\tsk The mourner said he did not want it\tsk it was not the
value of the ass\tsk but the loss of him.\tsk The ass, he said, he was assured,
loved him\tsk and upon this told them a long story of a mischance upon their
passage over the Pyrenean mountains, which had separated them from each
other three days; during which time the ass had sought him as much as he
had sought the ass, and that they had scarce either eaten or drank till
they met.

Thou hast one comfort, friend, said I, at least, in the loss of thy poor
beast; I’m sure thou hast been a merciful master to him.\tsk Alas! said the
mourner, I thought so when he was alive\tsk but now that he is dead, I think
otherwise.\tsk I fear the weight of myself and my afflictions together have
been too much for him\tsk they have shortened the poor creature’s days, and
I fear I have them to answer for.\tsk Shame on the world! said I to
myself.\tsk Did we but love each other as this poor soul loved his
ass\tsk ’twould be something.\tsk 




\head{24pt}{NAMPONT}
\head{24pt}{THE POSTILION}


\dropcap{T}{HE} concern which the poor fellow’s story threw me into required some
attention; the postilion paid not the least to it, but set off upon the
\i{pavé} in a full gallop.

The thirstiest soul in the most sandy desert of Arabia could not have
wished more for a cup of cold water, than mine did for grave and quiet
movements; and I should have had an high opinion of the postilion had he
but stolen off with me in something like a pensive pace.\tsk On the contrary,
as the mourner finished his lamentation, the fellow gave an unfeeling
lash to each of his beasts, and set off clattering like a thousand
devils.

I called to him as loud as I could, for heaven’s sake to go slower\tsk and
the louder I called, the more unmercifully he galloped.\tsk The deuce take
him and his galloping too\tsk said I\tsk he’ll go on tearing my nerves to pieces
till he has worked me into a foolish passion, and then he’ll go slow that
I may enjoy the sweets of it.

The postilion managed the point to a miracle: by the time he had got to
the foot of a steep hill, about half a league from Nampont\tsk he had put me
out of temper with him\tsk and then with myself, for being so.

My case then required a different treatment; and a good rattling gallop
would have been of real service to me.\tsk 

\tsk Then, prithee, get on\tsk get on, my good lad, said I.

The postilion pointed to the hill.\tsk I then tried to return back to the
story of the poor German and his ass\tsk but I had broke the clue\tsk and could
no more get into it again, than the postilion could into a trot.

\tsk The deuce go, said I, with it all!  Here am I sitting as candidly
disposed to make the best of the worst, as ever wight was, and all runs
counter.

There is one sweet lenitive at least for evils, which Nature holds out to
us: so I took it kindly at her hands, and fell asleep; and the first word
which roused me was \i{Amiens}.

\tsk Bless me! said I, rubbing my eyes\tsk this is the very town where my poor
lady is to come.




\head{24pt}{AMIENS}


\dropcap{T}{HE} words were scarce out of my mouth when the Count de L\tsk ’s post-chaise,
with his sister in it, drove hastily by: she had just time to make me a
bow of recognition\tsk and of that particular kind of it, which told me she
had not yet done with me.  She was as good as her look; for, before I had
quite finished my supper, her brother’s servant came into the room with a
billet, in which she said she had taken the liberty to charge me with a
letter, which I was to present myself to Madame R\tsk  the first morning I
had nothing to do at Paris.  There was only added, she was sorry, but
from what \i{penchant} she had not considered, that she had been prevented
telling me her story\tsk that she still owed it to me; and if my route
should ever lay through Brussels, and I had not by then forgot the name
of Madame de L\tsk \tsk that Madame de L\tsk  would be glad to discharge her
obligation.

Then I will meet thee, said I, fair spirit! at Brussels\tsk ’tis only
returning from Italy through Germany to Holland, by the route of
Flanders, home\tsk ’twill scarce be ten posts out of my way; but, were it
ten thousand! with what a moral delight will it crown my journey, in
sharing in the sickening incidents of a tale of misery told to me by such
a sufferer?  To see her weep! and, though I cannot dry up the fountain of
her tears, what an exquisite sensation is there still left, in wiping
them away from off the cheeks of the first and fairest of women, as I’m
sitting with my handkerchief in my hand in silence the whole night beside
her?

There was nothing wrong in the sentiment; and yet I instantly reproached
my heart with it in the bitterest and most reprobate of expressions.

It had ever, as I told the reader, been one of the singular blessings of
my life, to be almost every hour of it miserably in love with some one;
and my last flame happening to be blown out by a whiff of jealousy on the
sudden turn of a corner, I had lighted it up afresh at the pure taper of
Eliza but about three months before\tsk swearing, as I did it, that it
should last me through the whole journey.\tsk Why should I dissemble the
matter?  I had sworn to her eternal fidelity\tsk she had a right to my whole
heart\tsk to divide my affections was to lessen them\tsk to expose them was to
risk them: where there is risk there may be loss\tsk and what wilt thou
have, Yorick, to answer to a heart so full of trust and confidence\tsk so
good, so gentle, and unreproaching!

\tsk I will not go to Brussels, replied I, interrupting myself.\tsk But my
imagination went on\tsk I recalled her looks at that crisis of our
separation, when neither of us had power to say adieu!  I look’d at the
picture she had tied in a black riband about my neck\tsk and blush’d as I
look’d at it.\tsk I would have given the world to have kiss’d it\tsk but was
ashamed.\tsk And shall this tender flower, said I, pressing it between my
hands\tsk shall it be smitten to its very root,\tsk and smitten, Yorick! by
thee, who hast promised to shelter it in thy breast?

Eternal Fountain of Happiness! said I, kneeling down upon the ground\tsk be
thou my witness\tsk and every pure spirit which tastes it, be my witness
also, That I would not travel to Brussels, unless Eliza went along with
me, did the road lead me towards heaven!

In transports of this kind, the heart, in spite of the understanding,
will always say too much.




\head{24pt}{THE LETTER}
\head{24pt}{AMIENS}


\dropcap{F}{ORTUNE} had not smiled upon La Fleur; for he had been unsuccessful in his
feats of chivalry\tsk and not one thing had offered to signalise his zeal
for my service from the time that he had entered into it, which was
almost four-and-twenty hours.  The poor soul burn’d with impatience; and
the Count de L\tsk ’s servant coming with the letter, being the first
practicable occasion which offer’d, La Fleur had laid hold of it; and, in
order to do honour to his master, had taken him into a back parlour in
the auberge, and treated him with a cup or two of the best wine in
Picardy; and the Count de L\tsk ’s servant, in return, and not to be
behindhand in politeness with La Fleur, had taken him back with him to
the Count’s hotel.  La Fleur’s \i{prevenancy} (for there was a passport in
his very looks) soon set every servant in the kitchen at ease with him;
and as a Frenchman, whatever be his talents, has no sort of prudery in
shewing them, La Fleur, in less than five minutes, had pulled out his
fife, and leading off the dance himself with the first note, set the
\i{fille de chambre}, the \i{maître d’hôtel}, the cook, the scullion, and all
the house-hold, dogs and cats, besides an old monkey, a dancing: I
suppose there never was a merrier kitchen since the flood.

Madame de L\tsk , in passing from her brother’s apartments to her own,
hearing so much jollity below stairs, rung up her \i{fille de chambre} to
ask about it; and, hearing it was the English gentleman’s servant, who
had set the whole house merry with his pipe, she ordered him up.

As the poor fellow could not present himself empty, he had loaded himself
in going up stairs with a thousand compliments to Madame de L\tsk , on the
part of his master\tsk added a long apocrypha of inquiries after Madame de
L\tsk ’s health\tsk told her, that Monsieur his master was \i{au désespoire} for
her re-establishment from the fatigues of her journey\tsk and, to close all,
that Monsieur had received the letter which Madame had done him the
honour\tsk And he has done me the honour, said Madame de L\tsk , interrupting La
Fleur, to send a billet in return.

Madame de L\tsk  had said this with such a tone of reliance upon the fact,
that La Fleur had not power to disappoint her expectations\tsk he trembled
for my honour\tsk and possibly might not altogether be unconcerned for his
own, as a man capable of being attached to a master who could be wanting
\i{en égards vis à vis d’une femme}! so that when Madame de L\tsk  asked La
Fleur if he had brought a letter\tsk \i{O qu’oui}, said La Fleur: so laying
down his hat upon the ground, and taking hold of the flap of his right
side pocket with his left hand, he began to search for the letter with
his right\tsk then contrariwise.\tsk \i{Diable}! then sought every pocket\tsk pocket
by pocket, round, not forgetting his fob\tsk \i{Peste}!\tsk then La Fleur emptied
them upon the floor\tsk pulled out a dirty cravat,\tsk a handkerchief,\tsk a comb,\tsk a
whip lash\tsk a nightcap,\tsk then gave a peep into his hat,\tsk \i{Quelle
étourderie}!  He had left the letter upon the table in the auberge\tsk he
would run for it, and be back with it in three minutes.

I had just finished my supper when La Fleur came in to give me an account
of his adventure: he told the whole story simply as it was: and only
added that if Monsieur had forgot (\i{par hazard}) to answer Madame’s
letter, the arrangement gave him an opportunity to recover the \i{faux
pas}\tsk and if not, that things were only as they were.

Now I was not altogether sure of my \i{étiquette}, whether I ought to have
wrote or no\tsk but if I had\tsk a devil himself could not have been angry:
’twas but the officious zeal of a well meaning creature for my honour;
and, however he might have mistook the road\tsk or embarrassed me in so
doing\tsk his heart was in no fault,\tsk I was under no necessity to write\tsk and,
what weighed more than all\tsk he did not look as if he had done amiss.

\tsk ’Tis all very well, La Fleur, said I.\tsk ’Twas sufficient.  La Fleur flew
out of the room like lightning, and returned with pen, ink, and paper, in
his hand; and, coming up to the table, laid them close before me, with
such a delight in his countenance, that I could not help taking up the
pen.

I began and began again; and, though I had nothing to say, and that
nothing might have been expressed in half a dozen lines, I made half a
dozen different beginnings, and could no way please myself.

In short, I was in no mood to write.

La Fleur stepp’d out and brought a little water in a glass to dilute my
ink\tsk then fetch’d sand and seal-wax.\tsk It was all one; I wrote, and
blotted, and tore off, and burnt, and wrote again.\tsk \i{Le diable l’emporte}!
said I, half to myself\tsk I cannot write this self-same letter, throwing
the pen down despairingly as I said it.

As soon as I had cast down my pen, La Fleur advanced with the most
respectful carriage up to the table, and making a thousand apologies for
the liberty he was going to take, told me he had a letter in his pocket
wrote by a drummer in his regiment to a corporal’s wife, which he durst
say would suit the occasion.

I had a mind to let the poor fellow have his humour.\tsk Then prithee, said
I, let me see it.

La Fleur instantly pulled out a little dirty pocket book cramm’d full of
small letters and billet-doux in a sad condition, and laying it upon the
table, and then untying the string which held them all together, run them
over, one by one, till he came to the letter in question\tsk \i{La voila}!
said he, clapping his hands: so, unfolding it first, he laid it open
before me, and retired three steps from the table whilst I read it.




\head{24pt}{THE LETTER}


Madame,

Je suis pénétré de la douleur la plus vive, et réduit en même temps au
désespoir par ce retour imprévù du Caporal qui rend notre entrevûe de ce
soir la chose du monde la plus impossible.

Mais vive la joie! et toute la mienne sera de penser à vous.

L’amour n’est \i{rien} sans sentiment.

Et le sentiment est encore \i{moins} sans amour.

On dit qu’on ne doit jamais se désesperér.

On dit aussi que Monsieur le Caporal monte la garde Mercredi: alors ce
cera mon tour.

                           \i{Chacun à son tour}.

En attendant\tsk Vive l’amour! et vive la bagatelle!

                                                          Je suis, Madame,
                                          Avec tous les sentimens les plus
                                          respectueux et les plus tendres,
                                                              tout à vous,
                                                             JAQUES ROQUE.

It was but changing the Corporal into the Count\tsk and saying nothing about
mounting guard on Wednesday\tsk and the letter was neither right nor
wrong\tsk so, to gratify the poor fellow, who stood trembling for my honour,
his own, and the honour of his letter\tsk I took the cream gently off it,
and whipping it up in my own way, I seal’d it up and sent him with it to
Madame de L\tsk \tsk and the next morning we pursued our journey to Paris.




\head{24pt}{PARIS}


\dropcap{W}{HEN} a man can contest the point by dint of equipage, and carry all on
floundering before him with half a dozen of lackies and a couple of
cooks\tsk ’tis very well in such a place as Paris\tsk he may drive in at which
end of a street he will.

A poor prince who is weak in cavalry, and whose whole infantry does not
exceed a single man, had best quit the field, and signalize himself in
the cabinet, if he can get up into it\tsk I say \i{up into it}\tsk for there is no
descending perpendicular amongst ’em with a \lqq \i{Me voici}! \i{mes
enfans}\rqq\tsk here I am\tsk whatever many may think.

I own my first sensations, as soon as I was left solitary and alone in my
own chamber in the hotel, were far from being so flattering as I had
prefigured them.  I walked up gravely to the window in my dusty black
coat, and looking through the glass saw all the world in yellow, blue,
and green, running at the ring of pleasure.\tsk The old with broken lances,
and in helmets which had lost their vizards\tsk the young in armour bright
which shone like gold, beplumed with each gay feather of the
east\tsk all,\tsk all, tilting at it like fascinated knights in tournaments of
yore for fame and love.\tsk 

Alas, poor Yorick! cried I, what art thou doing here?  On the very first
onset of all this glittering clatter thou art reduced to an
atom\tsk seek\tsk seek some winding alley, with a tourniquet at the end of it,
where chariot never rolled or flambeau shot its rays\tsk there thou mayest
solace thy soul in converse sweet with some kind grisette of a barber’s
wife, and get into such coteries!\tsk 

\tsk May I perish! if I do, said I, pulling out the letter which I had to
present to Madame de R\tsk .\tsk I’ll wait upon this lady, the very first thing I
do.  So I called La Fleur to go seek me a barber directly\tsk and come back
and brush my coat.




\head{24pt}{THE WIG}
\head{24pt}{PARIS}


\dropcap{W}{HEN} the barber came, he absolutely refused to have any thing to do with
my wig: ’twas either above or below his art: I had nothing to do but to
take one ready made of his own recommendation.

\tsk But I fear, friend! said I, this buckle won’t stand.\tsk You may emerge it,
replied he, into the ocean, and it will stand.\tsk 

What a great scale is every thing upon in this city thought I.\tsk The utmost
stretch of an English periwig-maker’s ideas could have gone no further
than to have \lqq dipped it into a pail of water.\rqq\tsk What difference! ’tis like
Time to Eternity!

I confess I do hate all cold conceptions, as I do the puny ideas which
engender them; and am generally so struck with the great works of nature,
that for my own part, if I could help it, I never would make a comparison
less than a mountain at least.  All that can be said against the French
sublime, in this instance of it, is this\tsk That the grandeur is \i{more} in
the \i{word}, and \i{less} in the \i{thing}.  No doubt, the ocean fills the
mind with vast ideas; but Paris being so far inland, it was not likely I
should run post a hundred miles out of it, to try the experiment\tsk the
Parisian barber meant nothing.\tsk 

The pail of water standing beside the great deep, makes, certainly, but a
sorry figure in speech\tsk but, ’twill be said\tsk it has one advantage\tsk ’tis in
the next room, and the truth of the buckle may be tried in it, without
more ado, in a single moment.

In honest truth, and upon a more candid revision of the matter, \i{The
French expression professes more than it performs}.

I think I can see the precise and distinguishing marks of national
characters more in these nonsensical \i{minutiæ} than in the most important
matters of state; where great men of all nations talk and stalk so much
alike, that I would not give ninepence to choose amongst them.

I was so long in getting from under my barber’s hands, that it was too
late to think of going with my letter to Madame R\tsk  that night: but when a
man is once dressed at all points for going out, his reflections turn to
little account; so taking down the name of the Hôtel de Modene, where I
lodged, I walked forth without any determination where to go\tsk I shall
consider of that, said I, as I walk along.




\head{24pt}{THE PULSE}
\head{24pt}{PARIS}


HAIL, ye small sweet courtesies of life, for smooth do ye make the road
of it! like grace and beauty, which beget inclinations to love at first
sight: ’tis ye who open this door and let the stranger in.

\tsk Pray, Madame, said I, have the goodness to tell me which way I must turn
to go to the Opéra Comique?\tsk Most willingly, Monsieur, said she, laying
aside her work.\tsk 

I had given a cast with my eye into half a dozen shops, as I came along,
in search of a face not likely to be disordered by such an interruption:
till at last, this, hitting my fancy, I had walked in.

She was working a pair of ruffles, as she sat in a low chair, on the far
side of the shop, facing the door.

\tsk \i{Très volontiers}, most willingly, said she, laying her work down upon a
chair next her, and rising up from the low chair she was sitting in, with
so chearful a movement, and so cheerful a look, that had I been laying
out fifty louis d’ors with her, I should have said\tsk \lqq This woman is
grateful.\rqq

You must turn, Monsieur, said she, going with me to the door of the shop,
and pointing the way down the street I was to take\tsk you must turn first
to your left hand\tsk \i{mais prenez garde}\tsk there are two turns; and be so
good as to take the second\tsk then go down a little way and you’ll see a
church: and, when you are past it, give yourself the trouble to turn
directly to the right, and that will lead you to the foot of the Pont
Neuf, which you must cross\tsk and there any one will do himself the pleasure
to shew you.\tsk 

She repeated her instructions three times over to me, with the same
goodnatur’d patience the third time as the first\tsk and if \i{tones and
manners} have a meaning, which certainly they have, unless to hearts
which shut them out\tsk she seemed really interested that I should not lose
myself.

I will not suppose it was the woman’s beauty, notwithstanding she was the
handsomest grisette, I think, I ever saw, which had much to do with the
sense I had of her courtesy; only I remember, when I told her how much I
was obliged to her, that I looked very full in her eyes\tsk and that I
repeated my thanks as often as she had done her instructions.

I had not got ten paces from the door, before I found I had forgot every
tittle of what she had said\tsk so looking back, and seeing her still
standing in the door of the shop, as if to look whether I went right or
not\tsk I returned back to ask her, whether the first turn was to my right
or left\tsk for that I had absolutely forgot.\tsk Is it possible! said she, half
laughing.  ’Tis very possible, replied I, when a man is thinking more of
a woman than of her good advice.

As this was the real truth\tsk she took it, as every woman takes a matter of
right, with a slight curtsey.

\tsk \i{Attendez}! said she, laying her hand upon my arm to detain me, whilst
she called a lad out of the back shop to get ready a parcel of gloves.  I
am just going to send him, said she, with a packet into that quarter, and
if you will have the complaisance to step in, it will be ready in a
moment, and he shall attend you to the place.\tsk So I walk’d in with her to
the far side of the shop: and taking up the ruffle in my hand which she
laid upon the chair, as if I had a mind to sit, she sat down herself in
her low chair, and I instantly sat myself down beside her.

\tsk He will be ready, Monsieur, said she, in a moment.\tsk And in that moment,
replied I, most willingly would I say something very civil to you for all
these courtesies.  Any one may do a casual act of good nature, but a
continuation of them shews it is a part of the temperature; and
certainly, added I, if it is the same blood which comes from the heart
which descends to the extremes (touching her wrist) I am sure you must
have one of the best pulses of any woman in the world.\tsk Feel it, said she,
holding out her arm.  So laying down my hat, I took hold of her fingers
in one hand, and applied the two forefingers of my other to the artery.\tsk 

\tsk Would to heaven! my dear Eugenius, thou hadst passed by, and beheld me
sitting in my black coat, and in my lack-a-day-sical manner, counting the
throbs of it, one by one, with as much true devotion as if I had been
watching the critical ebb or flow of her fever.\tsk How wouldst thou have
laugh’d and moralized upon my new profession!\tsk and thou shouldst have
laugh’d and moralized on.\tsk Trust me, my dear Eugenius, I should have said,
\lqq There are worse occupations in this world \i{than feeling a woman’s
pulse}.\rqq\tsk But a grisette’s! thou wouldst have said\tsk and in an open shop!
Yorick\tsk 

\tsk So much the better: for when my views are direct, Eugenius, I care not
if all the world saw me feel it.




\head{24pt}{THE HUSBAND}
\head{24pt}{PARIS}


\initial{I}{HAD} counted twenty pulsations, and was going on fast towards the
fortieth, when her husband, coming unexpected from a back parlour into
the shop, put me a little out of my reckoning.\tsk ’Twas nobody but her
husband, she said\tsk so I began a fresh score.\tsk Monsieur is so good, quoth
she, as he pass’d by us, as to give himself the trouble of feeling my
pulse.\tsk The husband took off his hat, and making me a bow, said, I did him
too much honour\tsk and having said that, he put on his hat and walk’d out.

Good God! said I to myself, as he went out\tsk and can this man be the
husband of this woman!

Let it not torment the few who know what must have been the grounds of
this exclamation, if I explain it to those who do not.

In London a shopkeeper and a shopkeeper’s wife seem to be one bone and
one flesh: in the several endowments of mind and body, sometimes the one,
sometimes the other has it, so as, in general, to be upon a par, and
totally with each other as nearly as man and wife need to do.

In Paris, there are scarce two orders of beings more different: for the
legislative and executive powers of the shop not resting in the husband,
he seldom comes there\tsk in some dark and dismal room behind, he sits
commerce-less, in his thrum nightcap, the same rough son of Nature that
Nature left him.

The genius of a people, where nothing but the monarchy is \i{salique},
having ceded this department, with sundry others, totally to the
women\tsk by a continual higgling with customers of all ranks and sizes from
morning to night, like so many rough pebbles shook long together in a
bag, by amicable collisions they have worn down their asperities and
sharp angles, and not only become round and smooth, but will receive,
some of them, a polish like a brilliant\tsk Monsieur \i{le Mari} is little
better than the stone under your foot.

\tsk Surely\tsk surely, man! it is not good for thee to sit alone\tsk thou wast
made for social intercourse and gentle greetings; and this improvement of
our natures from it I appeal to as my evidence.

\tsk And how does it beat, Monsieur? said she.\tsk With all the benignity, said
I, looking quietly in her eyes, that I expected.\tsk She was going to say
something civil in return\tsk but the lad came into the shop with the
gloves.\tsk \i{Apropos}, said I, I want a couple of pairs myself.




\head{24pt}{THE GLOVES}
\head{24pt}{PARIS}


\dropcap{T}{HE} beautiful grisette rose up when I said this, and going behind the
counter, reach’d down a parcel and untied it: I advanced to the side over
against her: they were all too large.  The beautiful grisette measured
them one by one across my hand.\tsk It would not alter their dimensions.\tsk She
begg’d I would try a single pair, which seemed to be the least.\tsk She held
it open\tsk my hand slipped into it at once.\tsk It will not do, said I, shaking
my head a little.\tsk No, said she, doing the same thing.

There are certain combined looks of simple subtlety\tsk where whim, and
sense, and seriousness, and nonsense, are so blended, that all the
languages of Babel set loose together, could not express them\tsk they are
communicated and caught so instantaneously, that you can scarce say which
party is the infector.  I leave it to your men of words to swell pages
about it\tsk it is enough in the present to say again, the gloves would not
do; so, folding our hands within our arms, we both lolled upon the
counter\tsk it was narrow, and there was just room for the parcel to lay
between us.

The beautiful grisette looked sometimes at the gloves, then sideways to
the window, then at the gloves\tsk and then at me.  I was not disposed to
break silence\tsk I followed her example: so, I looked at the gloves, then
to the window, then at the gloves, and then at her\tsk and so on
alternately.

I found I lost considerably in every attack\tsk she had a quick black eye,
and shot through two such long and silken eyelashes with such
penetration, that she look’d into my very heart and reins.\tsk It may seem
strange, but I could actually feel she did.\tsk 

It is no matter, said I, taking up a couple of the pairs next me, and
putting them into my pocket.

I was sensible the beautiful grisette had not asked above a single livre
above the price.\tsk I wish’d she had asked a livre more, and was puzzling my
brains how to bring the matter about.\tsk Do you think, my dear Sir, said
she, mistaking my embarrassment, that I could ask a sous too much of a
stranger\tsk and of a stranger whose politeness, more than his want of
gloves, has done me the honour to lay himself at my mercy?\tsk \i{M’en croyez
capable}?\tsk Faith! not I, said I; and if you were, you are welcome.  So
counting the money into her hand, and with a lower bow than one generally
makes to a shopkeeper’s wife, I went out, and her lad with his parcel
followed me.




\head{24pt}{THE TRANSLATION}
\head{24pt}{PARIS}


\dropcap{T}{HERE} was nobody in the box I was let into but a kindly old French
officer.  I love the character, not only because I honour the man whose
manners are softened by a profession which makes bad men worse; but that
I once knew one\tsk for he is no more,\tsk and why should I not rescue one page
from violation by writing his name in it, and telling the world it was
Captain Tobias Shandy, the dearest of my flock and friends, whose
philanthropy I never think of at this long distance from his death\tsk but my
eyes gush out with tears.  For his sake I have a predilection for the
whole corps of veterans; and so I strode over the two back rows of
benches and placed myself beside him.

The old officer was reading attentively a small pamphlet, it might be the
book of the opera, with a large pair of spectacles.  As soon as I sat
down, he took his spectacles off, and putting them into a shagreen case,
return’d them and the book into his pocket together.  I half rose up, and
made him a bow.

Translate this into any civilized language in the world\tsk the sense is
this:

\lqq Here’s a poor stranger come into the box\tsk he seems as if he knew nobody;
and is never likely, was he to be seven years in Paris, if every man he
comes near keeps his spectacles upon his nose\tsk ’tis shutting the door of
conversation absolutely in his face\tsk and using him worse than a German.\rqq

The French officer might as well have said it all aloud: and if he had, I
should in course have put the bow I made him into French too, and told
him, \lqq I was sensible of his attention, and return’d him a thousand thanks
for it.\rqq

There is not a secret so aiding to the progress of sociality, as to get
master of this \i{short hand}, and to be quick in rendering the several
turns of looks and limbs with all their inflections and delineations,
into plain words.  For my own part, by long habitude, I do it so
mechanically, that, when I walk the streets of London, I go translating
all the way; and have more than once stood behind in the circle, where
not three words have been said, and have brought off twenty different
dialogues with me, which I could have fairly wrote down and sworn to.

I was going one evening to Martini’s concert at Milan, and, was just
entering the door of the hall, when the Marquisina di F\tsk  was coming out
in a sort of a hurry\tsk she was almost upon me before I saw her; so I gave
a spring to once side to let her pass.\tsk She had done the same, and on the
same side too; so we ran our heads together: she instantly got to the
other side to get out: I was just as unfortunate as she had been, for I
had sprung to that side, and opposed her passage again.\tsk We both flew
together to the other side, and then back\tsk and so on\tsk it was ridiculous:
we both blush’d intolerably: so I did at last the thing I should have
done at first\tsk I stood stock-still, and the Marquisina had no more
difficulty.  I had no power to go into the room, till I had made her so
much reparation as to wait and follow her with my eye to the end of the
passage.  She look’d back twice, and walk’d along it rather sideways, as
if she would make room for any one coming up stairs to pass her.\tsk No, said
I\tsk that’s a vile translation: the Marquisina has a right to the best
apology I can make her, and that opening is left for me to do it in\tsk so I
ran and begg’d pardon for the embarrassment I had given her, saying it
was my intention to have made her way.  She answered, she was guided by
the same intention towards me\tsk so we reciprocally thank’d each other.
She was at the top of the stairs; and seeing no \i{cicisbeo} near her, I
begg’d to hand her to her coach\tsk so we went down the stairs, stopping at
every third step to talk of the concert and the adventure.\tsk Upon my word,
Madame, said I, when I had handed her in, I made six different efforts to
let you go out.\tsk And I made six efforts, replied she, to let you enter.\tsk I
wish to heaven you would make a seventh, said I.\tsk With all my heart, said
she, making room.\tsk Life is too short to be long about the forms of it\tsk so
I instantly stepp’d in, and she carried me home with her.\tsk And what became
of the concert, St. Cecilia, who I suppose was at it, knows more than I.

I will only add, that the connexion which arose out of the translation
gave me more pleasure than any one I had the honour to make in Italy.




\head{24pt}{THE DWARF}
\head{24pt}{PARIS}


\initial{I}{HAD} never heard the remark made by any one in my life, except by one;
and who that was will probably come out in this chapter; so that being
pretty much unprepossessed, there must have been grounds for what struck
me the moment I cast my eyes over the parterre\tsk and that was, the
unaccountable sport of Nature in forming such numbers of dwarfs.\tsk No doubt
she sports at certain times in almost every corner of the world; but in
Paris there is no end to her amusements.\tsk The goddess seems almost as
merry as she is wise.

As I carried my idea out of the \i{Opéra Comique} with me, I measured every
body I saw walking in the streets by it.\tsk Melancholy application!
especially where the size was extremely little\tsk the face extremely
dark\tsk the eyes quick,\tsk the nose long,\tsk the teeth white,\tsk the jaw
prominent\tsk to see so many miserables, by force of accidents driven out of
their own proper class into the very verge of another, which it gives me
pain to write down\tsk every third man a pigmy!\tsk some by rickety heads and
hump backs\tsk others by bandy legs\tsk a third set arrested by the hand of
Nature in the sixth and seventh years of their growth\tsk a fourth, in their
perfect and natural state like dwarf apple trees; from the first
rudiments and stamina of their existence, never meant to grow higher.

A Medical Traveller might say, ’tis owing to undue bandages\tsk a Splenetic
one, to want of air\tsk and an Inquisitive Traveller, to fortify the system,
may measure the height of their houses\tsk the narrowness of their streets,
and in how few feet square in the sixth and seventh stories such numbers
of the bourgeoisie eat and sleep together; but I remember Mr. Shandy the
elder, who accounted for nothing like any body else, in speaking one
evening of these matters, averred that children, like other animals,
might be increased almost to any size, provided they came right into the
world; but the misery was, the citizens of were Paris so coop’d up, that
they had not actually room enough to get them.\tsk I do not call it getting
anything, said he\tsk ’tis getting nothing.\tsk Nay, continued he, rising in his
argument, ’tis getting worse than nothing, when all you have got after
twenty or five and twenty years of the tenderest care and most nutritious
aliment bestowed upon it, shall not at last be as high as my leg.  Now,
Mr. Shandy being very short, there could be nothing more said of it.

As this is not a work of reasoning, I leave the solution as I found it,
and content myself with the truth only of the remark, which is verified
in every lane and by-lane of Paris.  I was walking down that which leads
from the Carousal to the Palais Royal, and observing a little boy in some
distress at the side of the gutter which ran down the middle of it, I
took hold of his hand and help’d him over.  Upon turning up his face to
look at him after, I perceived he was about forty.\tsk Never mind, said I,
some good body will do as much for me when I am ninety.

I feel some little principles within me which incline me to be merciful
towards this poor blighted part of my species, who have neither size nor
strength to get on in the world.\tsk I cannot bear to see one of them trod
upon; and had scarce got seated beside my old French officer, ere the
disgust was exercised, by seeing the very thing happen under the box we
sat in.

At the end of the orchestra, and betwixt that and the first side box,
there is a small esplanade left, where, when the house is full, numbers
of all ranks take sanctuary.  Though you stand, as in the parterre, you
pay the same price as in the orchestra.  A poor defenceless being of this
order had got thrust somehow or other into this luckless place\tsk the night
was hot, and he was surrounded by beings two feet and a half higher than
himself.  The dwarf suffered inexpressibly on all sides; but the thing
which incommoded him most, was a tall corpulent German, near seven feet
high, who stood directly betwixt him and all possibility of his seeing
either the stage or the actors.  The poor dwarf did all he could to get a
peep at what was going forwards, by seeking for some little opening
betwixt the German’s arm and his body, trying first on one side, then the
other; but the German stood square in the most unaccommodating posture
that can be imagined\tsk the dwarf might as well have been placed at the
bottom of the deepest draw-well in Paris; so he civilly reached up his
hand to the German’s sleeve, and told him his distress.\tsk The German turn’d
his head back, looked down upon him as Goliah did upon David\tsk and
unfeelingly resumed his posture.

I was just then taking a pinch of snuff out of my monk’s little horn
box.\tsk And how would thy meek and courteous spirit, my dear monk! so
temper’d to \i{bear and forbear}!\tsk how sweetly would it have lent an ear to
this poor soul’s complaint!

The old French officer, seeing me lift up my eyes with an emotion, as I
made the apostrophe, took the liberty to ask me what was the matter?\tsk I
told him the story in three words; and added, how inhuman it was.

By this time the dwarf was driven to extremes, and in his first
transports, which are generally unreasonable, had told the German he
would cut off his long queue with his knife.\tsk The German look’d back
coolly, and told him he was welcome, if he could reach it.

An injury sharpen’d by an insult, be it to whom it will, makes every man
of sentiment a party: I could have leap’d out of the box to have
redressed it.\tsk The old French officer did it with much less confusion; for
leaning a little over, and nodding to a sentinel, and pointing at the
same time with his finger at the distress\tsk the sentinel made his way to
it.\tsk There was no occasion to tell the grievance\tsk the thing told himself;
so thrusting back the German instantly with his musket\tsk he took the poor
dwarf by the hand, and placed him before him.\tsk This is noble! said I,
clapping my hands together.\tsk And yet you would not permit this, said the
old officer, in England.

\tsk In England, dear Sir, said I, \i{we sit all at our ease}.

The old French officer would have set me at unity with myself, in case I
had been at variance\tsk by saying it was a \i{bon mot}\tsk and, as a \i{bon mot}
is always worth something at Paris, he offered me a pinch of snuff.




\head{24pt}{THE ROSE}
\head{24pt}{PARIS}


\dropcap{I}{T} was now my turn to ask the old French officer \lqq What was the matter?\rqq
for a cry of \lqq \i{Haussez les mains}, \i{Monsieur l’Abbé}!\rqq re-echoed from a
dozen different parts of the parterre, was as unintelligible to me, as my
apostrophe to the monk had been to him.

He told me it was some poor Abbé in one of the upper loges, who, he
supposed, had got planted perdu behind a couple of grisettes in order to
see the opera, and that the parterre espying him, were insisting upon his
holding up both his hands during the representation.\tsk And can it be
supposed, said I, that an ecclesiastic would pick the grisettes’ pockets?
The old French officer smiled, and whispering in my ear, opened a door of
knowledge which I had no idea of.

Good God! said I, turning pale with astonishment\tsk is it possible, that a
people so smit with sentiment should at the same time be so unclean, and
so unlike themselves\tsk \i{Quelle grossièrté}! added I.

The French officer told me, it was an illiberal sarcasm at the church,
which had begun in the theatre about the time the Tartuffe was given in
it by Molière: but like other remains of Gothic manners, was
declining.\tsk Every nation, continued he, have their refinements and
\i{grossièrtés}, in which they take the lead, and lose it of one another by
turns\tsk that he had been in most countries, but never in one where he
found not some delicacies, which others seemed to want.  \i{Le} POUR \i{et
le} CONTRE \i{se trouvent en chaque nation}; there is a balance, said he,
of good and bad everywhere; and nothing but the knowing it is so, can
emancipate one half of the world from the prepossession which it holds
against the other\tsk that the advantage of travel, as it regarded the
\i{sçavoir vivre}, was by seeing a great deal both of men and manners; it
taught us mutual toleration; and mutual toleration, concluded he, making
me a bow, taught us mutual love.

The old French officer delivered this with an air of such candour and
good sense, as coincided with my first favourable impressions of his
character\tsk I thought I loved the man; but I fear I mistook the
object\tsk ’twas my own way of thinking\tsk the difference was, I could not have
expressed it half so well.

It is alike troublesome to both the rider and his beast\tsk if the latter
goes pricking up his ears, and starting all the way at every object which
he never saw before.\tsk I have as little torment of this kind as any
creature alive; and yet I honestly confess, that many a thing gave me
pain, and that I blush’d at many a word the first month\tsk which I found
inconsequent and perfectly innocent the second.

Madame do Rambouliet, after an acquaintance of about six weeks with her,
had done me the honour to take me in her coach about two leagues out of
town.\tsk Of all women, Madame de Rambouliet is the most correct; and I never
wish to see one of more virtues and purity of heart.\tsk In our return back,
Madame de Rambouliet desired me to pull the cord.\tsk I asked her if she
wanted anything\tsk \i{Rien que pisser}, said Madame de Rambouliet.

Grieve not, gentle traveller, to let Madame de Rambouliet p\tsk ss on.\tsk And,
ye fair mystic nymphs! go each one \i{pluck your rose}, and scatter them in
your path\tsk for Madame de Rambouliet did no more.\tsk I handed Madame de
Rambouliet out of the coach; and had I been the priest of the chaste
Castalia, I could not have served at her fountain with a more respectful
decorum.






\head{24pt}{FOOTNOTES}




{588}  Post-horse.

{648}  Nosegay.

{649}  Hackney coach.

{652}  Plate, napkin, knife, fork and spoon.





\end{document}
