\documentclass[twoside]{article}
\usepackage[paperheight=395pt, paperwidth=240pt, width=166pt, height=308pt, top=42pt, headsep=2pt]{geometry}
\usepackage[noinfo, frame, a6, center, lualatex]{crop}
\usepackage{graphicx}
\usepackage{shandean}
%
%
\title{A Sentimental Journey through France and Italy}
\author{Laurence Sterne}
\date{1768}
\begin{document}
\def\vol{II}




                                    A
                           SENTIMENTAL JOURNEY
                                 THROUGH
                            FRANCE AND ITALY;


                              BY MR. YORICK.

                     [THE REV. LAURENCE STERNE, M.A.]

                        [FIRST PUBLISHED IN 1768.]





THE FILLE DE CHAMBRE.
PARIS.


WHAT the old French officer had delivered upon travelling, bringing
Polonius’s advice to his son upon the same subject into my head,—and that
bringing in Hamlet, and Hamlet the rest of Shakespeare’s works, I stopp’d
at the Quai de Conti in my return home, to purchase the whole set.

The bookseller said he had not a set in the world.  _Comment_! said I,
taking one up out of a set which lay upon the counter betwixt us.—He said
they were sent him only to be got bound, and were to be sent back to
Versailles in the morning to the Count de B—.

—And does the Count de B—, said I, read Shakespeare?  _C’est un esprit
fort_, replied the bookseller.—He loves English books! and what is more
to his honour, Monsieur, he loves the English too.  You speak this so
civilly, said I, that it is enough to oblige an Englishman to lay out a
louis d’or or two at your shop.—The bookseller made a bow, and was going
to say something, when a young decent girl about twenty, who by her air
and dress seemed to be _fille de chambre_ to some devout woman of
fashion, come into the shop and asked for _Les Égarements du Cœur et de
l’Esprit_: the bookseller gave her the book directly; she pulled out a
little green satin purse run round with a riband of the same colour, and
putting her finger and thumb into it, she took out the money and paid for
it.  As I had nothing more to stay me in the shop, we both walk’d out at
the door together.

—And what have you to do, my dear, said I, with _The Wanderings of the
Heart_, who scarce know yet you have one? nor, till love has first told
you it, or some faithless shepherd has made it ache, canst thou ever be
sure it is so.—_Le Dieu m’en garde_! said the girl.—With reason, said I,
for if it is a good one, ’tis pity it should be stolen; ’tis a little
treasure to thee, and gives a better air to your face, than if it was
dress’d out with pearls.

The young girl listened with a submissive attention, holding her satin
purse by its riband in her hand all the time.—’Tis a very small one, said
I, taking hold of the bottom of it—she held it towards me—and there is
very little in it, my dear, said I; but be but as good as thou art
handsome, and heaven will fill it.  I had a parcel of crowns in my hand
to pay for Shakespeare; and, as she had let go the purse entirely, I put
a single one in; and, tying up the riband in a bow-knot, returned it to
her.

The young girl made me more a humble courtesy than a low one:—’twas one
of those quiet, thankful sinkings, where the spirit bows itself down,—the
body does no more than tell it.  I never gave a girl a crown in my life
which gave me half the pleasure.

My advice, my dear, would not have been worth a pin to you, said I, if I
had not given this along with it: but now, when you see the crown, you’ll
remember it;—so don’t, my dear, lay it out in ribands.

Upon my word, Sir, said the girl, earnestly, I am incapable;—in saying
which, as is usual in little bargains of honour, she gave me her
hand:—_En vérité_, _Monsieur_, _je mettrai cet argent àpart_, said she.

When a virtuous convention is made betwixt man and woman, it sanctifies
their most private walks: so, notwithstanding it was dusky, yet as both
our roads lay the same way, we made no scruple of walking along the Quai
de Conti together.

She made me a second courtesy in setting off, and before we got twenty
yards from the door, as if she had not done enough before, she made a
sort of a little stop to tell me again—she thank’d me.

It was a small tribute, I told her, which I could not avoid paying to
virtue, and would not be mistaken in the person I had been rendering it
to for the world;—but I see innocence, my dear, in your face,—and foul
befall the man who ever lays a snare in its way!

The girl seem’d affected some way or other with what I said;—she gave a
low sigh:—I found I was not empowered to enquire at all after it,—so said
nothing more till I got to the corner of the Rue de Nevers, where, we
were to part.

—But is this the way, my dear, said I, to the Hotel de Modene?  She told
me it was;—or that I might go by the Rue de Gueneguault, which was the
next turn.—Then I’ll go, my dear, by the Rue de Gueneguault, said I, for
two reasons; first, I shall please myself, and next, I shall give you the
protection of my company as far on your way as I can.  The girl was
sensible I was civil—and said, she wished the Hotel de Modene was in the
Rue de St. Pierre.—You live there? said I.—She told me she was _fille de
chambre_ to Madame R—.—Good God! said I, ’tis the very lady for whom I
have brought a letter from Amiens.—The girl told me that Madame R—, she
believed, expected a stranger with a letter, and was impatient to see
him:—so I desired the girl to present my compliments to Madame R—, and
say, I would certainly wait upon her in the morning.

We stood still at the corner of the Rue de Nevers whilst this pass’d.—We
then stopped a moment whilst she disposed of her _Égarements du Cœur_,
&c. more commodiously than carrying them in her hand—they were two
volumes: so I held the second for her whilst she put the first into her
pocket; and then she held her pocket, and I put in the other after it.

’Tis sweet to feel by what fine spun threads our affections are drawn
together.

We set off afresh, and as she took her third step, the girl put her hand
within my arm.—I was just bidding her,—but she did it of herself, with
that undeliberating simplicity, which show’d it was out of her head that
she had never seen me before.  For my own part, I felt the conviction of
consanguinity so strongly, that I could not help turning half round to
look in her face, and see if I could trace out any thing in it of a
family likeness.—Tut! said I, are we not all relations?

When we arrived at the turning up of the Rue de Gueneguault, I stopp’d to
bid her adieu for good and all: the girl would thank me again for my
company and kindness.—She bid me adieu twice.—I repeated it as often; and
so cordial was the parting between us, that had it happened any where
else, I’m not sure but I should have signed it with a kiss of charity, as
warm and holy as an apostle.

But in Paris, as none kiss each other but the men,—I did, what amounted
to the same thing—

—I bid God bless her.




THE PASSPORT.
PARIS.


WHEN I got home to my hotel, La Fleur told me I had been enquired after
by the Lieutenant de Police.—The deuce take it! said I,—I know the
reason.  It is time the reader should know it, for in the order of things
in which it happened, it was omitted: not that it was out of my head; but
that had I told it then it might have been forgotten now;—and now is the
time I want it.

I had left London with so much precipitation, that it never enter’d my
mind that we were at war with France; and had reached Dover, and looked
through my glass at the hills beyond Boulogne, before the idea presented
itself; and with this in its train, that there was no getting there
without a passport.  Go but to the end of a street, I have a mortal
aversion for returning back no wiser than I set out; and as this was one
of the greatest efforts I had ever made for knowledge, I could less bear
the thoughts of it: so hearing the Count de —— had hired the packet, I
begg’d he would take me in his suite.  The Count had some little
knowledge of me, so made little or no difficulty,—only said, his
inclination to serve me could reach no farther than Calais, as he was to
return by way of Brussels to Paris; however, when I had once pass’d
there, I might get to Paris without interruption; but that in Paris I
must make friends and shift for myself.—Let me get to Paris, Monsieur le
Count, said I,—and I shall do very well.  So I embark’d, and never
thought more of the matter.

When La Fleur told me the Lieutenant de Police had been enquiring after
me,—the thing instantly recurred;—and by the time La Fleur had well told
me, the master of the hotel came into my room to tell me the same thing,
with this addition to it, that my passport had been particularly asked
after: the master of the hotel concluded with saying, He hoped I had
one.—Not I, faith! said I.

The master of the hotel retired three steps from me, as from an infected
person, as I declared this;—and poor La Fleur advanced three steps
towards me, and with that sort of movement which a good soul makes to
succour a distress’d one:—the fellow won my heart by it; and from that
single trait I knew his character as perfectly, and could rely upon it as
firmly, as if he had served me with fidelity for seven years.

_Mon seigneur_! cried the master of the hotel; but recollecting himself
as he made the exclamation, he instantly changed the tone of it.—If
Monsieur, said he, has not a passport (_apparemment_) in all likelihood
he has friends in Paris who can procure him one.—Not that I know of,
quoth I, with an air of indifference.—Then _certes_, replied he, you’ll
be sent to the Bastile or the Chatelet _au moins_.—Poo! said I, the King
of France is a good natur’d soul:—he’ll hurt nobody.—_Cela n’empêche
pas_, said he—you will certainly be sent to the Bastile to-morrow
morning.—But I’ve taken your lodgings for a month, answer’d I, and I’ll
not quit them a day before the time for all the kings of France in the
world.  La Fleur whispered in my ear, That nobody could oppose the king
of France.

_Pardi_! said my host, _ces Messieurs Anglois sont des gens très
extraordinaires_;—and, having both said and sworn it,—he went out.




THE PASSPORT.
THE HOTEL AT PARIS.


I COULD not find in my heart to torture La Fleur’s with a serious look
upon the subject of my embarrassment, which was the reason I had treated
it so cavalierly: and to show him how light it lay upon my mind, I dropt
the subject entirely; and whilst he waited upon me at supper, talk’d to
him with more than usual gaiety about Paris, and of the Opéra Comique.—La
Fleur had been there himself, and had followed me through the streets as
far as the bookseller’s shop; but seeing me come out with the young
_fille de chambre_, and that we walk’d down the Quai de Conti together,
La Fleur deem’d it unnecessary to follow me a step further;—so making his
own reflections upon it, he took a shorter cut,—and got to the hotel in
time to be inform’d of the affair of the police against my arrival.

As soon as the honest creature had taken away, and gone down to sup
himself, I then began to think a little seriously about my situation.—

—And here, I know, Eugenius, thou wilt smile at the remembrance of a
short dialogue which passed betwixt us the moment I was going to set
out:—I must tell it here.

Eugenius, knowing that I was as little subject to be overburden’d with
money as thought, had drawn me aside to interrogate me how much I had
taken care for.  Upon telling him the exact sum, Eugenius shook his head,
and said it would not do; so pull’d out his purse in order to empty it
into mine.—I’ve enough in conscience, Eugenius, said I.—Indeed, Yorick,
you have not, replied Eugenius; I know France and Italy better than
you.—But you don’t consider, Eugenius, said I, refusing his offer, that
before I have been three days in Paris, I shall take care to say or do
something or other for which I shall get clapp’d up into the Bastile, and
that I shall live there a couple of months entirely at the king of
France’s expense.—I beg pardon, said Eugenius drily: really I had forgot
that resource.

Now the event I treated gaily came seriously to my door.

Is it folly, or nonchalance, or philosophy, or pertinacity—or what is it
in me, that, after all, when La Fleur had gone down stairs, and I was
quite alone, I could not bring down my mind to think of it otherwise than
I had then spoken of it to Eugenius?

—And as for the Bastile; the terror is in the word.—Make the most of it
you can, said I to myself, the Bastile is but another word for a
tower;—and a tower is but another word for a house you can’t get out
of.—Mercy on the gouty! for they are in it twice a year.—But with nine
livres a day, and pen and ink, and paper, and patience, albeit a man
can’t get out, he may do very well within,—at least for a month or six
weeks; at the end of which, if he is a harmless fellow, his innocence
appears, and he comes out a better and wiser man than he went in.

I had some occasion (I forget what) to step into the court-yard, as I
settled this account; and remember I walk’d down stairs in no small
triumph with the conceit of my reasoning.—Beshrew the sombre pencil! said
I, vauntingly—for I envy not its powers, which paints the evils of life
with so hard and deadly a colouring.  The mind sits terrified at the
objects she has magnified herself, and blackened: reduce them to their
proper size and hue, she overlooks them.—’Tis true, said I, correcting
the proposition,—the Bastile is not an evil to be despised;—but strip it
of its towers—fill up the fosse,—unbarricade the doors—call it simply a
confinement, and suppose ’tis some tyrant of a distemper—and not of a
man, which holds you in it,—the evil vanishes, and you bear the other
half without complaint.

I was interrupted in the heyday of this soliloquy, with a voice which I
took to be of a child, which complained “it could not get out.”—I look’d
up and down the passage, and seeing neither man, woman, nor child, I went
out without farther attention.

In my return back through the passage, I heard the same words repeated
twice over; and, looking up, I saw it was a starling hung in a little
cage.—“I can’t get out,—I can’t get out,” said the starling.

I stood looking at the bird: and to every person who came through the
passage it ran fluttering to the side towards which they approach’d it,
with the same lamentation of its captivity.  “I can’t get out,” said the
starling.—God help thee! said I, but I’ll let thee out, cost what it
will; so I turned about the cage to get to the door: it was twisted and
double twisted so fast with wire, there was no getting it open without
pulling the cage to pieces.—I took both hands to it.

The bird flew to the place where I was attempting his deliverance, and
thrusting his head through the trellis pressed his breast against it as
if impatient.—I fear, poor creature! said I, I cannot set thee at
liberty.—“No,” said the starling,— “I can’t get out—I can’t get out,”
said the starling.

I vow I never had my affections more tenderly awakened; nor do I remember
an incident in my life, where the dissipated spirits, to which my reason
had been a bubble, were so suddenly call’d home.  Mechanical as the notes
were, yet so true in tune to nature were they chanted, that in one moment
they overthrew all my systematic reasonings upon the Bastile; and I
heavily walked upstairs, unsaying every word I had said in going down
them.

Disguise thyself as thou wilt, still, Slavery! said I,—still thou art a
bitter draught! and though thousands in all ages have been made to drink
of thee, thou art no less bitter on that account.—’Tis thou, thrice sweet
and gracious goddess, addressing myself to Liberty, whom all in public or
in private worship, whose taste is grateful, and ever will be so, till
Nature herself shall change.—No _tint_ of words can spot thy snowy
mantle, or chymic power turn thy sceptre into iron:—with thee to smile
upon him as he eats his crust, the swain is happier than his monarch,
from whose court thou art exiled!—Gracious Heaven! cried I, kneeling down
upon the last step but one in my ascent, grant me but health, thou great
Bestower of it, and give me but this fair goddess as my companion,—and
shower down thy mitres, if it seems good unto thy divine providence, upon
those heads which are aching for them!




THE CAPTIVE.
PARIS.


THE bird in his cage pursued me into my room; I sat down close to my
table, and leaning my head upon my hand, I began to figure to myself the
miseries of confinement.  I was in a right frame for it, and so I gave
full scope to my imagination.

I was going to begin with the millions of my fellow-creatures born to no
inheritance but slavery: but finding, however affecting the picture was,
that I could not bring it near me, and that the multitude of sad groups
in it did but distract me.—

—I took a single captive, and having first shut him up in his dungeon, I
then look’d through the twilight of his grated door to take his picture.

I beheld his body half-wasted away with long expectation and confinement,
and felt what kind of sickness of the heart it was which arises from hope
deferr’d.  Upon looking nearer I saw him pale and feverish: in thirty
years the western breeze had not once fann’d his blood;—he had seen no
sun, no moon, in all that time—nor had the voice of friend or kinsman
breathed through his lattice.—His children—

But here my heart began to bleed—and I was forced to go on with another
part of the portrait.

He was sitting upon the ground upon a little straw, in the furthest
corner of his dungeon, which was alternately his chair and bed: a little
calendar of small sticks were laid at the head, notch’d all over with the
dismal days and nights he had passed there;—he had one of these little
sticks in his hand, and, with a rusty nail he was etching another day of
misery to add to the heap.  As I darkened the little light he had, he
lifted up a hopeless eye towards the door, then cast it down,—shook his
head, and went on with his work of affliction.  I heard his chains upon
his legs, as he turned his body to lay his little stick upon the
bundle.—He gave a deep sigh.—I saw the iron enter into his soul!—I burst
into tears.—I could not sustain the picture of confinement which my fancy
had drawn.—I started up from my chair, and calling La Fleur: I bid him
bespeak me a remise, and have it ready at the door of the hotel by nine
in the morning.

I’ll go directly, said I, myself to Monsieur le Duc de Choiseul.

La Fleur would have put me to bed; but—not willing he should see anything
upon my cheek which would cost the honest fellow a heart-ache,—I told him
I would go to bed by myself,—and bid him go do the same.




THE STARLING.
ROAD TO VERSAILLES.


I GOT into my remise the hour I proposed: La Fleur got up behind, and I
bid the coachman make the best of his way to Versailles.

As there was nothing in this road, or rather nothing which I look for in
travelling, I cannot fill up the blank better than with a short history
of this self-same bird, which became the subject of the last chapter.

Whilst the Honourable Mr. — was waiting for a wind at Dover, it had been
caught upon the cliffs, before it could well fly, by an English lad who
was his groom; who, not caring to destroy it, had taken it in his breast
into the packet;—and, by course of feeding it, and taking it once under
his protection, in a day or two grew fond of it, and got it safe along
with him to Paris.

At Paris the lad had laid out a livre in a little cage for the starling,
and as he had little to do better the five months his master staid there,
he taught it, in his mother’s tongue, the four simple words—(and no
more)—to which I own’d myself so much its debtor.

Upon his master’s going on for Italy, the lad had given it to the master
of the hotel.  But his little song for liberty being in an _unknown_
language at Paris, the bird had little or no store set by him: so La
Fleur bought both him and his cage for me for a bottle of Burgundy.

In my return from Italy I brought him with me to the country in whose
language he had learned his notes; and telling the story of him to Lord
A—, Lord A— begg’d the bird of me;—in a week Lord A— gave him to Lord B—;
Lord B— made a present of him to Lord C—; and Lord C—’s gentleman sold
him to Lord D—’s for a shilling; Lord D— gave him to Lord E—; and so
on—half round the alphabet.  From that rank he pass’d into the lower
house, and pass’d the hands of as many commoners.  But as all these
wanted to _get in_, and my bird wanted to _get out_, he had almost as
little store set by him in London as in Paris.

It is impossible but many of my readers must have heard of him; and if
any by mere chance have ever seen him, I beg leave to inform them, that
that bird was my bird, or some vile copy set up to represent him.

[Picture: The starling as the crest of arms] I have nothing farther to
add upon him, but that from that time to this I have borne this poor
starling as the crest to my arms.—Thus:

—And let the herald’s officers twist his neck about if they dare.




THE ADDRESS.
VERSAILLES.


I SHOULD not like to have my enemy take a view of my mind when I am going
to ask protection of any man; for which reason I generally endeavour to
protect myself; but this going to Monsieur le Duc de C— was an act of
compulsion; had it been an act of choice, I should have done it, I
suppose, like other people.

How many mean plans of dirty address, as I went along, did my servile
heart form!  I deserved the Bastile for every one of them.

Then nothing would serve me when I got within sight of Versailles, but
putting words and sentences together, and conceiving attitudes and tones
to wreath myself into Monsieur le Duc de C—’s good graces.—This will do,
said I.—Just as well, retorted I again, as a coat carried up to him by an
adventurous tailor, without taking his measure.  Fool! continued I,—see
Monsieur le Duc’s face first;—observe what character is written in
it;—take notice in what posture he stands to hear you;—mark the turns and
expressions of his body and limbs;—and for the tone,—the first sound
which comes from his lips will give it you; and from all these together
you’ll compound an address at once upon the spot, which cannot disgust
the Duke;—the ingredients are his own, and most likely to go down.

Well! said I, I wish it well over.—Coward again! as if man to man was not
equal throughout the whole surface of the globe; and if in the field—why
not face to face in the cabinet too?  And trust me, Yorick, whenever it
is not so, man is false to himself and betrays his own succours ten times
where nature does it once.  Go to the Duc de C— with the Bastile in thy
looks;—my life for it, thou wilt be sent back to Paris in half an hour
with an escort.

I believe so, said I.—Then I’ll go to the Duke, by heaven! with all the
gaiety and debonairness in the world.—

—And there you are wrong again, replied I.—A heart at ease, Yorick, flies
into no extremes—’tis ever on its centre.—Well! well! cried I, as the
coachman turn’d in at the gates, I find I shall do very well: and by the
time he had wheel’d round the court, and brought me up to the door, I
found myself so much the better for my own lecture, that I neither
ascended the steps like a victim to justice, who was to part with life
upon the top most,—nor did I mount them with a skip and a couple of
strides, as I do when I fly up, Eliza! to thee to meet it.

As I entered the door of the saloon I was met by a person, who possibly
might be the _maître d’hôtel_, but had more the air of one of the under
secretaries, who told me the Duc de C— was busy.—I am utterly ignorant,
said I, of the forms of obtaining an audience, being an absolute
stranger, and what is worse in the present conjuncture of affairs, being
an Englishman too.—He replied, that did not increase the difficulty.—I
made him a slight bow, and told him, I had something of importance to say
to Monsieur le Duc.  The secretary look’d towards the stairs, as if he
was about to leave me to carry up this account to some one.—But I must
not mislead you, said I,—for what I have to say is of no manner of
importance to Monsieur le Duc de C— —but of great importance to
myself.—_C’est une autre affaire_, replied he.—Not at all, said I, to a
man of gallantry.—But pray, good sir, continued I, when can a stranger
hope to have access?—In not less than two hours, said he, looking at his
watch.  The number of equipages in the court-yard seemed to justify the
calculation, that I could have no nearer a prospect;—and as walking
backwards and forwards in the saloon, without a soul to commune with, was
for the time as bad as being in the Bastile itself, I instantly went back
to my remise, and bid the coachman drive me to the _Cordon Bleu_, which
was the nearest hotel.

I think there is a fatality in it;—I seldom go to the place I set out
for.




LE PATISSIER.
VERSAILLES.


BEFORE I had got half way down the street I changed my mind: as I am at
Versailles, thought I, I might as well take a view of the town; so I
pull’d the cord, and ordered the coachman to drive round some of the
principal streets.—I suppose the town is not very large, said I.—The
coachman begg’d pardon for setting me right, and told me it was very
superb, and that numbers of the first dukes and marquises and counts had
hotels.—The Count de B—, of whom the bookseller at the Quai de Conti had
spoke so handsomely the night before, came instantly into my mind.—And
why should I not go, thought I, to the Count de B—, who has so high an
idea of English books and English men—and tell him my story? so I changed
my mind a second time.—In truth it was the third; for I had intended that
day for Madame de R—, in the Rue St. Pierre, and had devoutly sent her
word by her _fille de chambre_ that I would assuredly wait upon her;—but
I am governed by circumstances;—I cannot govern them: so seeing a man
standing with a basket on the other side of the street, as if he had
something to sell, I bid La Fleur go up to him, and enquire for the
Count’s hotel.

La Fleur returned a little pale; and told me it was a Chevalier de St.
Louis selling pâtés.—It is impossible, La Fleur, said I.—La Fleur could
no more account for the phenomenon than myself; but persisted in his
story: he had seen the croix set in gold, with its red riband, he said,
tied to his buttonhole—and had looked into the basket and seen the pâtés
which the Chevalier was selling; so could not be mistaken in that.

Such a reverse in man’s life awakens a better principle than curiosity: I
could not help looking for some time at him as I sat in the remise:—the
more I look’d at him, his croix, and his basket, the stronger they wove
themselves into my brain.—I got out of the remise, and went towards him.

He was begirt with a clean linen apron which fell below his knees, and
with a sort of a bib that went half way up his breast; upon the top of
this, but a little below the hem, hung his croix.  His basket of little
pâtés was covered over with a white damask napkin; another of the same
kind was spread at the bottom; and there was a look of _propreté_ and
neatness throughout, that one might have bought his pâtés of him, as much
from appetite as sentiment.

He made an offer of them to neither; but stood still with them at the
corner of an hotel, for those to buy who chose it without solicitation.

He was about forty-eight;—of a sedate look, something approaching to
gravity.  I did not wonder.—I went up rather to the basket than him, and
having lifted up the napkin, and taking one of his pâtés into my hand,—I
begg’d he would explain the appearance which affected me.

He told me in a few words, that the best part of his life had passed in
the service, in which, after spending a small patrimony, he had obtained
a company and the croix with it; but that, at the conclusion of the last
peace, his regiment being reformed, and the whole corps, with those of
some other regiments, left without any provision, he found himself in a
wide world without friends, without a livre,—and indeed, said he, without
anything but this,—(pointing, as he said it, to his croix).—The poor
Chevalier won my pity, and he finished the scene with winning my esteem
too.

The king, he said, was the most generous of princes, but his generosity
could neither relieve nor reward everyone, and it was only his misfortune
to be amongst the number.  He had a little wife, he said, whom he loved,
who did the _pâtisserie_; and added, he felt no dishonour in defending
her and himself from want in this way—unless Providence had offer’d him a
better.

It would be wicked to withhold a pleasure from the good, in passing over
what happen’d to this poor Chevalier of St. Louis about nine months
after.

It seems he usually took his stand near the iron gates which lead up to
the palace, and as his croix had caught the eyes of numbers, numbers had
made the same enquiry which I had done.—He had told them the same story,
and always with so much modesty and good sense, that it had reach’d at
last the king’s ears;—who, hearing the Chevalier had been a gallant
officer, and respected by the whole regiment as a man of honour and
integrity,—he broke up his little trade by a pension of fifteen hundred
livres a year.

As I have told this to please the reader, I beg he will allow me to
relate another, out of its order, to please myself:—the two stories
reflect light upon each other,—and ’tis a pity they should be parted.




THE SWORD.
RENNES.


WHEN states and empires have their periods of declension, and feel in
their turns what distress and poverty is,—I stop not to tell the causes
which gradually brought the house d’E—, in Brittany, into decay.  The
Marquis d’E— had fought up against his condition with great firmness;
wishing to preserve, and still show to the world, some little fragments
of what his ancestors had been;—their indiscretions had put it out of his
power.  There was enough left for the little exigencies of
_obscurity_.—But he had two boys who looked up to him for _light_;—he
thought they deserved it.  He had tried his sword—it could not open the
way,—the _mounting_ was too expensive,—and simple economy was not a match
for it:—there was no resource but commerce.

In any other province in France, save Brittany, this was smiting the root
for ever of the little tree his pride and affection wish’d to see
re-blossom.—But in Brittany, there being a provision for this, he avail’d
himself of it; and, taking an occasion when the states were assembled at
Rennes, the Marquis, attended with his two boys, entered the court; and
having pleaded the right of an ancient law of the duchy, which, though
seldom claim’d, he said, was no less in force, he took his sword from his
side:—Here, said he, take it; and be trusty guardians of it, till better
times put me in condition to reclaim it.

The president accepted the Marquis’s sword: he staid a few minutes to see
it deposited in the archives of his house—and departed.

The Marquis and his whole family embarked the next day for Martinico,
and in about nineteen or twenty years of successful application to
business, with some unlook’d for bequests from distant branches of his
house, return home to reclaim his nobility, and to support it.

It was an incident of good fortune which will never happen to any
traveller but a Sentimental one, that I should be at Rennes at the very
time of this solemn requisition: I call it solemn;—it was so to me.

The Marquis entered the court with his whole family: he supported his
lady,—his eldest son supported his sister, and his youngest was at the
other extreme of the line next his mother;—he put his handkerchief to his
face twice.—

—There was a dead silence.  When the Marquis had approached within six
paces of the tribunal, he gave the Marchioness to his youngest son, and
advancing three steps before his family,—he reclaim’d his sword.  His
sword was given him, and the moment he got it into his hand he drew it
almost out of the scabbard:—’twas the shining face of a friend he had
once given up—he look’d attentively along it, beginning at the hilt, as
if to see whether it was the same,—when, observing a little rust which it
had contracted near the point, he brought it near his eye, and bending
his head down over it,—I think—I saw a tear fall upon the place.  I could
not be deceived by what followed.

“I shall find,” said he, “some _other way_ to get it off.”

When the Marquis had said this, he returned his sword into its scabbard,
made a bow to the guardians of it,—and, with his wife and daughter, and
his two sons following him, walk’d out.

O, how I envied him his feelings!




THE PASSPORT.
VERSAILLES.


I FOUND no difficulty in getting admittance to Monsieur le Count de B—.
The set of Shakespeares was laid upon the table, and he was tumbling them
over.  I walk’d up close to the table, and giving first such a look at
the books as to make him conceive I knew what they were,—I told him I had
come without any one to present me, knowing I should meet with a friend
in his apartment, who, I trusted, would do it for me:—it is my
countryman, the great Shakespeare, said I, pointing to his works—_et ayez
la bonté_, _mon cher ami_, apostrophizing his spirit, added I, _de me
faire cet honneur-là_.—

The Count smiled at the singularity of the introduction; and seeing I
look’d a little pale and sickly, insisted upon my taking an arm-chair; so
I sat down; and to save him conjectures upon a visit so out of all rule,
I told him simply of the incident in the bookseller’s shop, and how that
had impelled me rather to go to him with the story of a little
embarrassment I was under, than to any other man in France.—And what is
your embarrassment? let me hear it, said the Count.  So I told him the
story just as I have told it the reader.

—And the master of my hotel, said I, as I concluded it, will needs have
it, Monsieur le Count, that I shall be sent to the Bastile;—but I have no
apprehensions, continued I;—for, in falling into the hands of the most
polish’d people in the world, and being conscious I was a true man, and
not come to spy the nakedness of the land, I scarce thought I lay at
their mercy.—It does not suit the gallantry of the French, Monsieur le
Count, said I, to show it against invalids.

An animated blush came into the Count de B—’s cheeks as I spoke this.—_Ne
craignez rien_—Don’t fear, said he.—Indeed, I don’t, replied I
again.—Besides, continued I, a little sportingly, I have come laughing
all the way from London to Paris, and I do not think Monsieur le Duc de
Choiseul is such an enemy to mirth as to send me back crying for my
pains.

—My application to you, Monsieur le Count de B— (making him a low bow),
is to desire he will not.

The Count heard me with great good nature, or I had not said half as
much,—and once or twice said,—_C’est bien dit_.  So I rested my cause
there—and determined to say no more about it.

The Count led the discourse: we talk’d of indifferent things,—of books,
and politics, and men;—and then of women.—God bless them all! said I,
after much discourse about them—there is not a man upon earth who loves
them so much as I do: after all the foibles I have seen, and all the
satires I have read against them, still I love them; being firmly
persuaded that a man, who has not a sort of affection for the whole sex,
is incapable of ever loving a single one as he ought.

_Eh bien_!  _Monsieur l’Anglois_, said the Count, gaily;—you are not come
to spy the nakedness of the land;—I believe you;—_ni encore_, I dare say,
_that_ of our women!—But permit me to conjecture,—if, _par hazard_, they
fell into your way, that the prospect would not affect you.

I have something within me which cannot bear the shock of the least
indecent insinuation: in the sportability of chit-chat I have often
endeavoured to conquer it, and with infinite pain have hazarded a
thousand things to a dozen of the sex together,—the least of which I
could not venture to a single one to gain heaven.

Excuse me, Monsieur le Count, said I;—as for the nakedness of your land,
if I saw it, I should cast my eyes over it with tears in them;—and for
that of your women (blushing at the idea he had excited in me) I am so
evangelical in this, and have such a fellow-feeling for whatever is weak
about them, that I would cover it with a garment if I knew how to throw
it on:—But I could wish, continued I, to spy the nakedness of their
hearts, and through the different disguises of customs, climates, and
religion, find out what is good in them to fashion my own by:—and
therefore am I come.

It is for this reason, Monsieur le Count, continued I, that I have not
seen the Palais Royal,—nor the Luxembourg,—nor the Façade of the
Louvre,—nor have attempted to swell the catalogues we have of pictures,
statues, and churches.—I conceive every fair being as a temple, and would
rather enter in, and see the original drawings and loose sketches hung up
in it, than the Transfiguration of Raphael itself.

The thirst of this, continued I, as impatient as that which inflames the
breast of the connoisseur, has led me from my own home into France,—and
from France will lead me through Italy;—’tis a quiet journey of the heart
in pursuit of Nature, and those affections which arise out of her, which
make us love each other,—and the world, better than we do.

The Count said a great many civil things to me upon the occasion; and
added very politely, how much he stood obliged to Shakespeare for making
me known to him.—But _à propos_, said he;—Shakespeare is full of great
things;—he forgot a small punctilio of announcing your name:—it puts you
under a necessity of doing it yourself.




THE PASSPORT.
VERSAILLES.


THERE is not a more perplexing affair in life to me, than to set about
telling any one who I am,—for there is scarce any body I cannot give a
better account of than myself; and I have often wished I could do it in a
single word,—and have an end of it.  It was the only time and occasion in
my life I could accomplish this to any purpose;—for Shakespeare lying
upon the table, and recollecting I was in his books, I took up Hamlet,
and turning immediately to the grave-diggers’ scene in the fifth act, I
laid my finger upon Yorick, and advancing the book to the Count, with my
finger all the way over the name,—_Me voici_! said I.

Now, whether the idea of poor Yorick’s skull was put out of the Count’s
mind by the reality of my own, or by what magic he could drop a period of
seven or eight hundred years, makes nothing in this account;—’tis certain
the French conceive better than they combine;—I wonder at nothing in this
world, and the less at this; inasmuch as one of the first of our own
Church, for whose candour and paternal sentiments I have the highest
veneration, fell into the same mistake in the very same case:—“He could
not bear,” he said, “to look into the sermons wrote by the King of
Denmark’s jester.”  Good, my Lord said I; but there are two Yoricks.  The
Yorick your Lordship thinks of, has been dead and buried eight hundred
years ago; he flourished in Horwendillus’s court;—the other Yorick is
myself, who have flourished, my Lord, in no court.—He shook his head.
Good God! said I, you might as well confound Alexander the Great with
Alexander the Coppersmith, my lord!—“’Twas all one,” he replied.—

—If Alexander, King of Macedon, could have translated your Lordship, said
I, I’m sure your Lordship would not have said so.

The poor Count de B— fell but into the same _error_.

—_Et_, _Monsieur_, _est-il Yorick_? cried the Count.—_Je le suis_, said
I.—_Vous_?—_Moi_,—_moi qui ai l’honneur de vous parler_, _Monsieur le
Comte_.—_Mon Dieu_! said he, embracing me,—_Vous êtes Yorick_!

The Count instantly put the Shakespeare into his pocket, and left me
alone in his room.




THE PASSPORT.
VERSAILLES.


I COULD not conceive why the Count de B— had gone so abruptly out of the
room, any more than I could conceive why he had put the Shakespeare into
his pocket.—_Mysteries which must explain themselves are not worth the
loss of time which a conjecture about them takes up_: ’twas better to
read Shakespeare; so taking up “_Much Ado About Nothing_,” I transported
myself instantly from the chair I sat in to Messina in Sicily, and got so
busy with Don Pedro, and Benedict, and Beatrice, that I thought not of
Versailles, the Count, or the passport.

Sweet pliability of man’s spirit, that can at once surrender itself to
illusions, which cheat expectation and sorrow of their weary
moments!—Long,—long since had ye number’d out my days, had I not trod so
great a part of them upon this enchanted ground.  When my way is too
rough for my feet, or too steep for my strength, I get off it, to some
smooth velvet path, which Fancy has scattered over with rosebuds of
delights; and having taken a few turns in it, come back strengthened and
refresh’d.—When evils press sore upon me, and there is no retreat from
them in this world, then I take a new course;—I leave it,—and as I have a
clearer idea of the Elysian fields than I have of heaven, I force myself,
like Æneas, into them.—I see him meet the pensive shade of his forsaken
Dido, and wish to recognise it;—I see the injured spirit wave her head,
and turn off silent from the author of her miseries and dishonours;—I
lose the feelings for myself in hers, and in those affections which were
wont to make me mourn for her when I was at school.

_Surely this is not walking in a vain shadow—nor does man disquiet
himself_ in vain _by it_:—he oftener does so in trusting the issue of his
commotions to reason only.—I can safely say for myself, I was never able
to conquer any one single bad sensation in my heart so decisively, as
beating up as fast as I could for some kindly and gentle sensation to
fight it upon its own ground.

When I had got to the end of the third act the Count de B— entered, with
my passport in his hand.  Monsieur le Duc de C—, said the Count, is as
good a prophet, I dare say, as he is a statesman.  _Un homme qui rit_,
said the Duke, _ne sera jamais dangereux_.—Had it been for any one but
the king’s jester, added the Count, I could not have got it these two
hours.—_Pardonnez moi_, Monsieur le Count, said I—I am not the king’s
jester.—But you are Yorick?—Yes.—_Et vous plaisantez_?—I answered, Indeed
I did jest,—but was not paid for it;—’twas entirely at my own expense.

We have no jester at court, Monsieur le Count, said I; the last we had
was in the licentious reign of Charles II.;—since which time our manners
have been so gradually refining, that our court at present is so full of
patriots, who wish for _nothing_ but the honours and wealth of their
country;—and our ladies are all so chaste, so spotless, so good, so
devout,—there is nothing for a jester to make a jest of.—

_Voilà un persiflage_! cried the Count.




THE PASSPORT.
VERSAILLES.


AS the passport was directed to all lieutenant-governors, governors, and
commandants of cities, generals of armies, justiciaries, and all officers
of justice, to let Mr. Yorick the king’s jester, and his baggage, travel
quietly along, I own the triumph of obtaining the passport was not a
little tarnish’d by the figure I cut in it.—But there is nothing unmix’d
in this world; and some of the gravest of our divines have carried it so
far as to affirm, that enjoyment itself was attended even with a
sigh,—and that the greatest _they knew of_ terminated, _in a general
way_, in little better than a convulsion.

I remember the grave and learned Bevoriskius, in his Commentary upon the
Generations from Adam, very naturally breaks off in the middle of a note
to give an account to the world of a couple of sparrows upon the out-edge
of his window, which had incommoded him all the time he wrote, and at
last had entirely taken him off from his genealogy.

—’Tis strange! writes Bevoriskius; but the facts are certain, for I have
had the curiosity to mark them down one by one with my pen;—but the cock
sparrow, during the little time that I could have finished the other half
of this note, has actually interrupted me with the reiteration of his
caresses three-and-twenty times and a half.

How merciful, adds Bevoriskius, is heaven to his creatures!

Ill fated Yorick! that the gravest of thy brethren should be able to
write that to the world, which stains thy face with crimson to copy, even
in thy study.

But this is nothing to my travels.—So I twice,—twice beg pardon for it.




CHARACTER.
VERSAILLES.


AND how do you find the French? said the Count de B—, after he had given
me the passport.

The reader may suppose, that after so obliging a proof of courtesy, I
could not be at a loss to say something handsome to the enquiry.

—_Mais passe_, _pour cela_.—Speak frankly, said he: do you find all the
urbanity in the French which the world give us the honour of?—I had found
every thing, I said, which confirmed it.—_Vraiment_, said the Count, _les
François sont polis_.—To an excess, replied I.

The Count took notice of the word _excès_; and would have it I meant more
than I said.  I defended myself a long time as well as I could against
it.—He insisted I had a reserve, and that I would speak my opinion
frankly.

I believe, Monsieur le Count, said I, that man has a certain compass, as
well as an instrument; and that the social and other calls have occasion
by turns for every key in him; so that if you begin a note too high or
too low, there must be a want either in the upper or under part, to fill
up the system of harmony.—The Count de B— did not understand music, so
desired me to explain it some other way.  A polish’d nation, my dear
Count, said I, makes every one its debtor: and besides, Urbanity itself,
like the fair sex, has so many charms, it goes against the heart to say
it can do ill; and yet, I believe, there is but a certain line of
perfection, that man, take him altogether, is empower’d to arrive at:—if
he gets beyond, he rather exchanges qualities than gets them.  I must not
presume to say how far this has affected the French in the subject we are
speaking of;—but, should it ever be the case of the English, in the
progress of their refinements, to arrive at the same polish which
distinguishes the French, if we did not lose the _politesse du cœur_,
which inclines men more to humane actions than courteous ones,—we should
at least lose that distinct variety and originality of character, which
distinguishes them, not only from each other, but from all the world
besides.

I had a few of King William’s shillings, as smooth as glass, in my
pocket; and foreseeing they would be of use in the illustration of my
hypothesis, I had got them into my hand when I had proceeded so far:—

See, Monsieur le Count, said I, rising up, and laying them before him
upon the table,—by jingling and rubbing one against another for seventy
years together in one body’s pocket or another’s, they are become so much
alike, you can scarce distinguish one shilling from another.

The English, like ancient medals, kept more apart, and passing but few
people’s hands, preserve the first sharpnesses which the fine hand of
Nature has given them;—they are not so pleasant to feel,—but in return
the legend is so visible, that at the first look you see whose image and
superscription they bear.—But the French, Monsieur le Count, added I
(wishing to soften what I had said), have so many excellences, they can
the better spare this;—they are a loyal, a gallant, a generous, an
ingenious, and good temper’d people as is under heaven;—if they have a
fault—they are too _serious_.

_Mon Dieu_! cried the Count, rising out of his chair.

_Mais vous plaisantez_, said he, correcting his exclamation.—I laid my
hand upon my breast, and with earnest gravity assured him it was my most
settled opinion.

The Count said he was mortified he could not stay to hear my reasons,
being engaged to go that moment to dine with the Duc de C—.

But if it is not too far to come to Versailles to eat your soup with me,
I beg, before you leave France, I may have the pleasure of knowing you
retract your opinion,—or, in what manner you support it.—But, if you do
support it, Monsieur Anglois, said he, you must do it with all your
powers, because you have the whole world against you.—I promised the
Count I would do myself the honour of dining with him before I set out
for Italy;—so took my leave.




THE TEMPTATION.
PARIS.


WHEN I alighted at the hotel, the porter told me a young woman with a
bandbox had been that moment enquiring for me.—I do not know, said the
porter, whether she is gone away or not.  I took the key of my chamber of
him, and went upstairs; and when I had got within ten steps of the top of
the landing before my door, I met her coming easily down.

It was the fair _fille de chambre_ I had walked along the Quai de Conti
with; Madame de R— had sent her upon some commission to a _marchande des
modes_ within a step or two of the Hôtel de Modene; and as I had fail’d
in waiting upon her, had bid her enquire if I had left Paris; and if so,
whether I had not left a letter addressed to her.

As the fair _fille de chambre_ was so near my door, she returned back,
and went into the room with me for a moment or two whilst I wrote a card.

It was a fine still evening in the latter end of the month of May,—the
crimson window curtains (which were of the same colour as those of the
bed) were drawn close:—the sun was setting, and reflected through them so
warm a tint into the fair _fille de chambre’s_ face,—I thought she
blush’d;—the idea of it made me blush myself:—we were quite alone; and
that superinduced a second blush before the first could get off.

There is a sort of a pleasing half guilty blush, where the blood is more
in fault than the man:—’tis sent impetuous from the heart, and virtue
flies after it,—not to call it back, but to make the sensation of it more
delicious to the nerves:—’tis associated.—

But I’ll not describe it;—I felt something at first within me which was
not in strict unison with the lesson of virtue I had given her the night
before.—I sought five minutes for a card;—I knew I had not one.—I took up
a pen.—I laid it down again;—my hand trembled:—the devil was in me.

I know as well as any one he is an adversary, whom, if we resist, he will
fly from us;—but I seldom resist him at all; from a terror, though I may
conquer, I may still get a hurt in the combat;—so I give up the triumph
for security; and, instead of thinking to make him fly, I generally fly
myself.

The fair _fille de chambre_ came close up to the bureau where I was
looking for a card—took up first the pen I cast down, then offer’d to
hold me the ink; she offer’d it so sweetly, I was going to accept it;—but
I durst not;—I have nothing, my dear, said I, to write upon.—Write it,
said she, simply, upon anything.—

I was just going to cry out, Then I will write it, fair girl! upon thy
lips.—

If I do, said I, I shall perish;—so I took her by the hand, and led her
to the door, and begg’d she would not forget the lesson I had given
her.—She said, indeed she would not;—and, as she uttered it with some
earnestness, she turn’d about, and gave me both her hands, closed
together, into mine;—it was impossible not to compress them in that
situation;—I wish’d to let them go; and all the time I held them, I kept
arguing within myself against it,—and still I held them on.—In two
minutes I found I had all the battle to fight over again;—and I felt my
legs and every limb about me tremble at the idea.

The foot of the bed was within a yard and a half of the place where we
were standing.—I had still hold of her hands—and how it happened I can
give no account; but I neither ask’d her—nor drew her—nor did I think of
the bed;—but so it did happen, we both sat down.

I’ll just show you, said the fair _fille de chambre_, the little purse I
have been making to-day to hold your crown.  So she put her hand into her
right pocket, which was next me, and felt for it some time—then into the
left.—“She had lost it.”—I never bore expectation more quietly;—it was in
her right pocket at last;—she pull’d it out; it was of green taffeta,
lined with a little bit of white quilted satin, and just big enough to
hold the crown: she put it into my hand;—it was pretty; and I held it ten
minutes with the back of my hand resting upon her lap—looking sometimes
at the purse, sometimes on one side of it.

A stitch or two had broke out in the gathers of my stock; the fair _fille
de chambre_, without saying a word, took out her little housewife,
threaded a small needle, and sew’d it up.—I foresaw it would hazard the
glory of the day; and, as she pass’d her hand in silence across and
across my neck in the manœuvre, I felt the laurels shake which fancy had
wreath’d about my head.

A strap had given way in her walk, and the buckle of her shoe was just
falling off.—See, said the _fille de chambre_, holding up her foot.—I
could not, for my soul but fasten the buckle in return, and putting in
the strap,—and lifting up the other foot with it, when I had done, to see
both were right,—in doing it too suddenly, it unavoidably threw the fair
_fille de chambre_ off her centre,—and then—




THE CONQUEST.


YES,—and then—.  Ye whose clay-cold heads and luke-warm hearts can argue
down or mask your passions, tell me, what trespass is it that man should
have them? or how his spirit stands answerable to the Father of spirits
but for his conduct under them?

If Nature has so wove her web of kindness, that some threads of love and
desire are entangled with the piece,—must the whole web be rent in
drawing them out?—Whip me such stoics, great Governor of Nature! said I
to myself:—wherever thy providence shall place me for the trials of my
virtue;—whatever is my danger,—whatever is my situation,—let me feel the
movements which rise out of it, and which belong to me as a man,—and, if
I govern them as a good one, I will trust the issues to thy justice; for
thou hast made us, and not we ourselves.

As I finished my address, I raised the fair _fille de chambre_ up by the
hand, and led her out of the room:—she stood by me till I locked the door
and put the key in my pocket,—and then,—the victory being quite
decisive—and not till then, I press’d my lips to her cheek, and taking
her by the hand again, led her safe to the gate of the hotel.




THE MYSTERY.
PARIS.


If a man knows the heart, he will know it was impossible to go back
instantly to my chamber;—it was touching a cold key with a flat third to
it upon the close of a piece of music, which had call’d forth my
affections:—therefore, when I let go the hand of the _fille de chambre_,
I remained at the gate of the hotel for some time, looking at every one
who pass’d by,—and forming conjectures upon them, till my attention got
fix’d upon a single object which confounded all kind of reasoning upon
him.

It was a tall figure of a philosophic, serious, adust look, which passed
and repass’d sedately along the street, making a turn of about sixty
paces on each side of the gate of the hotel;—the man was about
fifty-two—had a small cane under his arm—was dress’d in a dark
drab-colour’d coat, waistcoat, and breeches, which seem’d to have seen
some years service:—they were still clean, and there was a little air of
frugal _propreté_ throughout him.  By his pulling off his hat, and his
attitude of accosting a good many in his way, I saw he was asking
charity: so I got a sous or two out of my pocket ready to give him, as he
took me in his turn.—He pass’d by me without asking anything—and yet did
not go five steps further before he ask’d charity of a little woman.—I
was much more likely to have given of the two.—He had scarce done with
the woman, when he pull’d off his hat to another who was coming the same
way.—An ancient gentleman came slowly—and, after him, a young smart
one.—He let them both pass, and ask’d nothing.  I stood observing him
half an hour, in which time he had made a dozen turns backwards and
forwards, and found that he invariably pursued the same plan.

There were two things very singular in this, which set my brain to work,
and to no purpose:—the first was, why the man should _only_ tell his
story to the sex;—and, secondly,—what kind of story it was, and what
species of eloquence it could be, which soften’d the hearts of the women,
which he knew ’twas to no purpose to practise upon the men.

There were two other circumstances, which entangled this mystery;—the one
was, he told every woman what he had to say in her ear, and in a way
which had much more the air of a secret than a petition;—the other was,
it was always successful.—He never stopp’d a woman, but she pull’d out
her purse, and immediately gave him something.

I could form no system to explain the phenomenon.

I had got a riddle to amuse me for the rest of the evening; so I walk’d
upstairs to my chamber.




THE CASE OF CONSCIENCE.
PARIS.


I WAS immediately followed up by the master of the hotel, who came into
my room to tell me I must provide lodgings elsewhere.—How so, friend?
said I.—He answered, I had had a young woman lock’d up with me two hours
that evening in my bedchamber, and ’twas against the rules of his
house.—Very well, said I, we’ll all part friends then,—for the girl is no
worse,—and I am no worse,—and you will be just as I found you.—It was
enough, he said, to overthrow the credit of his hotel.—_Voyez vous_,
Monsieur, said he, pointing to the foot of the bed we had been sitting
upon.—I own it had something of the appearance of an evidence; but my
pride not suffering me to enter into any detail of the case, I exhorted
him to let his soul sleep in peace, as I resolved to let mine do that
night, and that I would discharge what I owed him at breakfast.

I should not have minded, Monsieur, said he, if you had had twenty
girls—’Tis a score more, replied I, interrupting him, than I ever
reckon’d upon—Provided, added he, it had been but in a morning.—And does
the difference of the time of the day at Paris make a difference in the
sin?—It made a difference, he said, in the scandal.—I like a good
distinction in my heart; and cannot say I was intolerably out of temper
with the man.—I own it is necessary, resumed the master of the hotel,
that a stranger at Paris should have the opportunities presented to him
of buying lace and silk stockings and ruffles, _et tout cela_;—and ’tis
nothing if a woman comes with a band-box.—O, my conscience! said I, she
had one but I never look’d into it.—Then Monsieur, said he, has bought
nothing?—Not one earthly thing, replied I.—Because, said he, I could
recommend one to you who would use you _en conscience_.—But I must see
her this night, said I.—He made me a low bow, and walk’d down.

Now shall I triumph over this _maître d’hôtel_, cried I,—and what then?
Then I shall let him see I know he is a dirty fellow.—And what then?
What then?—I was too near myself to say it was for the sake of others.—I
had no good answer left;—there was more of spleen than principle in my
project, and I was sick of it before the execution.

In a few minutes the grisette came in with her box of lace.—I’ll buy
nothing, however, said I, within myself.

The grisette would show me everything.—I was hard to please: she would
not seem to see it; she opened her little magazine, and laid all her
laces one after another before me;—unfolded and folded them up again one
by one with the most patient sweetness.—I might buy,—or not;—she would
let me have everything at my own price:—the poor creature seem’d anxious
to get a penny; and laid herself out to win me, and not so much in a
manner which seem’d artful, as in one I felt simple and caressing.

If there is not a fund of honest gullibility in man, so much the
worse;—my heart relented, and I gave up my second resolution as quietly
as the first.—Why should I chastise one for the trespass of another?  If
thou art tributary to this tyrant of an host, thought I, looking up in
her face, so much harder is thy bread.

If I had not had more than four louis d’ors in my purse, there was no
such thing as rising up and showing her the door, till I had first laid
three of them out in a pair of ruffles.

—The master of the hotel will share the profit with her;—no matter,—then
I have only paid as many a poor soul has _paid_ before me, for an act he
_could_ not do, or think of.




THE RIDDLE.
PARIS.


WHEN La Fleur came up to wait upon me at supper, he told me how sorry the
master of the hotel was for his affront to me in bidding me change my
lodgings.

A man who values a good night’s rest will not lie down with enmity in his
heart, if he can help it.—So I bid La Fleur tell the master of the hotel,
that I was sorry on my side for the occasion I had given him;—and you may
tell him, if you will, La Fleur, added I, that if the young woman should
call again, I shall not see her.

This was a sacrifice not to him, but myself, having resolved, after so
narrow an escape, to run no more risks, but to leave Paris, if it was
possible, with all the virtue I enter’d it.

_C’est déroger à noblesse_, _Monsieur_, said La Fleur, making me a bow
down to the ground as he said it.—_Et encore_, _Monsieur_, said he, may
change his sentiments;—and if (_par hazard_) he should like to amuse
himself,—I find no amusement in it, said I, interrupting him.—

_Mon Dieu_! said La Fleur,—and took away.

In an hour’s time he came to put me to bed, and was more than commonly
officious:—something hung upon his lips to say to me, or ask me, which he
could not get off: I could not conceive what it was, and indeed gave
myself little trouble to find it out, as I had another riddle so much
more interesting upon my mind, which was that of the man’s asking charity
before the door of the hotel.—I would have given anything to have got to
the bottom of it; and that, not out of curiosity,—’tis so low a principle
of enquiry, in general, I would not purchase the gratification of it with
a two-sous piece;—but a secret, I thought, which so soon and so certainly
soften’d the heart of every woman you came near, was a secret at least
equal to the philosopher’s stone; had I both the Indies, I would have
given up one to have been master of it.

I toss’d and turn’d it almost all night long in my brains to no manner of
purpose; and when I awoke in the morning, I found my spirits as much
troubled with my dreams, as ever the King of Babylon had been with his;
and I will not hesitate to affirm, it would have puzzled all the wise men
of Paris as much as those of Chaldea to have given its interpretation.




LE DIMANCHE.
PARIS.


IT was Sunday; and when La Fleur came in, in the morning, with my coffee
and roll and butter, he had got himself so gallantly array’d, I scarce
knew him.

I had covenanted at Montreuil to give him a new hat with a silver button
and loop, and four louis d’ors, _pour s’adoniser_, when we got to Paris;
and the poor fellow, to do him justice, had done wonders with it.

He had bought a bright, clean, good scarlet coat, and a pair of breeches
of the same.—They were not a crown worse, he said, for the wearing.—I
wish’d him hang’d for telling me.—They look’d so fresh, that though I
knew the thing could not be done, yet I would rather have imposed upon my
fancy with thinking I had bought them new for the fellow, than that they
had come out of the Rue de Friperie.

This is a nicety which makes not the heart sore at Paris.

He had purchased, moreover, a handsome blue satin waistcoat, fancifully
enough embroidered:—this was indeed something the worse for the service
it had done, but ’twas clean scour’d;—the gold had been touch’d up, and
upon the whole was rather showy than otherwise;—and as the blue was not
violent, it suited with the coat and breeches very well: he had squeez’d
out of the money, moreover, a new bag and a solitaire; and had insisted
with the _fripier_ upon a gold pair of garters to his breeches knees.—He
had purchased muslin ruffles, _bien brodées_, with four livres of his own
money;—and a pair of white silk stockings for five more;—and to top all,
nature had given him a handsome figure, without costing him a sous.

He entered the room thus set off, with his hair dressed in the first
style, and with a handsome bouquet in his breast.—In a word, there was
that look of festivity in everything about him, which at once put me in
mind it was Sunday;—and, by combining both together, it instantly struck
me, that the favour he wish’d to ask of me the night before, was to spend
the day as every body in Paris spent it besides.  I had scarce made the
conjecture, when La Fleur, with infinite humility, but with a look of
trust, as if I should not refuse him, begg’d I would grant him the day,
_pour faire le galant vis-à-vis de sa maîtresse_.

Now it was the very thing I intended to do myself vis-à-vis Madame de
R—.—I had retained the remise on purpose for it, and it would not have
mortified my vanity to have had a servant so well dress’d as La Fleur
was, to have got up behind it: I never could have worse spared him.

But we must _feel_, not argue in these embarrassments.—The sons and
daughters of Service part with liberty, but not with nature, in their
contracts; they are flesh and blood, and have their little vanities and
wishes in the midst of the house of bondage, as well as their
task-masters;—no doubt, they have set their self-denials at a price,—and
their expectations are so unreasonable, that I would often disappoint
them, but that their condition puts it so much in my power to do it.

_Behold_,—_Behold_, _I am thy servant_—disarms me at once of the powers
of a master.—

Thou shalt go, La Fleur! said I.

—And what mistress, La Fleur, said I, canst thou have picked up in so
little a time at Paris?  La Fleur laid his hand upon his breast, and said
’twas a _petite demoiselle_, at Monsieur le Count de B—’s.—La Fleur had a
heart made for society; and, to speak the truth of him, let as few
occasions slip him as his master;—so that somehow or other,—but
how,—heaven knows,—he had connected himself with the demoiselle upon the
landing of the staircase, during the time I was taken up with my
passport; and as there was time enough for me to win the Count to my
interest, La Fleur had contrived to make it do to win the maid to his.
The family, it seems, was to be at Paris that day, and he had made a
party with her, and two or three more of the Count’s household, upon the
boulevards.

Happy people! that once a week at least are sure to lay down all your
cares together, and dance and sing and sport away the weights of
grievance, which bow down the spirit of other nations to the earth.




THE FRAGMENT.
PARIS.


LA FLEUR had left me something to amuse myself with for the day more than
I had bargain’d for, or could have enter’d either into his head or mine.

He had brought the little print of butter upon a currant leaf: and as the
morning was warm, and he had a good step to bring it, he had begg’d a
sheet of waste paper to put betwixt the currant leaf and his hand.—As
that was plate sufficient, I bade him lay it upon the table as it was;
and as I resolved to stay within all day, I ordered him to call upon the
_traîteur_, to bespeak my dinner, and leave me to breakfast by myself.

When I had finished the butter, I threw the currant-leaf out of the
window, and was going to do the same by the waste paper;—but stopping to
read a line first, and that drawing me on to a second and third,—I
thought it better worth; so I shut the window, and drawing a chair up to
it, I sat down to read it.

It was in the old French of Rabelais’s time, and for aught I know might
have been wrote by him:—it was moreover in a Gothic letter, and that so
faded and gone off by damps and length of time, it cost me infinite
trouble to make anything of it.—I threw it down; and then wrote a letter
to Eugenius;—then I took it up again, and embroiled my patience with it
afresh;—and then to cure that, I wrote a letter to Eliza.—Still it kept
hold of me; and the difficulty of understanding it increased but the
desire.

I got my dinner; and after I had enlightened my mind with a bottle of
Burgundy; I at it again,—and, after two or three hours poring upon it,
with almost as deep attention as ever Gruter or Jacob Spon did upon a
nonsensical inscription, I thought I made sense of it; but to make sure
of it, the best way, I imagined, was to turn it into English, and see how
it would look then;—so I went on leisurely, as a trifling man does,
sometimes writing a sentence,—then taking a turn or two,—and then looking
how the world went, out of the window; so that it was nine o’clock at
night before I had done it.—I then began and read it as follows.




THE FRAGMENT.
PARIS.


—NOW, as the notary’s wife disputed the point with the notary with too
much heat,—I wish, said the notary, (throwing down the parchment) that
there was another notary here only to set down and attest all this.—

—And what would you do then, Monsieur? said she, rising hastily up.—The
notary’s wife was a little fume of a woman, and the notary thought it
well to avoid a hurricane by a mild reply.—I would go, answered he, to
bed.—You may go to the devil, answer’d the notary’s wife.

Now there happening to be but one bed in the house, the other two rooms
being unfurnished, as is the custom at Paris, and the notary not caring
to lie in the same bed with a woman who had but that moment sent him pell
mell to the devil, went forth with his hat and cane and short cloak, the
night being very windy, and walk’d out, ill at ease, towards the Pont
Neuf.

Of all the bridges which ever were built, the whole world who have pass’d
over the Pont Neuf must own, that it is the noblest,—the finest,—the
grandest,—the lightest,—the longest,—the broadest, that ever conjoin’d
land and land together upon the face of the terraqueous globe.

    [_By this it seems as if the author of the fragment had not been a
                               Frenchman_.]

The worst fault which divines and the doctors of the Sorbonne can allege
against it is, that if there is but a capfull of wind in or about Paris,
’tis more blasphemously _sacre Dieu’d_ there than in any other aperture
of the whole city,—and with reason good and cogent, Messieurs; for it
comes against you without crying _garde d’eau_, and with such
unpremeditable puffs, that of the few who cross it with their hats on,
not one in fifty but hazards two livres and a half, which is its full
worth.

The poor notary, just as he was passing by the sentry, instinctively
clapp’d his cane to the side of it, but in raising it up, the point of
his cane catching hold of the loop of the sentinel’s hat, hoisted it over
the spikes of the ballustrade clear into the Seine.—

—’_Tis an ill wind_, said a boatman, who catched it, _which blows nobody
any good_.

The sentry, being a Gascon, incontinently twirled up his whiskers, and
levell’d his arquebuss.

Arquebusses in those days went off with matches; and an old woman’s paper
lantern at the end of the bridge happening to be blown out, she had
borrow’d the sentry’s match to light it:—it gave a moment’s time for the
Gascon’s blood to run cool, and turn the accident better to his
advantage.—’_Tis an ill wind_, said he, catching off the notary’s castor,
and legitimating the capture with the boatman’s adage.

The poor notary crossed the bridge, and passing along the Rue de Dauphine
into the fauxbourgs of St. Germain, lamented himself as he walked along
in this manner:—

Luckless man that I am! said the notary, to be the sport of hurricanes
all my days:—to be born to have the storm of ill language levell’d
against me and my profession wherever I go; to be forced into marriage by
the thunder of the church to a tempest of a woman;—to be driven forth out
of my house by domestic winds, and despoil’d of my castor by pontific
ones!—to be here, bareheaded, in a windy night, at the mercy of the ebbs
and flows of accidents!—Where am I to lay my head?—Miserable man! what
wind in the two-and-thirty points of the whole compass can blow unto
thee, as it does to the rest of thy fellow-creatures, good?

As the notary was passing on by a dark passage, complaining in this sort,
a voice call’d out to a girl, to bid her run for the next notary.—Now the
notary being the next, and availing himself of his situation, walk’d up
the passage to the door, and passing through an old sort of a saloon, was
usher’d into a large chamber, dismantled of everything but a long
military pike,—a breastplate,—a rusty old sword, and bandoleer, hung up,
equidistant, in four different places against the wall.

An old personage who had heretofore been a gentleman, and unless decay of
fortune taints the blood along with it, was a gentleman at that time, lay
supporting his head upon his hand in his bed; a little table with a taper
burning was set close beside it, and close by the table was placed a
chair:—the notary sat him down in it; and pulling out his inkhorn and a
sheet or two of paper which he had in his pocket, he placed them before
him; and dipping his pen in his ink, and leaning his breast over the
table, he disposed everything to make the gentleman’s last will and
testament.

Alas!  _Monsieur le Notaire_, said the gentleman, raising himself up a
little, I have nothing to bequeath, which will pay the expense of
bequeathing, except the history of myself, which I could not die in
peace, unless I left it as a legacy to the world: the profits arising out
of it I bequeath to you for the pains of taking it from me.—It is a story
so uncommon, it must be read by all mankind;—it will make the fortunes of
your house.—The notary dipp’d his pen into his inkhorn.—Almighty Director
of every event in my life! said the old gentleman, looking up earnestly,
and raising his hands towards heaven,—Thou, whose hand has led me on
through such a labyrinth of strange passages down into this scene of
desolation, assist the decaying memory of an old, infirm, and
broken-hearted man;—direct my tongue by the spirit of thy eternal truth,
that this stranger may set down nought but what is written in that BOOK,
from whose records, said he, clasping his hands together, I am to be
condemn’d or acquitted!—the notary held up the point of his pen betwixt
the taper and his eye.—

It is a story, _Monsieur le Notaire_, said the gentleman, which will
rouse up every affection in nature;—it will kill the humane, and touch
the heart of Cruelty herself with pity.—

—The notary was inflamed with a desire to begin, and put his pen a third
time into his ink-horn—and the old gentleman, turning a little more
towards the notary, began to dictate his story in these words:—

—And where is the rest of it, La Fleur? said I, as he just then enter’d
the room.




THE FRAGMENT, AND THE BOUQUET. {648}
PARIS.


WHEN La Fleur came up close to the table, and was made to comprehend what
I wanted, he told me there were only two other sheets of it, which he had
wrapped round the stalks of a bouquet to keep it together, which he had
presented to the demoiselle upon the boulevards.—Then prithee, La Fleur,
said I, step back to her to the Count de B—’s hotel, and see if thou
canst get it.—There is no doubt of it, said La Fleur;—and away he flew.

In a very little time the poor fellow came back quite out of breath, with
deeper marks of disappointment in his looks than could arise from the
simple irreparability of the fragment.  _Juste Ciel_! in less than two
minutes that the poor fellow had taken his last tender farewell of
her—his faithless mistress had given his _gage d’amour_ to one of the
Count’s footmen,—the footman to a young sempstress,—and the sempstress to
a fiddler, with my fragment at the end of it.—Our misfortunes were
involved together:—I gave a sigh,—and La Fleur echoed it back again to my
ear.

—How perfidious! cried La Fleur.—How unlucky! said I.

—I should not have been mortified, Monsieur, quoth La Fleur, if she had
lost it.—Nor I, La Fleur, said I, had I found it.

Whether I did or no will be seen hereafter.




THE ACT OF CHARITY.
PARIS.


THE man who either disdains or fears to walk up a dark entry may be an
excellent good man, and fit for a hundred things, but he will not do to
make a good Sentimental Traveller.—I count little of the many things I
see pass at broad noonday, in large and open streets.—Nature is shy, and
hates to act before spectators; but in such an unobserved corner you
sometimes see a single short scene of hers worth all the sentiments of a
dozen French plays compounded together,—and yet they are absolutely
fine;—and whenever I have a more brilliant affair upon my hands than
common, as they suit a preacher just as well as a hero, I generally make
my sermon out of ’em;—and for the text,—“Cappadocia, Pontus and Asia,
Phrygia and Pamphylia,”—is as good as any one in the Bible.

There is a long dark passage issuing out from the Opera Comique into a
narrow street; ’tis trod by a few who humbly wait for a _fiacre_, {649}
or wish to get off quietly o’foot when the opera is done.  At the end of
it, towards the theatre, ’tis lighted by a small candle, the light of
which is almost lost before you get half-way down, but near the door—’tis
more for ornament than use: you see it as a fixed star of the least
magnitude; it burns,—but does little good to the world, that we know of.

In returning along this passage, I discerned, as I approached within five
or six paces of the door, two ladies standing arm-in-arm with their backs
against the wall, waiting, as I imagined, for a _fiacre_;—as they were
next the door, I thought they had a prior right; so edged myself up
within a yard or little more of them, and quietly took my stand.—I was in
black, and scarce seen.

The lady next me was a tall lean figure of a woman, of about thirty-six;
the other of the same size and make, of about forty: there was no mark of
wife or widow in any one part of either of them;—they seem’d to be two
upright vestal sisters, unsapped by caresses, unbroke in upon by tender
salutations.—I could have wish’d to have made them happy:—their happiness
was destin’d that night, to come from another quarter.

A low voice, with a good turn of expression, and sweet cadence at the end
of it, begg’d for a twelve-sous piece betwixt them, for the love of
heaven.  I thought it singular that a beggar should fix the quota of an
alms—and that the sum should be twelve times as much as what is usually
given in the dark.—They both seemed astonished at it as much as
myself.—Twelve sous! said one.—A twelve-sous piece! said the other,—and
made no reply.

The poor man said, he knew not how to ask less of ladies of their rank;
and bow’d down his head to the ground.

Poo! said they,—we have no money.

The beggar remained silent for a moment or two, and renew’d his
supplication.

—Do not, my fair young ladies, said he, stop your good ears against
me.—Upon my word, honest man! said the younger, we have no change.—Then
God bless you, said the poor man, and multiply those joys which you can
give to others without change!—I observed the elder sister put her hand
into her pocket.—I’ll see, said she, if I have a sous.  A sous! give
twelve, said the supplicant; Nature has been bountiful to you, be
bountiful to a poor man.

—I would friend, with all my heart, said the younger, if I had it.

My fair charitable! said he, addressing himself to the elder,—what is it
but your goodness and humanity which makes your bright eyes so sweet,
that they outshine the morning even in this dark passage? and what was it
which made the Marquis de Santerre and his brother say so much of you
both as they just passed by?

The two ladies seemed much affected; and impulsively, at the same time
they both put their hands into their pocket, and each took out a
twelve-sous piece.

The contest betwixt them and the poor supplicant was no more;—it was
continued betwixt themselves, which of the two should give the
twelve-sous piece in charity;—and, to end the dispute, they both gave it
together, and the man went away.




THE RIDDLE EXPLAINED.
PARIS.


I STEPPED hastily after him: it was the very man whose success in asking
charity of the women before the door of the hotel had so puzzled me;—and
I found at once his secret, or at least the basis of it:—’twas flattery.

Delicious essence! how refreshing art thou to Nature! how strongly are
all its powers and all its weaknesses on thy side! how sweetly dost thou
mix with the blood, and help it through the most difficult and tortuous
passages to the heart!

The poor man, as he was not straiten’d for time, had given it here in a
larger dose: ’tis certain he had a way of bringing it into a less form,
for the many sudden cases he had to do with in the streets: but how he
contrived to correct, sweeten, concentre, and qualify it,—I vex not my
spirit with the enquiry;—it is enough the beggar gained two twelve-sous
pieces—and they can best tell the rest, who have gained much greater
matters by it.




PARIS.


WE get forwards in the world, not so much by doing services, as receiving
them; you take a withering twig, and put it in the ground; and then you
water it, because you have planted it.

Monsieur le Count de B—, merely because he had done me one kindness in
the affair of my passport, would go on and do me another, the few days he
was at Paris, in making me known to a few people of rank; and they were
to present me to others, and so on.

I had got master of my _secret_ just in time to turn these honours to
some little account; otherwise, as is commonly the case, I should have
dined or supp’d a single time or two round, and then, by _translating_
French looks and attitudes into plain English, I should presently have
seen, that I had hold of the _couvert_ {652} of some more entertaining
guest; and in course should have resigned all my places one after
another, merely upon the principle that I could not keep them.—As it was,
things did not go much amiss.

I had the honour of being introduced to the old Marquis de B—: in days of
yore he had signalized himself by some small feats of chivalry in the
_Cour d’Amour_, and had dress’d himself out to the idea of tilts and
tournaments ever since.—The Marquis de B— wish’d to have it thought the
affair was somewhere else than in his brain.  “He could like to take a
trip to England,” and asked much of the English ladies.—Stay where you
are, I beseech you, Monsieur le Marquis, said I.—_Les Messieurs Anglois_
can scarce get a kind look from them as it is.—The Marquis invited me to
supper.

Monsieur P—, the farmer-general, was just as inquisitive about our taxes.
They were very considerable, he heard.—If we knew but how to collect
them, said I, making him a low bow.

I could never have been invited to Mons. P—’s concerts upon any other
terms.

I had been misrepresented to Madame de Q— as an _esprit_.—Madame de Q—
was an _esprit_ herself: she burnt with impatience to see me, and hear me
talk.  I had not taken my seat, before I saw she did not care a sous
whether I had any wit or no;—I was let in, to be convinced she had.  I
call heaven to witness I never once opened the door of my lips.

Madame de V— vow’d to every creature she met—“She had never had a more
improving conversation with a man in her life.”

There are three epochas in the empire of a French woman.—She is
coquette,—then deist,—then _dévote_: the empire during these is never
lost,—she only changes her subjects when thirty-five years and more have
unpeopled her dominion of the slaves of love, she re-peoples it with
slaves of infidelity,—and then with the slaves of the church.

Madame de V— was vibrating betwixt the first of those epochas: the colour
of the rose was fading fast away;—she ought to have been a deist five
years before the time I had the honour to pay my first visit.

She placed me upon the same sofa with her, for the sake of disputing the
point of religion more closely.—In short Madame de V— told me she
believed nothing.—I told Madame de V— it might be her principle, but I
was sure it could not be her interest to level the outworks, without
which I could not conceive how such a citadel as hers could be
defended;—that there was not a more dangerous thing in the world than for
a beauty to be a deist;—that it was a debt I owed my creed not to conceal
it from her;—that I had not been five minutes sat upon the sofa beside
her, but I had begun to form designs;—and what is it, but the sentiments
of religion, and the persuasion they had excited in her breast, which
could have check’d them as they rose up?

We are not adamant, said I, taking hold of her hand;—and there is need of
all restraints, till age in her own time steals in and lays them on
us.—But my dear lady, said I, kissing her hand,—’tis too—too soon.

I declare I had the credit all over Paris of unperverting Madame de
V—.—She affirmed to Monsieur D— and the Abbé M—, that in one half hour I
had said more for revealed religion, than all their Encyclopædia had said
against it.—I was listed directly into Madame de V—’s _coterie_;—and she
put off the epocha of deism for two years.

I remember it was in this _coterie_, in the middle of a discourse, in
which I was showing the necessity of a _first_ cause, when the young
Count de Faineant took me by the hand to the farthest corner of the room,
to tell me my _solitaire_ was pinn’d too straight about my neck.—It
should be _plus badinant_, said the Count, looking down upon his own;—but
a word, Monsieur Yorick, _to the wise_—

And _from the wise_, Monsieur le Count, replied I, making him a bow,—_is
enough_.

The Count de Faineant embraced me with more ardour than ever I was
embraced by mortal man.

For three weeks together I was of every man’s opinion I met.—_Pardi_! _ce
Monsieur Yorick a autant d’esprit que nous autres_.—_Il raisonne bien_,
said another.—_C’est un bon enfant_, said a third.—And at this price I
could have eaten and drank and been merry all the days of my life at
Paris; but ’twas a dishonest _reckoning_;—I grew ashamed of it.—It was
the gain of a slave;—every sentiment of honour revolted against it;—the
higher I got, the more was I forced upon my _beggarly system_;—the better
the _coterie_,—the more children of Art;—I languish’d for those of
Nature: and one night, after a most vile prostitution of myself to half a
dozen different people, I grew sick,—went to bed;—order’d La Fleur to get
me horses in the morning to set out for Italy.




MARIA.
MOULINES.


I NEVER felt what the distress of plenty was in any one shape till
now,—to travel it through the Bourbonnois, the sweetest part of
France,—in the heyday of the vintage, when Nature is pouring her
abundance into every one’s lap, and every eye is lifted up,—a journey,
through each step of which Music beats time to _Labour_, and all her
children are rejoicing as they carry in their clusters: to pass through
this with my affections flying out, and kindling at every group before
me,—and every one of them was pregnant with adventures.—

Just heaven!—it would fill up twenty volumes;—and alas! I have but a few
small pages left of this to crowd it into,—and half of these must be
taken up with the poor Maria my friend, Mr. Shandy, met with near
Moulines.

The story he had told of that disordered maid affected me not a little in
the reading; but when I got within the neighbourhood where she lived, it
returned so strong into the mind, that I could not resist an impulse
which prompted me to go half a league out of the road, to the village
where her parents dwelt, to enquire after her.

’Tis going, I own, like the Knight of the Woeful Countenance in quest of
melancholy adventures.  But I know not how it is, but I am never so
perfectly conscious of the existence of a soul within me, as when I am
entangled in them.

The old mother came to the door; her looks told me the story before she
open’d her mouth.—She had lost her husband; he had died, she said, of
anguish, for the loss of Maria’s senses, about a month before.—She had
feared at first, she added, that it would have plunder’d her poor girl of
what little understanding was left;—but, on the contrary, it had brought
her more to herself:—still, she could not rest.—Her poor daughter, she
said, crying, was wandering somewhere about the road.

Why does my pulse beat languid as I write this? and what made La Fleur,
whose heart seem’d only to be tuned to joy, to pass the back of his hand
twice across his eyes, as the woman stood and told it?  I beckoned to the
postilion to turn back into the road.

When we had got within half a league of Moulines, at a little opening in
the road leading to a thicket, I discovered poor Maria sitting under a
poplar.  She was sitting with her elbow in her lap, and her head leaning
on one side within her hand:—a small brook ran at the foot of the tree.

I bid the postilion go on with the chaise to Moulines—and La Fleur to
bespeak my supper;—and that I would walk after him.

She was dress’d in white, and much as my friend described her, except
that her hair hung loose, which before was twisted within a silk net.—She
had superadded likewise to her jacket, a pale green riband, which fell
across her shoulder to the waist; at the end of which hung her pipe.—Her
goat had been as faithless as her lover; and she had got a little dog in
lieu of him, which she had kept tied by a string to her girdle: as I
looked at her dog, she drew him towards her with the string.—“Thou shalt
not leave me, Sylvio,” said she.  I look’d in Maria’s eyes and saw she
was thinking more of her father than of her lover, or her little goat;
for, as she utter’d them, the tears trickled down her cheeks.

I sat down close by her; and Maria let me wipe them away as they fell,
with my handkerchief.—I then steep’d it in my own,—and then in hers,—and
then in mine,—and then I wip’d hers again;—and as I did it, I felt such
undescribable emotions within me, as I am sure could not be accounted for
from any combinations of matter and motion.

I am positive I have a soul; nor can all the books with which
materialists have pester’d the world ever convince me to the contrary.




MARIA.


WHEN Maria had come a little to herself, I ask’d her if she remembered a
pale thin person of a man, who had sat down betwixt her and her goat
about two years before?  She said she was unsettled much at that time,
but remembered it upon two accounts:—that ill as she was, she saw the
person pitied her; and next, that her goat had stolen his handkerchief,
and she had beat him for the theft;—she had wash’d it, she said, in the
brook, and kept it ever since in her pocket to restore it to him in case
she should ever see him again, which, she added, he had half promised
her.  As she told me this, she took the handkerchief out of her pocket to
let me see it; she had folded it up neatly in a couple of vine leaves,
tied round with a tendril;—on opening it, I saw an S. marked in one of
the corners.

She had since that, she told me, stray’d as far as Rome, and walk’d round
St. Peter’s once,—and return’d back;—that she found her way alone across
the Apennines;—had travell’d over all Lombardy, without money,—and
through the flinty roads of Savoy without shoes:—how she had borne it,
and how she had got supported, she could not tell;—but _God tempers the
wind_, said Maria, _to the shorn lamb_.

Shorn indeed! and to the quick, said I: and wast thou in my own land,
where I have a cottage, I would take thee to it, and shelter thee: thou
shouldst eat of my own bread and drink of my own cup;—I would be kind to
thy Sylvio;—in all thy weaknesses and wanderings I would seek after thee
and bring thee back;—when the sun went down I would say my prayers: and
when I had done thou shouldst play thy evening song upon thy pipe, nor
would the incense of my sacrifice be worse accepted for entering heaven
along with that of a broken heart!

Nature melted within me, as I utter’d this; and Maria observing, as I
took out my handkerchief, that it was steep’d too much already to be of
use, would needs go wash it in the stream.—And where will you dry it,
Maria? said I.—I’ll dry it in my bosom, said she:—’twill do me good.

And is your heart still so warm, Maria? said I.

I touch’d upon the string on which hung all her sorrows:—she look’d with
wistful disorder for some time in my face; and then, without saying any
thing, took her pipe and play’d her service to the Virgin.—The string I
had touched ceased to vibrate;—in a moment or two Maria returned to
herself,—let her pipe fall,—and rose up.

And where are you going, Maria? said I.—She said, to Moulines.—Let us go,
said I, together.—Maria put her arm within mine, and lengthening the
string, to let the dog follow,—in that order we enter’d Moulines.




MARIA.
MOULINES.


THOUGH I hate salutations and greetings in the market-place, yet, when we
got into the middle of this, I stopp’d to take my last look and last
farewell of Maria.

Maria, though not tall, was nevertheless of the first order of fine
forms:—affliction had touched her looks with something that was scarce
earthly;—still she was feminine;—and so much was there about her of all
that the heart wishes, or the eye looks for in woman, that could the
traces be ever worn out of her brain, and those of Eliza out of mine, she
should _not only eat of my bread and drink of my own cup_, but Maria
should lie in my bosom, and be unto me as a daughter.

Adieu, poor luckless maiden!—Imbibe the oil and wine which the compassion
of a stranger, as he journeyeth on his way, now pours into thy
wounds;—the Being, who has twice bruised thee, can only bind them up for
ever.




THE BOURBONNNOIS.


THERE was nothing from which I had painted out for my self so joyous a
riot of the affections, as in this journey in the vintage, through this
part of France; but pressing through this gate, of sorrow to it, my
sufferings have totally unfitted me.  In every scene of festivity, I saw
Maria in the background of the piece, sitting pensive under her poplar;
and I had got almost to Lyons before I was able to cast a shade across
her.

—Dear Sensibility! source inexhausted of all that’s precious in our joys,
or costly in our sorrows! thou chainest thy martyr down upon his bed of
straw—and ’tis thou who lift’st him up to Heaven!—Eternal Fountain of our
feelings!—’tis here I trace thee—and this is thy “_divinity which stirs
within me_;”—not that, in some sad and sickening moments, “_my soul
shrinks back upon herself_, _and startles at destruction_;”—mere pomp of
words!—but that I feel some generous joys and generous cares beyond
myself;—all comes from thee, great—great SENSORIUM of the world! which
vibrates, if a hair of our heads but falls upon the ground, in the
remotest desert of thy creation.—Touch’d with thee, Eugenius draws my
curtain when I languish—hears my tale of symptoms, and blames the weather
for the disorder of his nerves.  Thou giv’st a portion of it sometimes to
the roughest peasant who traverses the bleakest mountains;—he finds the
lacerated lamb of another’s flock.—This moment I behold him leaning with
his head against his crook, with piteous inclination looking down upon
it!—Oh! had I come one moment sooner! it bleeds to death!—his gentle
heart bleeds with it.—

Peace to thee, generous swain!—I see thou walkest off with anguish,—but
thy joys shall balance it;—for, happy is thy cottage,—and happy is the
sharer of it,—and happy are the lambs which sport about you!




THE SUPPER.


A SHOE coming loose from the fore foot of the thill-horse, at the
beginning of the ascent of mount Taurira, the postilion dismounted,
twisted the shoe off, and put it in his pocket; as the ascent was of five
or six miles, and that horse our main dependence, I made a point of
having the shoe fastened on again, as well as we could; but the postilion
had thrown away the nails, and the hammer in the chaise box being of no
great use without them, I submitted to go on.

He had not mounted half a mile higher, when, coming to a flinty piece of
road, the poor devil lost a second shoe, and from off his other fore
foot.  I then got out of the chaise in good earnest; and seeing a house
about a quarter of a mile to the left hand, with a great deal to do I
prevailed upon the postilion to turn up to it.  The look of the house,
and of every thing about it, as we drew nearer, soon reconciled me to the
disaster.—It was a little farm-house, surrounded with about twenty acres
of vineyard, about as much corn;—and close to the house, on one side, was
a _potagerie_ of an acre and a half, full of everything which could make
plenty in a French peasant’s house;—and, on the other side, was a little
wood, which furnished wherewithal to dress it.  It was about eight in the
evening when I got to the house—so I left the postilion to manage his
point as he could;—and, for mine, I walked directly into the house.

The family consisted of an old grey-headed man and his wife, with five or
six sons and sons-in-law, and their several wives, and a joyous genealogy
out of them.

They were all sitting down together to their lentil-soup; a large wheaten
loaf was in the middle of the table; and a flagon of wine at each end of
it promised joy through the stages of the repast:—’twas a feast of love.

The old man rose up to meet me, and with a respectful cordiality would
have me sit down at the table; my heart was set down the moment I enter’d
the room; so I sat down at once like a son of the family; and to invest
myself in the character as speedily as I could, I instantly borrowed the
old man’s knife, and taking up the loaf, cut myself a hearty luncheon;
and, as I did it, I saw a testimony in every eye, not only of an honest
welcome, but of a welcome mix’d with thanks that I had not seem’d to
doubt it.

Was it this? or tell me, Nature, what else it was that made this morsel
so sweet,—and to what magic I owe it, that the draught I took of their
flagon was so delicious with it, that they remain upon my palate to this
hour?

If the supper was to my taste,—the grace which followed it was much more
so.




THE GRACE.


WHEN supper was over, the old man gave a knock upon the table with the
haft of his knife, to bid them prepare for the dance: the moment the
signal was given, the women and girls ran altogether into a back
apartment to tie up their hair,—and the young men to the door to wash
their faces, and change their sabots; and in three minutes every soul was
ready upon a little esplanade before the house to begin.—The old man and
his wife came out last, and placing me betwixt them, sat down upon a sofa
of turf by the door.

The old man had some fifty years ago been no mean performer upon the
_vielle_,—and at the age he was then of, touch’d it well enough for the
purpose.  His wife sung now and then a little to the tune,—then
intermitted,—and join’d her old man again, as their children and
grand-children danced before them.

It was not till the middle of the second dance, when, from some pauses in
the movements, wherein they all seemed to look up, I fancied I could
distinguish an elevation of spirit different from that which is the cause
or the effect of simple jollity.  In a word, I thought I beheld
_Religion_ mixing in the dance:—but, as I had never seen her so engaged,
I should have look’d upon it now as one of the illusions of an
imagination which is eternally misleading me, had not the old man, as
soon as the dance ended, said, that this was their constant way; and that
all his life long he had made it a rule, after supper was over, to call
out his family to dance and rejoice; believing, he said, that a cheerful
and contented mind was the best sort of thanks to heaven that an
illiterate peasant could pay,—

Or a learned prelate either, said I.




THE CASE OF DELICACY.


WHEN you have gained the top of Mount Taurira, you run presently down to
Lyons:—adieu, then, to all rapid movements!  ’Tis a journey of caution;
and it fares better with sentiments, not to be in a hurry with them; so I
contracted with a _voiturin_ to take his time with a couple of mules, and
convoy me in my own chaise safe to Turin, through Savoy.

Poor, patient, quiet, honest people! fear not: your poverty, the treasury
of your simple virtues, will not be envied you by the world, nor will
your valleys be invaded by it.—Nature! in the midst of thy disorders,
thou art still friendly to the scantiness thou hast created: with all thy
great works about thee, little hast thou left to give, either to the
scythe or to the sickle;—but to that little thou grantest safety and
protection; and sweet are the dwellings which stand so shelter’d.

Let the way-worn traveller vent his complaints upon the sudden turns and
dangers of your roads,—your rocks,—your precipices;—the difficulties of
getting up,—the horrors of getting down,—mountains impracticable,—and
cataracts, which roll down great stones from their summits, and block his
road up.—The peasants had been all day at work in removing a fragment of
this kind between St. Michael and Madane; and, by the time my _voiturin_
got to the place, it wanted full two hours of completing before a passage
could any how be gain’d: there was nothing but to wait with
patience;—’twas a wet and tempestuous night; so that by the delay, and
that together, the _voiturin_ found himself obliged to put up five miles
short of his stage at a little decent kind of an inn by the roadside.

I forthwith took possession of my bedchamber—got a good fire—order’d
supper; and was thanking heaven it was no worse, when a _voiturin_ arrived
with a lady in it and her servant maid.

As there was no other bed-chamber in the house, the hostess,—without much
nicety, led them into mine, telling them, as she usher’d them in, that
there was nobody in it but an English gentleman;—that there were two good
beds in it, and a closet within the room which held another.  The accent
in which she spoke of this third bed, did not say much for it;—however,
she said there were three beds and but three people, and she durst say,
the gentleman would do anything to accommodate matters.—I left not the
lady a moment to make a conjecture about it—so instantly made a
declaration that I would do anything in my power.

As this did not amount to an absolute surrender of my bed-chamber, I
still felt myself so much the proprietor, as to have a right to do the
honours of it;—so I desired the lady to sit down,—pressed her into the
warmest seat,—called for more wood,—desired the hostess to enlarge the
plan of the supper, and to favour us with the very best wine.

The lady had scarce warm’d herself five minutes at the fire, before she
began to turn her head back, and give a look at the beds; and the oftener
she cast her eyes that way, the more they return’d perplexd;—I felt for
her—and for myself: for in a few minutes, what by her looks, and the case
itself, I found myself as much embarrassed as it was possible the lady
could be herself.

That the beds we were to lie in were in one and the same room, was enough
simply by itself to have excited all this;—but the position of them, for
they stood parallel, and so very close to each other as only to allow
space for a small wicker chair betwixt them, rendered the affair still
more oppressive to us;—they were fixed up moreover near the fire; and the
projection of the chimney on one side, and a large beam which cross’d the
room on the other, formed a kind of recess for them that was no way
favourable to the nicety of our sensations:—if anything could have added
to it, it was that the two beds were both of them so very small, as to
cut us off from every idea of the lady and the maid lying together; which
in either of them, could it have been feasible, my lying beside them,
though a thing not to be wish’d, yet there was nothing in it so terrible
which the imagination might not have pass’d over without torment.

As for the little room within, it offer’d little or no consolation to us:
’twas a damp, cold closet, with a half dismantled window-shutter, and
with a window which had neither glass nor oil paper in it to keep out the
tempest of the night.  I did not endeavour to stifle my cough when the
lady gave a peep into it; so it reduced the case in course to this
alternative—That the lady should sacrifice her health to her feelings,
and take up with the closet herself, and abandon the bed next mine to her
maid,—or that the girl should take the closet, &c., &c.

The lady was a Piedmontese of about thirty, with a glow of health in her
cheeks.  The maid was a Lyonoise of twenty, and as brisk and lively a
French girl as ever moved.—There were difficulties every way,—and the
obstacle of the stone in the road, which brought us into the distress,
great as it appeared whilst the peasants were removing it, was but a
pebble to what lay in our ways now.—I have only to add, that it did not
lessen the weight which hung upon our spirits, that we were both too
delicate to communicate what we felt to each other upon the occasion.

We sat down to supper; and had we not had more generous wine to it than a
little inn in Savoy could have furnish’d, our tongues had been tied up,
till necessity herself had set them at liberty;—but the lady having a few
bottles of Burgundy in her voiture, sent down her _fille de chambre_ for
a couple of them; so that by the time supper was over, and we were left
alone, we felt ourselves inspired with a strength of mind sufficient to
talk, at least, without reserve upon our situation.  We turn’d it every
way, and debated and considered it in all kinds of lights in the course
of a two hours’ negotiation; at the end of which the articles were
settled finally betwixt us, and stipulated for in form and manner of a
treaty of peace,—and I believe with as much religion and good faith on
both sides as in any treaty which has yet had the honour of being handed
down to posterity.

They were as follow:—

First, as the right of the bed-chamber is in Monsieur,—and he thinking
the bed next to the fire to be the warmest, he insists upon the
concession on the lady’s side of taking up with it.

Granted, on the part of Madame; with a proviso, That as the curtains of
that bed are of a flimsy transparent cotton, and appear likewise too
scanty to draw close, that the _fille de chambre_ shall fasten up the
opening, either by corking pins, or needle and thread, in such manner as
shall be deem’d a sufficient barrier on the side of Monsieur.

2dly.  It is required on the part of Madame, that Monsieur shall lie the
whole night through in his _robe de chambre_.

Rejected: inasmuch as Monsieur is not worth a _robe de chambre_; he
having nothing in his portmanteau but six shirts and a black silk pair of
breeches.

The mentioning the silk pair of breeches made an entire change of the
article,—for the breeches were accepted as an equivalent for the _robe de
chambre_; and so it was stipulated and agreed upon, that I should lie in
my black silk breeches all night.

3dly.  It was insisted upon and stipulated for by the lady, that after
Monsieur was got to bed, and the candle and fire extinguished, that
Monsieur should not speak one single word the whole night.

Granted; provided Monsieur’s saying his prayers might not be deemed an
infraction of the treaty.

There was but one point forgot in this treaty, and that was the manner in
which the lady and myself should be obliged to undress and get to
bed;—there was but one way of doing it, and that I leave to the reader to
devise; protesting as I do it, that if it is not the most delicate in
nature, ’tis the fault of his own imagination,—against which this is not
my first complaint.

Now, when we were got to bed, whether it was the novelty of the
situation, or what it was, I know not; but so it was, I could not shut my
eyes; I tried this side, and that, and turn’d and turn’d again, till a
full hour after midnight; when Nature and patience both wearing out,—O,
my God! said I.

—You have broke the treaty, Monsieur, said the lady, who had no more
slept than myself.—I begg’d a thousand pardons—but insisted it was no
more than an ejaculation.  She maintained ’twas an entire infraction of
the treaty—I maintain’d it was provided for in the clause of the third
article.

The lady would by no means give up her point, though she weaken’d her
barrier by it; for in the warmth of the dispute, I could hear two or
three corking pins fall out of the curtain to the ground.

Upon my word and honour, Madame, said I,—stretching my arm out of bed by
way of asseveration.—

(I was going to have added, that I would not have trespassed against the
remotest idea of decorum for the world);—

But the _fille de chambre_ hearing there were words between us, and
fearing that hostilities would ensue in course, had crept silently out of
her closet, and it being totally dark, had stolen so close to our beds,
that she had got herself into the narrow passage which separated them,
and had advanced so far up as to be in a line betwixt her mistress and
me:—

So that when I stretch’d out my hand I caught hold of the _fille de
chambre’s_—

                                * * * * *

                                 THE END

                                * * * * *




FOOTNOTES.


{557}  All the effects of strangers (Swiss and Scotch excepted) dying in
France, are seized by virtue of this law, though the heir be upon the
spot—the profit of these contingencies being farmed, there is no redress.

{562}  A chaise, so called, in France, from its holding but one person.

{580}  Vide S—’s Travels: [_i.e._ Dr. Smollett’s “Travels through France
and Italy.”—ED.]

{588}  Post-horse.

{648}  Nosegay.

{649}  Hackney coach.

{652}  Plate, napkin, knife, fork and spoon.





\end{document}
